\documentclass[aps,onecolumn]{revtex4}
\usepackage{graphicx}
\usepackage{amssymb,amsfonts,amsmath,amsthm}
\usepackage{chemarr}
\usepackage{bm}
\usepackage{pslatex}
\usepackage{mathptmx}
\usepackage{xfrac}
\usepackage{xcolor}

\newcommand{\mychem}[1]{\mathtt{#1}}
\begin{document}

\section{Equations}
\subsection{Assumed mechanism}
We suppose the following mechanism
\begin{equation}
	\mychem{Li}^{x}_{out} + E \xrightleftharpoons[k_d^{x}]{k_a^{x}}
	\lbrace\mychem{Li}^xE\rbrace_{out} 
	\xrightleftharpoons[k_r^{x}]{k_f^{x}} 
	\lbrace\mychem{Li}^xE\rbrace_{in}
	\xrightarrow{k_t^x} E + \mychem{Li}^{x}_{in}
\end{equation}

\subsection{Full differential system}
\begin{equation}
	\left\lbrace
	\begin{array}{ccl}
	\partial_t [\mychem{Li}^6_{in}] & = &k_t^6[\lbrace\mychem{Li}^6E\rbrace_{in}  \\
	\partial_t [\mychem{Li}^7_{in}] & = &k_t^7[\lbrace\mychem{Li}^7E\rbrace_{in}  \\
	\\
	\partial_t [E] & = &
	-\left(k_a^6[\mychem{Li}^6_{out}]+k_a^7[\mychem{Li}^7_{out}]\right)[E] 
	+\left(k_d^6[\lbrace\mychem{Li}^6E\rbrace_{out}]+k_d^7[\lbrace\mychem{Li}^7E\rbrace_{out}]\right)
	+\left(k_t^6[\lbrace\mychem{Li}^6E\rbrace_{in} ]+k_t^7[\lbrace\mychem{Li}^7E\rbrace_{in} ]\right)
	\\
	\\
	\partial_t[\lbrace\mychem{Li}^6E\rbrace_{out}] & = & 
	k_a^6[\mathtt{Li}^6_{out}][E] - (k_d^6+k_f^6) [\lbrace\mychem{Li}^6E\rbrace_{out}]
	+k_r^6[\lbrace\mychem{Li}^6E\rbrace_{in}]
	= k_a^6 \left([\mathtt{Li}^6_{out}][E] -K_m^6 [\lbrace\mychem{Li}^6E\rbrace_{out}] + J_m^6 [\lbrace\mychem{Li}^6E\rbrace_{in}]\right)
	\\
	\partial_t[\lbrace\mychem{Li}^7E\rbrace_{out}] & = & 
	k_a^7[\mathtt{Li}^7_{out}][E] - (k_d^7+k_f^7) [\lbrace\mychem{Li}^7E\rbrace_{out}]
	+k_r^7[\lbrace\mychem{Li}^7E\rbrace_{in}]
	= k_a^7 \left([\mathtt{Li}^7_{out}][E] -K_m^7 [\lbrace\mychem{Li}^7E\rbrace_{out}] + J_m^7 [\lbrace\mychem{Li}^7E\rbrace_{in}\right)
	\\
	\\
	\partial_t[\lbrace\mychem{Li}^6E\rbrace_{in}] & = & 
	k_f^6 [\lbrace\mychem{Li}^6E\rbrace_{out}] - (k_r^6+k_t^6) [\lbrace\mychem{Li}^6E\rbrace_{in}]
	= k_f^6 \left( [\lbrace\mychem{Li}^6E\rbrace_{out}] - \alpha_6 [\lbrace\mychem{Li}^6E\rbrace_{in}]\right)
	\\
	\partial_t[\lbrace\mychem{Li}^7E\rbrace_{in}] & = & 
	k_f^7 [\lbrace\mychem{Li}^7E\rbrace_{out}] - (k_r^7+k_t^7) [\lbrace\mychem{Li}^7E\rbrace_{in}]
	= k_f^7 \left( [\lbrace\mychem{Li}^7E\rbrace_{out}] - \alpha_7 [\lbrace\mychem{Li}^7E\rbrace_{in}]\right)
	\\
	\end{array}
	\right. 
\end{equation}
\centerline{\it Remarks: $\alpha^x$ depends on turnover!!!!}

\subsection{Steady-State Approximation}
After induction time, we assume that intermediates are at there steady state level
and
\begin{equation}
	E_0 = [E] 
	+ [\lbrace\mychem{Li}^6E\rbrace_{out}]
	+ [\lbrace\mychem{Li}^7E\rbrace_{out}]
	+ [\lbrace\mychem{Li}^6E\rbrace_{in}]
	+ [\lbrace\mychem{Li}^7E\rbrace_{in}]
\end{equation}
We first obtain
\begin{equation}
	\left\lbrace
	\begin{array}{rcl}
	~[\lbrace\mychem{Li}^6E\rbrace_{out}] & = & \alpha_6 [\lbrace\mychem{Li}^6E\rbrace_{in}]\\%
	~[\lbrace\mychem{Li}^7E\rbrace_{out}] & = & \alpha_7 [\lbrace\mychem{Li}^7E\rbrace_{in}]\\
	\end{array}
	\right.
\end{equation}
so that we have three remaining equations
\begin{equation}
	\left\lbrace
	\begin{array}{rcl}
	0 & = & [\mathtt{Li}^6_{out}][E] + \left( J_m^6 - \alpha_6 K_m^6 \right) [\lbrace\mychem{Li}^6E\rbrace_{in}]\\
	0  & = & [\mathtt{Li}^7_{out}][E] + \left( J_m^7 - \alpha_7 K_m^7 \right) [\lbrace\mychem{Li}^7E\rbrace_{in}]\\
	E_0 & = & [E] + (1+\alpha_6) [\lbrace\mychem{Li}^6E\rbrace_{in}] + (1+\alpha_7) [\lbrace\mychem{Li}^7E\rbrace_{in}]\\
	\end{array}
	\right.
\end{equation}
We define
\begin{equation}
L_m^x = \alpha_x K_m^x - J_m^x = \dfrac{ (k_f^x+k_d^x) k_t^x + k_d^x k_r^x}{k_a^x k_f^x}
\end{equation}
\centerline{\it Remarks: $L_m^x$ depends on turnover!!!!}
and we get
\begin{equation}
\left\lbrace
	\begin{array}{rcl}
	0   & = & [\mathtt{Li}^6_{out}][E] - L_m^6 [\lbrace\mychem{Li}^6E\rbrace_{in}]\\
	0   & = & [\mathtt{Li}^7_{out}][E] - L_m^7 [\lbrace\mychem{Li}^7E\rbrace_{in}]\\
	E_0 & = & [E] + (1+\alpha_6) [\lbrace\mychem{Li}^6E\rbrace_{in}] + (1+\alpha_7) [\lbrace\mychem{Li}^7E\rbrace_{in}]\\
	\end{array}
	\right.
\end{equation}
Finally, we get
\begin{equation}
	E_0 = [E] \left( 1 + \dfrac{1+\alpha_6}{L_m^6} [\mathtt{Li}^6_{out}] + \dfrac{1+\alpha_7}{L_m^7} [\mathtt{Li}^7_{out}] \right)
\end{equation}
and the intake rates:
\begin{equation}
	\left\lbrace
	\begin{array}{rclcl}
	\partial_t [\mychem{Li}^6_{in}] & = & k_t^6[\lbrace\mychem{Li}^6E\rbrace_{in}] & = & 
	\dfrac{k_t^6E_0[\mathtt{Li}^6_{out}]}{L_m^6\left( 1 + \dfrac{1+\alpha_6}{L_m^6} [\mathtt{Li}^6_{out}] + \dfrac{1+\alpha_7}{L_m^7} [\mathtt{Li}^7_{out}] \right)} \\
	\\
	\partial_t [\mychem{Li}^7_{in}] & = & k_t^7[\lbrace\mychem{Li}^7E\rbrace_{in}] & = & 
	\dfrac{k_t^7E_0[\mathtt{Li}^7_{out}]}{L_m^7\left( 1 + \dfrac{1+\alpha_6}{L_m^6} [\mathtt{Li}^6_{out}] + \dfrac{1+\alpha_7}{L_m^7} [\mathtt{Li}^7_{out}] \right)} \\
	\end{array}
	\right.
\end{equation}


\section{Diffusion}
The diffusion coefficient is inversely proportional to the square root of the mass. The Langevin equation for a species $i$ is
\begin{equation}
	\vec{r}_i\left(t+\delta t\right) = \vec{r}_i\left(t\right)  + \dfrac{\delta t}{m_i\xi_i} \vec{f}(t) + \delta \vec{r}_i^G, \;\;\xi_i = \dfrac{k_BT}{m_iD_i}
\end{equation}
and each component of $\vec{r}_i^G$ is a zero mean Gaussian with a variance of $2D_i\delta t$.\\
The random walk version is that the a particle move on a sphere of radius $\sqrt{D_i\delta t}$ a each step...
\end{document}