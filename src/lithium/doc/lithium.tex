\documentclass[aps,onecolumn]{revtex4}
\usepackage{graphicx}
\usepackage{amssymb,amsfonts,amsmath,amsthm}
\usepackage{chemarr}
\usepackage{bm}
\usepackage{pslatex}
\usepackage{mathptmx}
\usepackage{xfrac}
\usepackage{xcolor}

\newcommand{\mychem}[1]{\mathtt{#1}}
\begin{document}

\section{Equations}
\subsection{Assumed mechanism}
We suppose the following mechanism
\begin{equation}
	\mychem{Li}^{x}_{out} + E \xrightleftharpoons[k_d^{x}]{k_a^{x}}
	\lbrace\mychem{Li}^xE\rbrace_{out} 
	\xrightleftharpoons[k_r^{x}]{k_f^{x}} 
	\lbrace\mychem{Li}^xE\rbrace_{in}
	\xrightarrow{k_t^x} E + \mychem{Li}^{x}_{in}
\end{equation}

\subsection{Full differential system}
\begin{equation}
	\left\lbrace
	\begin{array}{ccl}
	\partial_t [\mychem{Li}^6_{in}] & = &k_t^6[\lbrace\mychem{Li}^6E\rbrace_{in}  \\
	\partial_t [\mychem{Li}^7_{in}] & = &k_t^7[\lbrace\mychem{Li}^7E\rbrace_{in}  \\
	\\
	\partial_t [E] & = &
	-\left(k_a^6[\mychem{Li}^6_{out}]+k_a^7[\mychem{Li}^7_{out}]\right)[E] 
	+\left(k_d^6[\lbrace\mychem{Li}^6E\rbrace_{out}]+k_d^7[\lbrace\mychem{Li}^7E\rbrace_{out}]\right)
	+\left(k_t^6[\lbrace\mychem{Li}^6E\rbrace_{in} ]+k_t^7[\lbrace\mychem{Li}^7E\rbrace_{in} ]\right)
	\\
	\\
	\partial_t[\lbrace\mychem{Li}^6E\rbrace_{out}] & = & 
	k_a^6[\mathtt{Li}^6_{out}][E] - (k_d^6+k_f^6) [\lbrace\mychem{Li}^6E\rbrace_{out}]
	+k_r^6[\lbrace\mychem{Li}^6E\rbrace_{in}]
	= k_a^6 \left([\mathtt{Li}^6_{out}][E] -K_m^6 [\lbrace\mychem{Li}^6E\rbrace_{out}] + J_m^6 [\lbrace\mychem{Li}^6E\rbrace_{in}]\right)
	\\
	\partial_t[\lbrace\mychem{Li}^7E\rbrace_{out}] & = & 
	k_a^7[\mathtt{Li}^7_{out}][E] - (k_d^7+k_f^7) [\lbrace\mychem{Li}^7E\rbrace_{out}]
	+k_r^7[\lbrace\mychem{Li}^7E\rbrace_{in}]
	= k_a^7 \left([\mathtt{Li}^7_{out}][E] -K_m^7 [\lbrace\mychem{Li}^7E\rbrace_{out}] + J_m^7 [\lbrace\mychem{Li}^7E\rbrace_{in}\right)
	\\
	\\
	\partial_t[\lbrace\mychem{Li}^6E\rbrace_{in}] & = & 
	k_f^6 [\lbrace\mychem{Li}^6E\rbrace_{out}] - (k_r^6+k_t^6) [\lbrace\mychem{Li}^6E\rbrace_{in}]
	= k_f^6 \left( [\lbrace\mychem{Li}^6E\rbrace_{out}] - \alpha_6 [\lbrace\mychem{Li}^6E\rbrace_{in}]\right)
	\\
	\partial_t[\lbrace\mychem{Li}^7E\rbrace_{in}] & = & 
	k_f^7 [\lbrace\mychem{Li}^6E\rbrace_{out}] - (k_r^7+k_t^7) [\lbrace\mychem{Li}^7E\rbrace_{in}]
	= k_f^7 \left( [\lbrace\mychem{Li}^7E\rbrace_{out}] - \alpha_7 [\lbrace\mychem{Li}^7E\rbrace_{in}]\right)
	\\
	\end{array}
	\right. 
\end{equation}

\subsection{Steady-State Approximation}
After induction time, we assume that intermediates are at there steady state level
and
\begin{equation}
	E_0 = [E] 
	+ [\lbrace\mychem{Li}^6E\rbrace_{out}]
	+ [\lbrace\mychem{Li}^7E\rbrace_{out}]
	+ [\lbrace\mychem{Li}^6E\rbrace_{in}]
	+ [\lbrace\mychem{Li}^7E\rbrace_{in}]
\end{equation}
We first obtain
\begin{equation}
	\left\lbrace
	\begin{array}{rcl}
	~[\lbrace\mychem{Li}^6E\rbrace_{out}] & = & \alpha_6 [\lbrace\mychem{Li}^6E\rbrace_{in}]\\%
	~[\lbrace\mychem{Li}^7E\rbrace_{out}] & = & \alpha_7 [\lbrace\mychem{Li}^7E\rbrace_{in}]\\
	\end{array}
	\right.
\end{equation}
so that we have three remaining equations
\begin{equation}
	\left\lbrace
	\begin{array}{rcl}
	0 & = & 
	\end{array}
	\right.
\end{equation}

\end{document}