\documentclass[aps,onecolumn,11pt]{revtex4}
%\documentclass[11pt]{article}
%\usepackage[cm]{fullpage}
\usepackage{graphicx}
\usepackage{amssymb,amsfonts,amsmath,amsthm}
\usepackage{chemarr}
\usepackage{bm}
\usepackage{pslatex}
\usepackage{mathptmx}
\usepackage{xfrac}
\usepackage{xcolor}

\newcommand{\mychem}[1]{\mathtt{#1}}
\newcommand{\myconc}[1]{\left\lbrack{#1}\right\rbrack}

\newcommand{\spLi}[1]{{~^{\mychem{#1}}\mychem{Li}}}
\newcommand{\Li}[1]{\myconc{\spLi{#1}}}

\newcommand{\spEout}{\mychem{E}}
\newcommand{\Eout}{\myconc{\spEout}}

\newcommand{\spLiEin}[1]{\left\lbrace\spLi{#1}\spEout\right\rbrace_{\mathrm{in}}}
\newcommand{\LiEin}[1]{\myconc{\spLiEin{#1}}}

\newcommand{\spLiEout}[1]{\left\lbrace\spLi{#1}\spEout\right\rbrace_{\mathrm{out}}}
\newcommand{\LiEout}[1]{\myconc{\spLiEout{#1}}}

\newcommand{\spLiIn}[1]{{\spLi{#1}}_{\mathrm{in}}}
\newcommand{\LiIn}[1]{\myconc{\spLiIn{#1}}}

\newcommand{\spLiOut}[1]{{\spLi{#1}}_{\mathrm{out}}}
\newcommand{\LiOut}[1]{\myconc{\spLiOut{#1}}}

\newcommand{\spEHin}{\mychem{EH}}
\newcommand{\EHin}{\myconc{\spEHin}}
\newcommand{\spproton}{\mychem{H}}
\newcommand{\proton}{\myconc{\spproton}}

\newcommand{\mytrn}[1]{{#1}^{\!\mathsf{T}}}
\newcommand{\mymat}[1]{{\bm{#1}}}
\newcommand{\mydet}[1]{{\left|{#1}\right|}}

\begin{document}

\section{Mechanism}
$$
	\delta^7Li = \left(
		\dfrac{\left(\dfrac{Li^7}{Li^6}\right)_{sample}}
		{\left(\dfrac{Li^7}{Li^6}\right)_{standard}}
		 -1 
	\right) \times 1000
$$


\begin{equation}
	 \spLiOut{x} +  \spEout  
	 \xrightleftharpoons[k_d^x]{k_a^x} 
	 \spLiEout{x}
	  \xrightleftharpoons[k_r^x]{k_f^x} 
	  \spLiEin{x} 
	  \xrightleftharpoons[k_q^x]{\mychem{H},\;k_p^x} \underbrace{\spEHin}_{\xrightarrow[]{k_h} \mychem{E} + \mychem{H}_{\mathrm{out}}} + \underbrace{\spLiIn{x}}_{\xrightarrow[]{k_l^x} \spLiOut{x}}
\end{equation}

\section{Hypothesis}
\begin{itemize}
\item The flip stage is slower than the enzyme interaction with lithium.
\item $\proton$ is a parameter
\item $\LiOut{6}$ and  $\LiOut{7}$ are parameters
\end{itemize}

Accordingly, we consider that we have the four equations
\begin{equation}
%\left\lbrace
	\begin{array}{rcll}
	 \spLiOut{x} +  \spEout &  \xrightleftharpoons[]{} & \spLiEout{x}, & J_x = \dfrac{\LiEout{x}}{\LiOut{x} \Eout} = \dfrac{k_a^x}{k_d^x}\\
	 \\
	 \spLiEin{x} & \xrightleftharpoons[k_q^x]{\mychem{H},\;k_p^x} & {\spEHin}+ \spLiIn{x}, &K_x = \dfrac{\EHin \LiIn{x}}{\LiEin{x}\proton} = \dfrac{k_p^x}{k_q^x} \\
	\end{array}
	%\right.
\end{equation}

We have the vector of concentrations
\begin{equation}
	\vec{X} = 
	\begin{pmatrix}
	\Eout\\
	\EHin\\
	\LiEout{6}\\
	\LiEin{6}\\
	\LiIn{6}\\
	\LiEout{7}\\
	\LiEin{7}\\
	\LiIn{7}\\
	\end{pmatrix}
\end{equation}
The chemical coupling 4-vector is
\begin{equation}
\vec{\Gamma} = 
\begin{pmatrix}
J_6 \LiOut{6} \Eout - \LiEout{6}\\
J_7 \LiOut{7} \Eout - \LiEout{7}\\
K_6 \LiEin{6}\proton - \EHin \LiIn{6}\\
K_7 \LiEin{7}\proton - \EHin \LiIn{7}\\
\end{pmatrix}
\end{equation}

We will substitute
\begin{equation}
\left\lbrace
\begin{array}{rcl}
	\LiEout{x} & = & J_x \LiOut{x} \Eout\\
	\\
	\LiEin{x}  & = & \dfrac{\EHin \LiIn{x}}{K_x\proton}\\
\end{array}
\right.
\end{equation}
in the matter conservation
\begin{equation}
\left\lbrace
\begin{array}{rcl}
	E_0 & = &\displaystyle \Eout + \EHin + \sum_x \left(\LiEout{x}+\LiEin{x}\right)\\
	\\
	& = & \Eout \left\lbrack1+J_6 \LiOut{6} + J_7 \LiOut{7}\right\rbrack + 
	\EHin 
	\left\lbrack
		1+\dfrac{1}{h}\left(\dfrac{\LiIn{6}}{K_6}+\dfrac{\LiIn{7}}{K_7}\right)
	\right\rbrack
	\\
\end{array}
\right.
\end{equation}
so that
\begin{equation}
	\Eout = \dfrac{E_0 - \EHin \left( 1 + \dfrac{\LiIn{6}}{K'_6} + \dfrac{\LiIn{7}}{K'_7}\right)}
	{1+J'_6+J'_7}
\end{equation}

\section{Starting Points and Rates}

At $t=0$,
\begin{equation}
	\vec{X}_0 = 
	\begin{pmatrix}
	\Eout_0    & = & \frac{E_0}{1+J_6 \LiOut{6} + J_7 \LiOut{7}}  \\
	\EHin_0    & = & 0 \\
	\LiEout{6} & = & \frac{J_6 \LiOut{6} E_0} {1+J_6 \LiOut{6} + J_7 \LiOut{7}} \\
	\LiEin{6}  & = & 0 \\
	\LiIn{6}   & = & 0 \\
	\LiEout{7} & = & \frac{J_7 \LiOut{7} E_0} {1+J_6 \LiOut{6} + J_7 \LiOut{7}}   \\
	\LiEin{7}  & = & 0 \\
	\LiIn{7}   & = & 0 \\
	\end{pmatrix}.
\end{equation}

At any time, the rates are
\begin{equation}
\vec{\rho} = 
\begin{pmatrix}
	\sigma_h\\
	-\sigma_h\\
	\sigma_r^6-\sigma_f^6\\
	\sigma_f^6-\sigma_r^6\\
	-\sigma_l^6\\
	\sigma_r^7-\sigma_f^7\\
	\sigma_f^7-\sigma_r^7\\
	-\sigma_l^7\\
\end{pmatrix}
\end{equation}
and the steady state conditions become
\begin{equation}
	\left\lbrace
	\begin{array}{rcl}
		\sigma_l^x & = & \sigma_f^x - \sigma_r^x\\
		\sigma_h   & = & \sigma_l^6 + \sigma_l^7\\
	\end{array}
	\right.
\end{equation}

We get
\begin{equation}
\label{eq:ss}
\left\lbrace
	\begin{array}{rcl}
	k_l^x \left(\LiIn{x}-\theta_x\right) & = & k_f^x J'_x \Eout - k_r^x \EHin \dfrac{\LiIn{x}}{K'_x}\\
	k_h \EHin & = & k_l^6 \left(\LiIn{6}-\theta_6\right) + k_l^7 \left(\LiIn{7}-\theta_7\right)\\
	\end{array}
\right.
\end{equation}
knowing that
\begin{equation}
	\Eout = \dfrac{E_0 - \EHin \left( 1 + \dfrac{\LiIn{6}}{K'_6} + \dfrac{\LiIn{7}}{K'_7}\right)}
	{1+J'_6+J'_7}
\end{equation}

\section{Long Time Behavior: steady state}
It has no analytical solution, but it appears that there is no isotope ratio difference at
long time during the experiments.
As it is impossible to find an analytical solution to \eqref{eq:ss}, we say that
\begin{equation}
	\LiIn{6} = \alpha_6 \LiOut{6},\;\;\LiIn{7}=\alpha_7\LiOut{7}
\end{equation}
and we linearise the equations to find that
\begin{equation}
	\begin{pmatrix}
	\alpha_6\\
	\alpha_7\\
	\end{pmatrix}
	=
	\begin{pmatrix}
	\alpha_6^0\\
	\alpha_7^0\\
	\end{pmatrix}
	+
	\mymat{A} 
	\begin{pmatrix}
	\LiOut{6}\\
	\LiOut{7}\\
	\end{pmatrix}
	+\cdots.
\end{equation}
To ensure that $\alpha_6\approx\alpha_7$, we find to equalities arising from $\alpha_6^0=\alpha_7^0$ and 
the equality of the first and second row of $\mymat{A}$.
We get
\begin{equation}
\begin{array}{rcl}
	\dfrac{J_6 k_f^6}{k_l^6}  & = & \dfrac{J_7 k_f^7}{k_l^7}\\
	\dfrac{J_6K_6k_f^6}{k_r^6} & = & \dfrac{J_7K_7k_f^7}{k_r^7}\\
\end{array}
\end{equation}

\section{Asymptotic method}
We note that we have two decoupled systems,
so that with
\begin{equation}
	\mymat{\Phi} = \partial_{\vec{X}} \vec{\Gamma}
\end{equation}

\begin{equation}
\mymat{W} = \mytrn{\mymat{\nu}} \mymat{\Phi} = 
\begin{pmatrix}
	-(1+J'_6) & -J'_6 & 0 & 0\\
	-J'_7 & -(1+J'_7) & 0 & 0\\
	0 & 0 & -(\LiIn{6}+K'_6+\EHin) & -\LiIn{6}\\
	0 & 0 & -\LiIn{7} & -(\LiIn{7}+K'_7+\EHin) & \\
\end{pmatrix}
= \begin{pmatrix}
\mymat{W}_U & \mymat{0}\\
\mymat{0} & \mymat{W}_L\\
\end{pmatrix}
\end{equation}

Using the ORTHOGONAL matrices
\begin{equation}
\mymat{P}_U = 
\begin{pmatrix}
1 & 0 & 0 & 0\\
0 & 1 & 0 & 0\\
\end{pmatrix},\;\;
\mymat{P}_L = 
\begin{pmatrix}
0 & 0 & 1 & 0\\
0 & 0 & 0 & 1\\
\end{pmatrix}
\end{equation}
we have
\begin{equation}
\mymat{W}_U = \mymat{P}_U \mymat{W} \mytrn{\mymat{P}}_U, \;\;
\mymat{W}_L = \mymat{P}_L \mymat{W} \mytrn{\mymat{P}}_L
\end{equation}
and
\begin{equation}
\mymat{W}^{-1} = 
\begin{pmatrix}
\dfrac{1}{\mydet{\mymat{W}_U}} \mymat{W}_U^\ast & \mymat{0} \\
\mymat{0} & \dfrac{1}{\mydet{\mymat{W}_L}} \mymat{W}_L^\ast \\
\end{pmatrix}
=
\dfrac{1}{\mydet{\mymat{W}_U}} \mytrn{\mymat{P}}_U \mymat{W}_U^\ast \mymat{P}_U
+
\dfrac{1}{\mydet{\mymat{W}_L}} \mytrn{\mymat{P}}_L \mymat{W}_L^\ast \mymat{P}_L
\end{equation}
and we get
\begin{equation}
\partial_t \vec{X} = \left(\mymat{1} - \mytrn{\mymat{\nu}} \mymat{W}^{-1} \mymat{\Phi} \right) \vec{\rho}
\end{equation}

So we have $8$ species, $4$ equilibria an matter conservation, so that we have 3 degrees of freedom.
Using $\EHin$, $\LiIn{6}$ and $\LiIn{7}$, we will live in $\mymat{W}_L$ space only.
We use the projection matrix
\begin{equation}
	\mymat{Q} = 
	\begin{pmatrix}
	0 & 1 & 0 & 0 &0 & 0 & 0 & 0\\
	0 & 0 & 0 & 0 &1 & 0 & 0 & 0\\
	0 & 0 & 0 & 0 &0 & 0 & 0 & 1\\
	\end{pmatrix}
\end{equation}
so that
\begin{equation}
	\vec{Y} = \mymat{Q}\vec{X}
\end{equation}
and
\begin{equation}
	\partial_t \vec{Y} = 
	\left(
	\mymat{Q} - \mymat{Q} \mytrn{\mymat{\nu}} \mymat{W}^{-1} \mymat{\Phi} \right) \vec{\rho}
	=
	\dfrac{1}{\mydet{\mymat{W}_L}}
	\left(
		\mydet{\mymat{W}_L}\mymat{Q} - \mymat{Q} \mytrn{\mymat{\nu}} \mytrn{\mymat{P}}_L \mymat{W}_L^\ast \mymat{P}_L \mymat{\Phi} 
	\right)
	 \vec{\rho}
\end{equation}

\section{Channel Simulation}
%$$
%D_7 = 6.8\pm0.3 \times 10^{−5} cm^2/s
%$$
We take a channel with an attraction zone of size $R$ and length $H$. We take a particle with a 3D diffusion coefficient $D$
and we want to make a simulation with a time step $\delta t$, so that the length travelled by the particle is
\begin{equation}
	\delta l = \sqrt{D\delta t}
\end{equation}
We used adimensional units so that 
\begin{equation}
	\delta \tau = \dfrac{D\delta t}{R^2},\;\;\delta \lambda = \dfrac{\delta l}{R} = \sqrt{\delta \tau}
\end{equation}

\end{document}