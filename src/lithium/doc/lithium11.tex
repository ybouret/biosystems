\documentclass[aps,onecolumn,11pt]{revtex4}
%\documentclass[11pt]{article}
%\usepackage[cm]{fullpage}
\usepackage{graphicx}
\usepackage{amssymb,amsfonts,amsmath,amsthm}
\usepackage{chemarr}
\usepackage{bm}
\usepackage{pslatex}
\usepackage{mathptmx}
\usepackage{xfrac}
\usepackage{xcolor}

\newcommand{\mychem}[1]{\mathtt{#1}}
\newcommand{\myconc}[1]{\left\lbrack{#1}\right\rbrack}

\newcommand{\spLi}[1]{{~^{\mychem{#1}}\mychem{Li}}}
\newcommand{\Li}[1]{\myconc{\spLi{#1}}}

\newcommand{\spEout}{\mychem{E}}
\newcommand{\Eout}{\myconc{\spEout}}

\newcommand{\spLiEin}[1]{\left\lbrace\spLi{#1}\spEout\right\rbrace_{\mathrm{in}}}
\newcommand{\LiEin}[1]{\myconc{\spLiEin{#1}}}

\newcommand{\spLiE}[1]{\left\lbrace\spLi{#1}\spEout\right\rbrace}
\newcommand{\LiE}[1]{\myconc{\spLiE{#1}}}


\newcommand{\spLiEout}[1]{\left\lbrace\spLi{#1}\spEout\right\rbrace_{\mathrm{out}}}
\newcommand{\LiEout}[1]{\myconc{\spLiEout{#1}}}

\newcommand{\spLiIn}[1]{{\spLi{#1}}_{\mathrm{in}}}
\newcommand{\LiIn}[1]{\myconc{\spLiIn{#1}}}

\newcommand{\spLiOut}[1]{{\spLi{#1}}_{\mathrm{out}}}
\newcommand{\LiOut}[1]{\myconc{\spLiOut{#1}}}

\newcommand{\spEHin}{\mychem{EH}}
\newcommand{\EHin}{\myconc{\spEHin}}
\newcommand{\spproton}{\mychem{H}}
\newcommand{\proton}{\myconc{\spproton}}

\newcommand{\mytrn}[1]{{#1}^{\!\mathsf{T}}}
\newcommand{\mymat}[1]{{\bm{#1}}}
\newcommand{\mydet}[1]{{\left|{#1}\right|}}

\begin{document}

\section{Mechanism}
$$
	\delta^7Li = \left(
		\dfrac{\left(\dfrac{Li^7}{Li^6}\right)_{sample}}
		{\left(\dfrac{Li^7}{Li^6}\right)_{standard}}
		 -1 
	\right) \times 1000
$$


\begin{equation}
	 \spLiOut{x} +  \spEout  
	 \xrightleftharpoons[k_d^x]{k_a^x} 
	 \spLiE{x}
	  \xrightleftharpoons[k_q^x]{\mychem{H},\;k_p^x} \underbrace{\spEHin}_{\xrightarrow[]{k_h} \mychem{E} + \mychem{H}_{\mathrm{out}}} + \underbrace{\spLiIn{x}}_{\xrightarrow[]{k_l^x} \spLiOut{x}}
\end{equation}

\section{Hypothesis}
\begin{itemize}
\item The flip stage is slower than the enzyme interaction with lithium.
\item $\proton$ is a parameter
\item $\LiOut{6}$ and  $\LiOut{7}$ are parameters
\end{itemize}


We have the vector of concentrations
\begin{equation}
	\vec{X} = 
	\begin{pmatrix}
	\Eout\\
	\EHin\\
	\LiE{6}\\
	\LiIn{6}\\
	\LiE{7}\\
	\LiIn{7}\\
	\end{pmatrix},\;\;
	\partial_t \vec{X} =
	\begin{pmatrix}
	\sum_x\left(v_d^x-v_a^x\right) + v_h\\
	\sum_x\left(v_p^x-v_q^x\right) - v_h\\
	v_a^6-v_d^6+v_q^6-v_p^6\\
	v_p^6-v_q^6-v_l^6\\
	v_a^7-v_d^7+v_q^7-v_p^7\\
	v_p^7-v_q^7-v_l^7\\
	\end{pmatrix}
\end{equation}
And at any time
\begin{equation} 
	\label{eq:E0}
	E_0 = \Eout + \EHin +  \LiE{6} + \LiE{7}
\end{equation}

\section{Semi Stationary}
\subsection{Subsystem}
We consider that we have the two equations
\begin{equation}
%\left\lbrace
	\begin{array}{rcll}
	 \spLiOut{x} +  \spEout &  \xrightleftharpoons[]{} & \spLiE{x}, & J_x = \dfrac{\LiE{x}}{\LiOut{x} \Eout} = \dfrac{k_a^x}{k_d^x}\\
	 \end{array}
\end{equation}
leading to a constraint vector
\begin{equation}
\vec{\Gamma} = 
\begin{pmatrix}
	J_6' \Eout - \LiE{6} \\
	J_7' \Eout - \LiE{7} \\
\end{pmatrix}
\end{equation}
which already imposes to \eqref{eq:E0}
\begin{equation}
	E_0 = \EHin + \Eout \left(1+J'_6+J'_7\right)
\end{equation}

We find, using
\begin{equation}
	\vec{Y} = \begin{pmatrix}
	\EHin\\
	\Li{6}\\
	\Li{7}\\
	\end{pmatrix}
\end{equation}
that
\begin{equation}
	\partial_t \vec{Y} = \dot{\vec{Y}}_0 - \left\lbrack \mymat{\Omega}\vec{Y} + \vec{F}\left(\vec{Y}\right)\right\rbrack
\end{equation}
with
\begin{equation}
	\dot{\vec{Y}}_0 = 
	\begin{pmatrix}
	V_0\\
	V_6+\Lambda_6\\
	V_7+\Lambda_7\\
	\end{pmatrix},
	\;\;
	\left\lbrace
	\begin{array}{rcl}
	V_0 & = &E_0 \omega_0\\
	V_6 & = &E_0 \omega_6\\
	V_7 & = &E_0 \omega_7\\
	\omega_x & = & \dfrac{J_x' k_p^x \proton}{1+J_6'+J_7'}\\
	\Lambda_x & = & k_l^x \theta_x \\
	\end{array}
	\right.
\end{equation}

\begin{equation}
	\mymat{\Omega} = 
	\begin{pmatrix}
	\omega_0 + k_h & 0 & 0 \\
	\omega_6 & k_l^6 & 0 \\
	\omega_7 & 0 & k_l^7 \\
	\end{pmatrix}
\end{equation}

\begin{equation}
	\vec{F} = 
	\begin{pmatrix}
	k_q^6 \Li{6} \EHin + k_q^7 \Li{7} \EHin \\
	k_q^6 \Li{6} \EHin\\
	k_q^7 \Li{7} \EHin\\
	\end{pmatrix}
\end{equation}

\subsection{First order solution}
We solve
\begin{equation}
	\partial_t \vec{Y}_1 = \dot{\vec{Y}}_0 - \mymat{\Omega}\vec{Y}_1
\end{equation}
We first solve
\begin{equation}
	\partial_t \EHin_1 = V_0 - \left(\omega_0+k_h\right)  \EHin_1
\end{equation}
into
\begin{equation}
	\EHin_1 = E_0 \dfrac{\omega_0}{\omega_0+k_h}
	\left\lbrack
		 1-e^{-\left(\omega_0+k_h\right) t}
	\right\rbrack,\;\;\widetilde{\EHin}_1 = E_0 \dfrac{\omega_0}{\omega_0+k_h}
\end{equation}
Then we have to solve
\begin{equation}
	\partial_t \Li{x}_1 = \Lambda_x + V_x - \omega_x  \EHin_1 - k_l^x \Li{x}_1.
\end{equation}
We compute
\begin{equation}
	\widetilde{\Li{x}}_1 = \theta_x + \dfrac{E_0}{k_l^x}  \dfrac{\omega_x k_h}{\omega_0+k_h}
\end{equation}
%and we recognize the long time solution and a short time solution.

and we get 
\begin{equation}
	\Li{x}_1 = A_x(t) e^{-k_l^x t } 
\end{equation}
so that
\begin{equation}
	A_x = \int_0^t \left(\Lambda_x + V_x - \omega_x  \EHin_1 \right) e^{k_l^x u }\,\mathrm{d}u
\end{equation}
With 
\begin{equation}
	\sigma_x = E_0 \dfrac{\omega_0\omega_x}{\omega_0+k_h} < V_x = E_0 \omega_x
\end{equation}
\begin{equation}
	A_x = \int_0^t 
		\left(\Lambda_x + V_x - \sigma_x \right) e^{k_l^x u } 
		+ \sigma_x e^{\left\lbrack k_l^x  - \left(\omega_0+k_h\right) \right\rbrack u}
	\,\mathrm{d}u
\end{equation}
so that
\begin{equation}
	A_x = \dfrac{1}{k_l^x}\left(\Lambda_x + V_x - \sigma_x \right)\left(e^{k_l^x t}-1\right)
	+ \sigma_x \dfrac{
	e^{\left\lbrack k_l^x  - \left(\omega_0+k_h\right) \right\rbrack t} - 1
	}{\left\lbrack k_l^x  - \left(\omega_0+k_h\right) \right\rbrack}
\end{equation}
and
\begin{equation}
	\Li{x}_1 = \underbrace{\dfrac{1}{k_l^x}\left(\Lambda_x + V_x - \sigma_x \right)}_{\widetilde{\Li{x}}_1}\left(1-e^{-k_l^x t}\right)
	+\sigma_x \dfrac{
	e^{- \left(\omega_0+k_h\right)  t} - e^{-k_l^x t}
	}{ k_l^x  - \left(\omega_0+k_h\right) }
\end{equation}
We have the long time behaviour and the spiking behaviour.
We get
\begin{equation}
	\partial_t \Li{x}_1 = \left(\Lambda_x + V_x - \sigma_x \right) e^{-k_l^x t }
	+ \dfrac{\sigma_x}{k_l^x  - \left(\omega_0+k_h\right)}
	\left(k_l^x e^{-k_l^x t} - \left(\omega_0+k_h\right) e^{- \left(\omega_0+k_h\right)  t}\right)
\end{equation}

\begin{equation}
	\begin{array}{rl}
	\partial_t \Li{x}_1 = 0 & \Leftrightarrow 0=\left(\Lambda_x + V_x - \sigma_x \right) 
	+ \dfrac{\sigma_x}{k_l^x  - \left(\omega_0+k_h\right)}
	\left(k_l^x   - \left(\omega_0+k_h\right) e^{ \left\lbrack k_l^x - \left(\omega_0+k_h\right)\right\rbrack  t}\right)\\
	\end{array}
\end{equation}

\subsection{time reduction}
We set $\tau = \left(\omega_0+k_h\right) t$, and we define
\begin{equation}
	\left\lbrace
	\begin{array}{rcl}
	\eta_j    & = & \dfrac{\omega_j}{\omega_0+k_h}\;\;(\eta_0=\eta_6+\eta_7)\\
	\lambda_j & = & \dfrac{k_l^j}{\omega_0+k_h}   \\
	\chi_j    & = & \dfrac{k_q^j}{\omega_0+k_h}   \\
	\end{array}
	\right.
\end{equation}
so that


\begin{equation}
	\partial_\tau \vec{Y} = 
	\begin{pmatrix}
		E_0 \eta_0 \\
		E_0 \eta_6 + \lambda_6 \theta_6\\
		E_0 \eta_7 + \lambda_7 \theta_7\\
	\end{pmatrix}
	-
	\begin{pmatrix}
	1&0&0\\
	\eta_6&\lambda_6&0\\
	\eta_7&0&\lambda_7\\
	\end{pmatrix} 
	\vec{Y}
	-
	\underbrace{
	\begin{pmatrix}
	\chi_6 \EHin \Li{6} + \chi_7 \EHin \Li{7}\\
	\chi_6 \EHin \Li{6} \\
	\chi_7 \EHin \Li{7} \\
	\end{pmatrix}
	}_{\vec{G}\left(\vec{Y}\right)}
\end{equation}

\end{document}