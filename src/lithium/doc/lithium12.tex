\documentclass[aps,onecolumn,11pt]{revtex4}
%\documentclass[11pt]{article}
%\usepackage[cm]{fullpage}
\usepackage{graphicx}
\usepackage{amssymb,amsfonts,amsmath,amsthm}
\usepackage{chemarr}
\usepackage{bm}
\usepackage{pslatex}
\usepackage{mathptmx}
\usepackage{xfrac}
\usepackage{xcolor}
\usepackage{bookman}

\newcommand{\mychem}[1]{\mathtt{#1}}
\newcommand{\myconc}[1]{\left\lbrack{#1}\right\rbrack}

\newcommand{\spLi}[1]{{~^{\mychem{#1}}\mychem{Li}}}
\newcommand{\Li}[1]{\myconc{\spLi{#1}}}

\newcommand{\spEout}{\mychem{E}}
\newcommand{\Eout}{\myconc{\spEout}}

\newcommand{\spLiEin}[1]{\left\lbrace\spLi{#1}\spEout\right\rbrace_{\mathrm{in}}}
\newcommand{\LiEin}[1]{\myconc{\spLiEin{#1}}}

\newcommand{\spLiE}[1]{\left\lbrace\spLi{#1}\spEout\right\rbrace}
\newcommand{\LiE}[1]{\myconc{\spLiE{#1}}}


\newcommand{\spLiEout}[1]{\left\lbrace\spLi{#1}\spEout\right\rbrace_{\mathrm{out}}}
\newcommand{\LiEout}[1]{\myconc{\spLiEout{#1}}}

\newcommand{\spLiIn}[1]{{\spLi{#1}}_{\mathrm{in}}}
\newcommand{\LiIn}[1]{\myconc{\spLiIn{#1}}}

\newcommand{\spLiOut}[1]{{\spLi{#1}}_{\mathrm{out}}}
\newcommand{\LiOut}[1]{\myconc{\spLiOut{#1}}}

\newcommand{\spEHin}{\mychem{EH}}
\newcommand{\EHin}{\myconc{\spEHin}}
\newcommand{\spproton}{\mychem{H}}
\newcommand{\proton}{\myconc{\spproton}}

\newcommand{\mytrn}[1]{{#1}^{\!\mathsf{T}}}
\newcommand{\mymat}[1]{{\bm{#1}}}
\newcommand{\mydet}[1]{{\left|{#1}\right|}}

\newcommand{\ratioLi}{ {\left(\dfrac{\Li{7}}{\Li{6}}\right)} }
\newcommand{\deltaLi}{ {\delta\!\!\spLi{7}} }
\begin{document}

\section{Mechanism}
$$
	\deltaLi = \left(
		\dfrac{\left(\dfrac{Li^7}{Li^6}\right)_{sample}}
		{\left(\dfrac{Li^7}{Li^6}\right)_{standard}}
		 -1 
	\right) \times 1000
$$


\begin{equation}
	 \spLiOut{x} +  \spEout  
	 \xrightleftharpoons[k_d^x]{k_a^x} 
	 \spLiE{x}
	  \xrightleftharpoons[k_q^x]{\mychem{H},\;k_p^x} \underbrace{\spEHin}_{\xrightarrow[]{k_h} \mychem{E} + \mychem{H}_{\mathrm{out}}} + \underbrace{\spLiIn{x}}_{\xrightarrow[]{k_l^x} \spLiOut{x}}
\end{equation}

\section{Hypothesis}
\begin{itemize}
\item $\proton$ is a slowly varying parameter $h(t)$.
\item $\LiOut{6}$ and  $\LiOut{7}$ are parameters.
\item the second order return is negligible w.r.t first order recycling, ie $k_q^x\simeq0$.
\end{itemize}

We have the phase space described by
\begin{equation}
 \vec{X} = 
        \begin{pmatrix}
        \Eout\\
        \EHin\\
        \LiE{6}\\
        \LiIn{6}\\
        \LiE{7}\\
        \LiIn{7}\\
        \end{pmatrix}
\end{equation}

At any time, we must have
\begin{equation} 
	\label{eq:E0}
	E_0 = \Eout + \EHin +  \LiE{6} + \LiE{7}
\end{equation}
which is always true by the law of mass action.
The "slow" rate vector is
\begin{equation}
	\partial_t\vec{X} = 
	\begin{pmatrix}
		v_h\\
		p_6+p_7-v_h\\
		-p_6\\
		p_6-l_6\\
		-p_7\\
		p_7-l_7\\
	\end{pmatrix}
	,\;\;\text{ with }
	\left\lbrace
	\begin{array}{rcl}
	v_h & = & k_h \EHin\\
	p_x & = & k_p \proton \LiE{x}\\
	l_x & = & k_l^x \left(\Li{x}-\theta_x\right)\\
	\end{array}
	\right.
\end{equation}

\section{Semi Stationary}
\subsection{Subsystem}
We consider that we have the two equations
\begin{equation}
%\left\lbrace
	\begin{array}{rcll}
	 \spLiOut{x} +  \spEout &  \xrightleftharpoons[]{} & \spLiE{x}, & J_x = \dfrac{\LiE{x}}{\LiOut{x} \Eout} = \dfrac{k_a^x}{k_d^x}\\
	 \end{array}
\end{equation}
leading to a constraint vector $\vec{\Gamma}$
with 
\begin{equation}
	J'_x = J_x \LiOut{x}
\end{equation}

\begin{equation}
\vec{\Gamma} = 
\begin{pmatrix}
	J_6' \Eout - \LiE{6} \\
	J_7' \Eout - \LiE{7} \\
\end{pmatrix}
\end{equation}
which already simplifies the matter conservation \eqref{eq:E0} into
\begin{equation}
	E_0 = \EHin + \Eout \left(1+J'_6+J'_7\right).
\end{equation}

%We find, using
%\begin{equation}
%	\vec{Y} = \begin{pmatrix}
%	\alpha = \EHin/E_0\\
%	\Li{6}\\
%	\Li{7}\\
%	\end{pmatrix}
%\end{equation}
%that
%\begin{equation}
%	\partial_t \vec{Y} = \dot{\vec{Y}}_0 - \mymat{\Omega}\vec{Y}
%\end{equation}

We define
\begin{equation}
\left\lbrace
	\begin{array}{rcl}
		\alpha & = & \dfrac{\EHin}{E_0}\\
		\\
		\kappa_x & = & \dfrac{J_x'}{1+J_6'+J_7'} k_p^x\\
		\\
		\kappa_0 & = & \kappa_6 + \kappa_7\\
	\end{array}
\right.
\end{equation}

We started with 6 species, we have two constraints and matter conservation, so we are left with 3 degrees of freedom whose equations are given by:
\begin{equation}
\boxed{
\left\lbrace
\begin{array}{rcl}
	\partial_t \alpha  & = & \kappa_0 \proton - \alpha \left(k_h + \kappa_0 \proton\right)\\
	\partial_t \Li{x}  & = & \left(k_l^x \theta_x + E_0\kappa_x \proton\right) -  \left(E_0\kappa_x \proton \alpha +k_l^x \Li{x}\right)\\
\end{array}
\right.
}
\end{equation}

\subsection{Solving}

The approximation is to use $\proton$ as a parameter, which leads to:

\begin{equation}
\boxed{
	\alpha(t) = \underbrace{\dfrac{\kappa_0 \proton}{k_h + \kappa_0 \proton}}_{\alpha_\infty\left(\proton\right)}\left[1-e^{\displaystyle-\left(k_h+\kappa_0\proton\right)t}  \right]
	}
\end{equation}

which is the evolution of the NHE passing from the outer side to the inner side, a "simple" exponential growth/relaxation.\\
We also get the steady states:
\begin{equation}
\boxed{
	\label{eq:LiInf}
	\Li{x}_\infty = \theta_x + E_0 \left[1-\alpha_\infty\right].\dfrac{\kappa_x}{k_l^x} \proton
	}
\end{equation}
We have to solve an equation in the shape of
\begin{equation}
	\partial_t \Li{x} = \Lambda_x + V_x \left(1-\alpha\right) - k_l^x \Li{x} \;\; (V_x = E_0 \kappa_x \proton,\;\Lambda_x = k_l^x \theta_x)
\end{equation}
so that
\begin{equation}
	\Li{x} = A_x(t) e^{-k_l^x t}
\end{equation}
and
\begin{equation}
	\partial_t A_x = \left[ \Lambda_x + V_x \left(1-\alpha\right) \right] e^{ k_l^x t}
\end{equation}
to yield
\begin{equation}
	A(t) = \Li{x}_\infty \left(e^{ k_l^x t } - 1 \right)
	+ \int_0^t V_x\alpha_\infty e^{ \left[ k_l^x - (k_h+\kappa_0\proton)\right] u} \, \mathrm{d} u
\end{equation}
and finally
\begin{equation}
\boxed{
	\Li{x} = \Li{x}_\infty \left(1-e^{ -k_l^x t }\right) + 
	E_0 \kappa_x \proton \alpha_\infty \dfrac{e^{-(k_h+\kappa_0\proton)t} - e^{-k_l^xt} }{k_l^x - (k_h+\kappa_0\proton)}
	}
\end{equation}

\subsection{Simplifying}
\begin{itemize}
\item
There is no isotopic separation at $t\to\infty$, which means that
the lithiums just follow the GHK steady state equation.
It follows that
\begin{equation}
\boxed{
	\Li{x}_\infty = \beta \LiOut{x}
	}
\end{equation}
and that in the expression of $\Li{x}_\infty$ in \eqref{eq:LiInf}, the second term is negligible, or equal for $\Li{6}$ and $\Li{7}$.
\item The leak is expected to be a diffusive process, with almost no difference in diffusion coefficient (see papers),
so that 
\begin{equation}
	k_l^6 \simeq k_l^7 \simeq k_l
\end{equation}
and we now rescale the problem w.r.t THE leak rate using
\begin{equation}
	\tau = k_l t.
\end{equation}
\end{itemize}

We finally obtain
\begin{equation}
\Li{x} = \beta \LiOut{x} \left(1-e^{ -\tau }\right) + 
	\dfrac{E_0 \kappa_x \proton \alpha_\infty}{k_l}
	 \dfrac{e^{-\frac{(k_h+\kappa_0\proton)}{k_l}\tau} - e^{-\tau} }{1 - \dfrac{(k_h+\kappa_0\proton)}{k_l}}
\end{equation}
and using
\begin{equation}
	\sigma_h = \dfrac{(k_h+\kappa_0\proton)}{k_l}
\end{equation}

\begin{equation}
	\Li{x} = \beta \LiOut{x} \left(1-e^{ -\tau }\right) + 
	\dfrac{E_0 \kappa_x \proton \alpha_\infty}{k_l}
	\left[
	 \dfrac{e^{-\sigma_h\tau} - e^{-\tau} }{1 - \sigma_h}\right]
\end{equation}
and remember that $\proton$, $\alpha_\infty$ and $\sigma_h$ are time dependent through $\proton=h(t)$, and all the $\kappa$ terms are linked to the external lithium concentrations, and to the exchange rate $k_p$.

\subsection{Raw interpretation}
Along with the exponential growth/relaxation of $\Li{x}$ to its steady-state value, there is a bell-shaped function which appears.
What is interesting is that
\begin{equation}
	\left[
	 \dfrac{e^{-\sigma_h\tau} - e^{-\tau} }{1 - \sigma_h}\right] 
	 \simeq \tau - \dfrac{1+\sigma_h}{2} \tau^2 + \ldots
\end{equation}
The first term shows that there is an intake independent of the recycling, and which is driven by the enzyme, and which, for each lithium, will be proportional to a mix a thermodynamic and kinetic constants.\\
Globally, the bell-shaped function is peaked for
\begin{equation}
	\tau_{max} = \dfrac{\ln(\sigma_h)}{\sigma_h-1}\;\;\;(\tau_{max}=1 \text{ if } \sigma_h=1)
\end{equation}
Since the prefactor might be different, there is a possibility to have a significant isotopic separation before $\tau_{max}$ is reached.

\subsection{Rewriting}
We use the expression to rewrite
\begin{equation}
	\Li{x} =  \beta\LiOut{x}\left\lbrack  \left(1-e^{ -\tau }\right) + 
	\underbrace{
	\dfrac{J_x k_p^x}{k_l}
	\dfrac{E_0}{\beta}
	\dfrac{\proton \alpha_\infty}{1+J_6'+J_7'}
	}_{\phi_x}
	\underbrace{
	\left[
	 \dfrac{e^{-\sigma_h\tau} - e^{-\tau} }{1 - \sigma_h}\right]
	 }_{B\left(\tau,\sigma_h\right)}
	 \right\rbrack
\end{equation}
so that
\begin{equation}
	\dfrac{\Li{7} }{\Li{6}} = \dfrac{\LiOut{7} }{\LiOut{6}} 
	\dfrac{\left(1-e^{ -\tau }\right) + \phi_7 B\left(\tau,\sigma_h\right)}{\left(1-e^{ -\tau }\right) + \phi_6 B\left(\tau,\sigma_h\right) } 
	= \rho_7 \; \Omega\left(\tau,\phi_6,\phi_7,\sigma\right)
\end{equation}
with
\begin{equation}
	\Omega\left(\tau,\phi_6,\phi_7,\sigma\right) \simeq
	\dfrac{1+\phi_7}{1+\phi_6}
	- \dfrac{\phi_7-\phi_6}{2\left(1+\phi_6\right)^2} \sigma \tau
\end{equation}	

We have
\begin{equation}
	\delta\spLi{7} = 10^{3}\left\lbrack \dfrac{\ratioLi}{\ratioLi_{ref}} - 1\right\rbrack
\end{equation}
so that
\begin{equation}
	\ratioLi = \ratioLi_{ref} \left[ 1 + 10^{-3}  \deltaLi \right]
\end{equation}

\begin{equation}
	\ratioLi_{out} = \ratioLi_{ref} \left[ 1 + 10^{-3}  \deltaLi_{out} \right]
\end{equation}
and
\begin{equation}
\boxed{
	\dfrac{\ratioLi}{\ratioLi_{out}} = \dfrac{1+10^{-3}\deltaLi}{1+10^{-3}\deltaLi_{out}} = \Omega\left(\tau,\phi_6,\phi_7,\sigma\right)
	}
\end{equation}

\begin{equation}
{
	\left\lbrack\dfrac{\ratioLi}{\ratioLi_{out}} - 1 \right\rbrack =
	 \dfrac{10^{-3}\left(\deltaLi-\deltaLi_{out}\right)}{1+10^{-3}\deltaLi_{out}} = \tilde\Omega\left(\tau,\phi_6,\phi_7,\sigma\right)
}
\end{equation}
and
\begin{equation}
	\tilde\Omega\left(\tau,\phi_6,\phi_7,\sigma\right) = \Omega\left(\tau,\phi_6,\phi_7,\sigma\right) -1 \simeq \dfrac{\phi_7-\phi_6}{1+\phi_6} - \dfrac{\phi_7-\phi_6}{2\left(1+\phi_6\right)^2} \sigma \tau
\end{equation}

We have
\begin{equation}
	\tilde\Omega = \dfrac{\left(\phi_7-\phi_6\right)B\left(\tau,\sigma_h\right)}{\left(1-e^{ -\tau }\right) + \phi_6 B\left(\tau,\sigma_h\right)}
\end{equation}	
using
\begin{equation}
	\phi_7 = \left(1+\gamma_7\right)\phi_6
\end{equation}
we get
\begin{equation}
	 \tilde\Omega\left(\tau,\phi_6,\gamma_7,\sigma\right) = \gamma_7 
	 \dfrac{\phi_6 B\left(\tau,\sigma_h\right)}{\left(1-e^{ -\tau }\right) + \phi_6 B\left(\tau,\sigma_h\right)}
	 = \gamma_7 \dfrac{\phi_6}{1+\phi_6} \left[ \dfrac{\left(1+\phi_6\right)B\left(\tau,\sigma_h\right) }{\left(1-e^{ -\tau }\right) + \phi_6 B\left(\tau,\sigma_h\right)} \right]
\end{equation}

\subsection{special cases}
When $\sigma\to1$, we have
\begin{equation}
	B\left(\tau,\sigma\to1\right) = - \left[ \dfrac{e^{-\sigma\tau}-e^{-\tau}}{\sigma-1} \right] 
	= -\left[\partial_\sigma e^{-\sigma\tau} \right]_{\sigma\to1} 
\end{equation}
\begin{equation}
	B\left(\tau,\sigma\to1\right) \simeq -\left[\partial_\sigma e^{-\sigma\tau}\vert_{\sigma\to1} + \left(\sigma-1\right) \partial_\sigma^2 e^{-\sigma\tau}\vert_{\sigma\to1} + \dfrac{\left(\sigma-1\right)^2}{2} \partial_\sigma^3 e^{-\sigma\tau}\vert_{\sigma\to1}\right]
\end{equation}

\begin{equation}
	B\left(\tau,\sigma\to1\right) \simeq 
	-\left[ -\tau e^{-\tau} + \left(\sigma-1\right) \tau^2 e^{-\tau} - \left(\sigma-1\right)^2 \dfrac{\tau^3}{2} e^{-\tau} \right]	
\end{equation}
\begin{equation}
	B\left(\tau,\sigma\to1\right) \simeq 
	\tau e^{-\tau} 
	\left[ 
	1 - \left(\sigma-1\right) \tau \left(1-\dfrac{\left(\sigma-1\right) \tau}{2} \right)
	\right]	
\end{equation}

\subsection{variations}
The function
\begin{equation}
	\tilde\Omega = \dfrac{\left(\phi_7-\phi_6\right)B\left(\tau,\sigma_h\right)}{\left(1-e^{ -\tau }\right) + \phi_6 B\left(\tau,\sigma_h\right)}
	= \gamma_7 \dfrac{\phi_6}{1+\phi_6} \left[ \dfrac{\left(1+\phi_6\right)B\left(\tau,\sigma_h\right) }{\left(1-e^{ -\tau }\right) + \phi_6 B\left(\tau,\sigma_h\right)} \right]
\end{equation}
varies as
\begin{equation}
	\omega = \left(1+\phi_6\right)\dfrac{B\left(\tau,\sigma_h\right)}{\left(1-e^{ -\tau }\right) + \phi_6 B\left(\tau,\sigma_h\right)}
\end{equation}
which is decreasing for any $\sigma>0,\phi_6>0$.
\end{document}

