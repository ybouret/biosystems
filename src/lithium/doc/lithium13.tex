\documentclass[aps,onecolumn,12pt]{revtex4}
%\documentclass[11pt]{article}
%\usepackage[cm]{fullpage}
\usepackage{graphicx}
\usepackage{amssymb,amsfonts,amsmath,amsthm}
\usepackage{chemarr}
\usepackage{bm}
\usepackage{pslatex}
\usepackage{xfrac}
\usepackage{xcolor}
\usepackage{bookman}
\usepackage{dsfont}
\usepackage{mathptmx}

\newcommand{\mychem}[1]{\mathtt{#1}}
\newcommand{\myconc}[1]{\left\lbrack{#1}\right\rbrack}

\newcommand{\spLi}[1]{{~^{\mychem{#1}}\mychem{Li}}}
\newcommand{\Li}[1]{\myconc{\spLi{#1}}}

\newcommand{\spEout}{\mychem{E}}
\newcommand{\Eout}{\myconc{\spEout}}

\newcommand{\spLiEin}[1]{\left\lbrace\spLi{#1}\spEout\right\rbrace_{\mathrm{in}}}
\newcommand{\LiEin}[1]{\myconc{\spLiEin{#1}}}

\newcommand{\spLiE}[1]{\left\lbrace\spLi{#1}\spEout\right\rbrace}
\newcommand{\LiE}[1]{\myconc{\spLiE{#1}}}


\newcommand{\spLiEout}[1]{\left\lbrace\spLi{#1}\spEout\right\rbrace_{\mathrm{out}}}
\newcommand{\LiEout}[1]{\myconc{\spLiEout{#1}}}

\newcommand{\spLiIn}[1]{{\spLi{#1}}_{\mathrm{in}}}
\newcommand{\LiIn}[1]{\myconc{\spLiIn{#1}}}

\newcommand{\spLiOut}[1]{{\spLi{#1}}_{\mathrm{out}}}
\newcommand{\LiOut}[1]{\myconc{\spLiOut{#1}}}

\newcommand{\spEHin}{\mychem{EH}}
\newcommand{\EHin}{\myconc{\spEHin}}
\newcommand{\spproton}{\mychem{H}}
\newcommand{\proton}{\myconc{\spproton}}

\newcommand{\mytrn}[1]{{#1}^{\!\mathsf{T}}}
\newcommand{\mymat}[1]{{\bm{#1}}}
\newcommand{\mydet}[1]{{\left|{#1}\right|}}

\newcommand{\ratioLi}{ {\left(\dfrac{\Li{7}}{\Li{6}}\right)} }
\newcommand{\deltaLi}{ {\delta\!\!\spLi{7}} }
\begin{document}

\section{Proposed Mechanism}
$$
	\deltaLi = \left(
		\dfrac{\left(\dfrac{Li^7}{Li^6}\right)_{sample}}
		{\left(\dfrac{Li^7}{Li^6}\right)_{standard}}
		 -1 
	\right) \times 1000
$$


\begin{equation}
	 \spLiOut{x} +  \spEout  
	 \xrightleftharpoons[k_d^x]{k_a^x} 
	 \spLiE{x}
	  \xrightleftharpoons[k_q^x]{\mychem{H},\;k_p^x} \underbrace{\spEHin}_{\xrightarrow[]{k_h} \mychem{E} + \mychem{H}_{\mathrm{out}}} + \underbrace{\spLiIn{x}}_{\xrightleftharpoons[]{k_l^x} \spLiOut{x}}
\end{equation}

\section{Scheme}

\subsection{Hypothesis}
\begin{itemize}
\item $\proton$ is a slowly varying parameter $h(t)$.
\item $\LiOut{6}$ and  $\LiOut{7}$ are parameters.
%\item the second order return is negligible w.r.t first order recycling, ie $k_q^x\simeq0$.%%not yet?
\end{itemize}

\subsection{Kinetics}
We have the phase space described by
\begin{equation}
 \vec{X} = 
        \begin{pmatrix}
        \Eout\\
        \EHin\\
        \LiE{6}\\
        \LiIn{6}\\
        \LiE{7}\\
        \LiIn{7}\\
        \end{pmatrix}
\end{equation}

At any time, we  have
\begin{equation} 
	\label{eq:E0}
	E_0 = \Eout + \EHin +  \LiE{6} + \LiE{7}
\end{equation}

\subsection{Secondary Hypothesis}
We consider that we have the two equations
\begin{equation}
%\left\lbrace
	\begin{array}{rcll}
	 \spLiOut{x} +  \spEout &  \xrightleftharpoons[]{} & \spLiE{x}, & J_x = \dfrac{\LiE{x}}{\LiOut{x} \Eout} = \dfrac{k_a^x}{k_d^x}\\
	 \end{array}
\end{equation}
leading to a constraint vector $\vec{\Gamma}$
with 
\begin{equation}
	J'_x = J_x \LiOut{x}
\end{equation}

\begin{equation}
\vec{\Gamma} = 
\begin{pmatrix}
	J_6' \Eout - \LiE{6} \\
	J_7' \Eout - \LiE{7} \\
\end{pmatrix}
\end{equation}
which already simplifies the matter conservation \eqref{eq:E0} into
\begin{equation}
	E_0 = \EHin + \Eout \left(1+J'_6+J'_7\right).
\end{equation}
And we have the topology for the two equations as
\begin{equation}
	\label{eq:Nu}
	\mymat{\nu}=\begin{pmatrix}-1 & 0 & 1 & 0 & 0 & 0\cr -1 & 0 & 0 & 0 & 1 & 0\end{pmatrix}
\end{equation}
and
\begin{equation}
	\partial_{\vec{X}}\vec{\Gamma} = 
	\begin{pmatrix}J'_6 & 0 & -1 & 0 & 0 & 0\cr J'_7 & 0 & 0 & 0 & -1 & 0\end{pmatrix}
\end{equation}

\subsection{Rates}

The "slow" rate vector is
\begin{equation}
	\partial_t\vec{X}_{slow} = 
	\begin{pmatrix}
		v_h\\
		p_6-q_6+p_7-q_7-v_h\\
		q_6-p_6\\
		p_6-l_6-q_6\\
		q_7-p_7\\
		p_7-l_7-q_7\\
	\end{pmatrix}
	,\;\;\text{ with }
	\left\lbrace
	\begin{array}{rcll}
	v_h & = & k_h \EHin & \text{(recycling)}\\
	p_x & = & k_p^x \proton \LiE{x} & \text{(forward transfer)}\\
	q_x & = & k_q^x \EHin \LiIn{x} & \text{(reverse transfer)} \\
	l_x & = & k_l^x \left(\LiIn{x}-\theta_x\right) & \text{(leak)}\\
	\end{array}
	\right.
\end{equation}

\subsection{Semi-Stationary Equations}
We define
\begin{equation}
	\mymat{W} = \mymat{\Phi}\mytrn{\mymat{\nu}} = \begin{pmatrix} -J'_6-1 & -J'_6 \cr -J'_7 & -J'_7-1\end{pmatrix}
	,\;\mymat{W}^\ast = \begin{pmatrix} -J'_6-1 & J'_7 \cr J'_6 & -J'_7-1\end{pmatrix}
	,\;\;\delta^\ast=\det(\mymat{W})=1+J'_6+J'_7.
\end{equation}
and
\begin{equation}
	\mymat{\chi}^\ast = \mathds{1}_6-\mytrn{\mymat{\nu}}\mymat{W}^\ast\mymat{\Phi}
\end{equation}

\begin{equation}
	\partial_t\vec{X} = \dfrac{1}{\delta^\ast}
	\mymat{\chi}^\ast \partial_t\vec{X}_{slow}
\end{equation}
and we find
\begin{equation}
	\vec{Y} = \begin{pmatrix} \EHin \cr \LiIn{6} \cr \LiIn{7} \end{pmatrix}
	,\;\partial_t \vec{Y} = 
	\begin{pmatrix}
	p_6-q_6+p_7-q_7-v_h\\
	p_6-q_6-l_6\\
	p_7-q_7-l_7
	\end{pmatrix}
\end{equation}
with the expressions
\begin{equation}
\left\lbrace
	\begin{array}{rcl}
	v_h & = & k_h \EHin \\
	q_x & = & k_q^x \EHin \LiIn{x}  \\
	l_x & = & k_l^x \left(\LiIn{x}-\theta_x\right)\\
	p_x & = & k_p^x \proton \LiE{x}\\
	\end{array}
\right.
\end{equation}
with
\begin{equation}
	\LiE{x} = J'_x \Eout,\;\;\Eout=\dfrac{E_0-\EHin}{\delta^\ast}
\end{equation}
so that
\begin{equation}
	p_x = k_p^x \proton  J'_x \dfrac{E_0-\EHin}{\delta^\ast}
\end{equation}
We now use
\begin{equation}
	\begin{array}{rcl}
	\alpha    & = &\dfrac{\EHin}{E_0}\\
	\lambda_x & = & \LiIn{x}\\
	\end{array}
\end{equation}
and we get
\begin{equation}
	\left\lbrace
	\begin{array}{rcl}
	v_h & = & k_h E_0 \alpha \\
	q_x & = & k_q^x E_0 \alpha \lambda_x  \\
	l_x & = & k_l^x \left(\lambda_x -\theta_x\right)\\
	p_x & = & k_p^x E_0 \proton \dfrac{J'_x}{\delta^\ast}(1-\alpha) \\
	\end{array}
\right.
\end{equation}

We define the set of reduced constants
\begin{equation}
\left\lbrace
	\begin{array}{rcl}
	\kappa_x & = & k_p^x \dfrac{J'_x}{1+J'_6+J'_7}\\
	\kappa_0 & = & \kappa_6+\kappa_7\\
	\omega_0 & = & \kappa_0 \proton\\
	\omega_x & = & \kappa_x \proton \\
	\omega_0 & = & \omega_6 + \omega_7\\
	p_x & = & \omega_x E_0 \left(1-\alpha\right)\\
	\end{array}
\right.
\end{equation}


then
\begin{equation}
\boxed{
\left\lbrace
	\begin{array}{rcl}
		\partial_t\alpha    & = & \omega_0 - \alpha\left\lbrack k_h+\omega_0+{\sum_x k_q^x \lambda_x} \right\rbrack\\
		\partial_t\lambda_x & = & \left\lbrack k_l^x\theta_x+\omega_x E_0\right\rbrack
		-\left\lbrack
			\omega_x E_0\alpha +  \lambda_x \left(k_l^x+ k_q^x E_0\alpha\right)
		\right\rbrack\\
	\end{array}
\right.
}
\end{equation}

\subsection{Looking for steady-state}
\begin{equation}
	\hat{\lambda}_x = \dfrac{\left\lbrack k_l^x\theta_x+\omega_x E_0 \left(1-\hat{\alpha}\right)\right\rbrack}{k_l^x+ k_q^x E_0\hat{\alpha}}
\end{equation}


\end{document}
