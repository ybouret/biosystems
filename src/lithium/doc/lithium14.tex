\documentclass[aps,onecolumn,12pt]{revtex4}
\usepackage{graphicx}
\usepackage{amssymb,amsfonts,amsmath,amsthm}
\usepackage{chemarr}
\usepackage{bm}
\usepackage{pslatex}
\usepackage{xfrac}
\usepackage{xcolor}
\usepackage{bookman}
\usepackage{dsfont}
\usepackage{mathptmx}

\newcommand{\mychem}[1]{\mathtt{#1}}
\newcommand{\myconc}[1]{\left\lbrack{#1}\right\rbrack}

\newcommand{\spLi}[1]{{~^{\mychem{#1}}\mychem{Li}}}
\newcommand{\Li}[1]{\myconc{\spLi{#1}}}

\newcommand{\spEout}{\mychem{E}}
\newcommand{\Eout}{\myconc{\spEout}}

\newcommand{\spLiEin}[1]{\left\lbrace\spLi{#1}\spEout\right\rbrace_{\mathrm{in}}}
\newcommand{\LiEin}[1]{\myconc{\spLiEin{#1}}}

\newcommand{\spLiE}[1]{\left\lbrace\spLi{#1}\spEout\right\rbrace}
\newcommand{\LiE}[1]{\myconc{\spLiE{#1}}}


\newcommand{\spLiEout}[1]{\left\lbrace\spLi{#1}\spEout\right\rbrace_{\mathrm{out}}}
\newcommand{\LiEout}[1]{\myconc{\spLiEout{#1}}}

\newcommand{\spLiIn}[1]{{\spLi{#1}}_{\mathrm{in}}}
\newcommand{\LiIn}[1]{\myconc{\spLiIn{#1}}}

\newcommand{\spLiOut}[1]{{\spLi{#1}}_{\mathrm{out}}}
\newcommand{\LiOut}[1]{\myconc{\spLiOut{#1}}}

\newcommand{\spEHin}{\mychem{EH}}
\newcommand{\EHin}{\myconc{\spEHin}}
\newcommand{\spproton}{\mychem{H}}
\newcommand{\proton}{\myconc{\spproton}}

\newcommand{\mytrn}[1]{{#1}^{\!\mathsf{T}}}
\newcommand{\mymat}[1]{{\bm{#1}}}
\newcommand{\mydet}[1]{{\left|{#1}\right|}}

\newcommand{\ratioLi}{ {\left(\dfrac{\Li{7}}{\Li{6}}\right)} }
\newcommand{\deltaLi}{ {\delta\!\!\!\spLi{7}} }
\newcommand{\deltaLiOut}{{\deltaLi}_{\mathrm{out}}}
\begin{document}

\section{Isotopic Separation}
$$
	\deltaLi = \left(
		\dfrac{\left(\dfrac{\Li{7}}{\Li{6}}\right)_{sample}}
		{\left(\dfrac{\Li{7}}{\Li{6}}\right)_{standard}}
		 -1 
	\right) \times 1000
$$

$$
	\left(\dfrac{\Li{7}}{\Li{6}}\right)_{sample} = \left(\dfrac{\Li{7}}{\Li{6}}\right)_{standard} \left[1+10^{-3}\deltaLi\right] = \rho_s \left[1+10^{-3}\deltaLi\right]
$$


\section{Proposed Mechanism}

\begin{equation}
	 \spLiOut{x} +  \spEout  
	 \xrightleftharpoons[k_x^d]{k_x^a} 
	 \spLiE{x}
	  \xrightleftharpoons[k_x^q]{\mychem{H},\;k_x^p} \underbrace{\spEHin}_{\xrightarrow[]{k_h} \mychem{E} + \mychem{H}_{\mathrm{out}}} + \underbrace{\spLiIn{x}}_{\xrightleftharpoons[]{k_x^l} \spLiOut{x}}
\end{equation}

\section{Scheme}

\subsection{Hypothesis}
\begin{itemize}
\item $\proton$ is a  varying parameter $h(t)$.
\item $\LiOut{6}$ and  $\LiOut{7}$ are parameters.
%\item Variables noted with a '$\tilde{~}$' are dependent on those "parametric" concentrations
\end{itemize}

\subsection{Kinetics}
We have the phase space described by
\begin{equation}
 \vec{X} = 
        \begin{pmatrix}
        \Eout\\
        \EHin\\
        \LiE{6}\\
        \LiIn{6}\\
        \LiE{7}\\
        \LiIn{7}\\
        \end{pmatrix}
\end{equation}

At any time, we  have
\begin{equation} 
	\label{eq:E0}
	E_0 = \Eout + \EHin +  \LiE{6} + \LiE{7}
\end{equation}

\subsection{Secondary Hypothesis}
We consider that we have the two equations
\begin{equation}
%\left\lbrace
	\begin{array}{rcll}
	 \spLiOut{x} +  \spEout &  \xrightleftharpoons[]{} & \spLiE{x}, & J_x = \dfrac{\LiE{x}}{\LiOut{x} \Eout} = \dfrac{k_x^a}{k_x^d}\\
	 \end{array}
\end{equation}
leading to a constraint vector $\vec{\Gamma}$
with 
\begin{equation}
	\tilde{J}_x = J_x \LiOut{x}
\end{equation}

\begin{equation}
\vec{\Gamma} = 
\begin{pmatrix}
	\tilde{J}_6 \Eout - \LiE{6} \\
	\tilde{J}_7 \Eout - \LiE{7} \\
\end{pmatrix}
\end{equation}
which already simplifies the matter conservation \eqref{eq:E0} into
\begin{equation}
	E_0 = \EHin + \Eout \left(1+\tilde{J}_6+\tilde{J}_7\right).
\end{equation}
And we have the topology for the two equations as
\begin{equation}
	\label{eq:Nu}
	\mymat{\nu}=\begin{pmatrix}-1 & 0 & 1 & 0 & 0 & 0\cr -1 & 0 & 0 & 0 & 1 & 0\end{pmatrix}
\end{equation}
and
\begin{equation}
	\partial_{\vec{X}}\vec{\Gamma} = 
	\begin{pmatrix}\tilde{J}_6 & 0 & -1 & 0 & 0 & 0\cr\tilde{J}_7 & 0 & 0 & 0 & -1 & 0\end{pmatrix}
\end{equation}

\subsection{Rates}

The "slow" rate vector is
\begin{equation}
	\partial_t\vec{X}_{slow} = 
	\begin{pmatrix}
		v_h\\
		p_6-q_6+p_7-q_7-v_h\\
		q_6-p_6\\
		p_6-l_6-q_6\\
		q_7-p_7\\
		p_7-l_7-q_7\\
	\end{pmatrix}
	,\;\;\text{ with }
	\left\lbrace
	\begin{array}{rcll}
	v_h & = & k_h \EHin & \text{(recycling)}\\
	p_x & = & k_x^p \proton \LiE{x} & \text{(forward transfer)}\\
	q_x & = & k_x^q \EHin \LiIn{x} & \text{(reverse transfer)} \\
	l_x & = & k_x  \left(\LiIn{x}-\tilde{\Theta}_x\right) & \text{(leak)}\\
	\end{array}
	\right.
\end{equation}
with (Goldman-Hodgkins-Katz)
\begin{equation}
	\tilde{\Theta}_x = \Theta \LiOut{x}
\end{equation}

\subsection{Semi-Stationary Equations}
We define
\begin{equation}
	\mymat{W} = \mymat{\Phi}\mytrn{\mymat{\nu}} = \begin{pmatrix} -\tilde{J}_6-1 & -\tilde{J}_6 \cr -\tilde{J}_7 & -\tilde{J}_7-1\end{pmatrix}
	,\;\mymat{W}^\ast = \begin{pmatrix} -\tilde{J}_6-1 & \tilde{J}_7 \cr \tilde{J}_6 & -\tilde{J}_7-1\end{pmatrix}
	,\;\; \tilde{D} =\det(\mymat{W})=1+\tilde{J}_6+\tilde{J}_7.
\end{equation}
and
\begin{equation}
	\mymat{\chi} = \tilde{D}\mathds{1}_6-\mytrn{\mymat{\nu}}\mymat{W}^\ast\mymat{\Phi}
\end{equation}

\begin{equation}
	\partial_t\vec{X} = \dfrac{1}{\tilde{D}}
	\mymat{\chi} \partial_t\vec{X}_{slow}
\end{equation}
and we find
\begin{equation}
	\vec{Y} = \begin{pmatrix} \EHin \cr \LiIn{6} \cr \LiIn{7} \end{pmatrix}
	,\;\partial_t \vec{Y} = 
	\begin{pmatrix}
	p_6-q_6+p_7-q_7-v_h\\
	p_6-q_6-l_6\\
	p_7-q_7-l_7
	\end{pmatrix}
\end{equation}
with the expressions
\begin{equation}
\left\lbrace
	\begin{array}{rcl}
	v_h & = & k_h \EHin \\
	q_x & = & k_x^q \EHin \Li{x}  \\
	l_x & = & k_x  \left(\Li{x}- \tilde{\Theta}_x\right)\\
	p_x & = & k_x^p \proton \LiE{x}\\
	\end{array}
\right.
\end{equation}
with
\begin{equation}
	\LiE{x} = \tilde{J}_x \Eout,\;\;\Eout=\dfrac{E_0-\EHin}{\tilde{D}}
\end{equation}
so that
\begin{equation}
	p_x = k_x^p \proton  \tilde{J}_x \dfrac{E_0-\EHin}{\tilde{D}}
\end{equation}
We now use
\begin{equation}
	\begin{array}{rcl}
	\alpha    & = &\dfrac{\EHin}{E_0}\\
	%\lambda_x & = & \LiIn{x}\\
	\end{array}
\end{equation}
and we get
\begin{equation}
	\left\lbrace
	\begin{array}{rcl}
	v_h & = & k_h E_0 \alpha \\
	q_x & = & k_x^q E_0 \alpha \Li{x}  \\
	l_x & = & k_x \left(\Li{x} - \tilde{\Theta}_x\right)\\
	p_x & = & k_x^p E_0 \proton \dfrac{\tilde{J}_x}{\tilde{D}}(1-\alpha) \\
	\end{array}
\right.
\end{equation}

We define the set of reduced constants
\begin{equation}
\left\lbrace
	\begin{array}{rcl}
	\omega_x & = &  k_x^p \proton \dfrac{J_x}{\tilde{D}}\LiOut{x}\\
	\omega_0 & = & \omega_6 + \omega_7\\
	p_x & = & \omega_x E_0 \left(1-\alpha\right)\\
	\end{array}
\right.
\end{equation}


then
\begin{equation}
\boxed{
\left\lbrace
	\begin{array}{rcl}
		\partial_t\alpha    & = & \omega_0 - \alpha\left\lbrack k_h+\omega_0+{\sum_x k_x^q \Li{x}} \right\rbrack\\
		\partial_t\Li{x} & = & \left\lbrack k_x\tilde{\Theta}_x+\omega_x E_0\right\rbrack
		-\left\lbrack
			\omega_x E_0\alpha +  \Li{x}  \left(k_x + k_x^q E_0\alpha\right)
		\right\rbrack\\
	\end{array}
\right.
}
\end{equation}

\subsection{Looking for steady-state}
\subsubsection{Constraints}
\begin{equation}
	\Li{x}_\infty = \dfrac{\left\lbrack k_x\tilde{\Theta}_x+\omega_x E_0 \left(1-\alpha_\infty\right)\right\rbrack}{k_x+ k_x^q E_0\alpha_\infty}
\end{equation}
Since experimentally, there exist ${\beta}$ (observed) and $\Theta$ (GHK) such that
\begin{equation}
	\Li{x}_\infty=\beta\LiOut{x},\;\tilde{\Theta}_x = \Theta \LiOut{x}
\end{equation}
we obtain that
\begin{equation}
		\beta = \dfrac{k_x\Theta + k_x^p \proton {J_x} E_0 (1-\alpha_\infty)/\tilde{D}}{k_l^x+ k_q^x E_0\alpha_\infty}
\end{equation}
\textit{which must be the same for both species!}

We have
\begin{equation}
\left\lbrace
	\begin{array}{rcll}
	k_x^q    & = & \bar{q}\,k_x,  &  \bar{q}\text{ inverse of concentration}\\
	k_x^pJ_x & = & \bar{p}^2\,k_x, & \bar{p}\text{ inverse of concentration}\\
	\end{array}
\right.
\end{equation}
leading to
\begin{equation}
	\beta = \dfrac{\Theta+\dfrac{\bar{p}^2 \proton E_0}{1+J_6\LiOut{6}+J_7\LiOut{7}} \left(1-\alpha_\infty\right)}{1+\bar{q}E_0\alpha_\infty}
\end{equation}
which read exactly as the GHK level (electroosmotic leaks) shifted by NHE intake and reduced by NHE output.

\centerline{\bf This has some meaning: leak modulated isotopic separation!!!}

\subsubsection{Expression}
OK, I computed them, $\alpha_\infty$ is $\alpha_0$ with a slight decrease if $\bar{q}$ increases...
For many reasons, let's assume $\bar{q}E_0\ll 1$, and see if a correction is necessary

\subsection{First Order Equations}
\subsubsection{Rewrite}
We now assume
\begin{equation}
\left\lbrace
\begin{array}{rcl|l}
	\partial_t\alpha    & = & \omega_0 - \alpha\left\lbrack k_h+\omega_0\right\rbrack & {\alpha_\infty} = \dfrac{\omega_0}{\omega_0+k_h}\\
	\partial_t\Li{x} & = & \left\lbrack k_x\tilde{\Theta}_x+\omega_x E_0\right\rbrack
		-\left\lbrack
			\omega_x E_0\alpha +  k_x^l\Li{x}
		\right\rbrack & \\
	{\beta} & = & \Theta + {\kappa} \left(1-\alpha_\infty\right) & \kappa = \dfrac{\bar{p}^2 \proton E_0}{\tilde{D}} \\
	\end{array}
\right.
\end{equation}
and using
$$
	\omega_x =  \dfrac{\bar{p}^2}{\tilde{D}} k_x \proton \LiOut{x}
$$
we get
$$
	\partial_t\Li{x} =
	  k_x \left(\LiOut{x} \left[ \Theta + \kappa \left(1-\alpha\right)\right] - \Li{x} \right)
$$
then
\begin{equation}
	\partial_t\left( \dfrac{\Li{x}}{\LiOut{x}} \right)= k_x \left( \Theta + \kappa(1-\alpha) - \left( \dfrac{\Li{x}}{\LiOut{x}} \right)\right)
\end{equation}
namely
\begin{equation}
	\partial_t \beta_x =  k_x \left( \Theta + \kappa(1-\alpha) - \beta_x\right)
\end{equation}
\subsubsection{Solutions for $\alpha$}
We assume that $\proton$ is constant, with two different values
\begin{equation}
	\left\lbrace
	\begin{array}{rll}
	\text{if } t\leq t_c, & \proton = h & (\omega_0,\beta,\kappa)\\ 
	\text{if } t>t_c, & \proton = h^\ast & (\omega_0^\ast,\beta^\ast,\kappa^\ast)\\
	\end{array}
	\right.
\end{equation}

Let us solve $\alpha(t)$
\begin{equation}
\left\lbrace
\begin{array}{lrcl}
t\leq t_c, & \alpha & = & \dfrac{\omega_0}{\omega_h} \left(1-e^{-\omega_h t}\right) = \alpha_\infty \left(1-e^{-\omega_h t}\right),\;\;\omega_h = k_h+\omega_0 \\
t>t_c, & \alpha^\ast & = & \alpha_c e^{-\omega_h^\ast\left(t-t_c\right)} + 
\dfrac{\omega_0^\ast}{\omega_h^\ast}
\left(1 - e^{-\omega_h^\ast(t-t_c)}\right) = \alpha_c e^{-\omega_h^\ast\left(t-t_c\right)}  + \alpha_\infty^\ast \left(1 - e^{-\omega_h^\ast(t-t_c)}\right) 
\\
\end{array}
\right.
\end{equation}

\subsubsection{Solutions for $\beta_x$}

\begin{itemize}
\item $t\leq t_c$
\begin{equation}
	\partial_t \beta_x =  k_x \left( \Theta + \kappa(1-\alpha(t)) - \beta_x\right)
\end{equation}
We find
\begin{equation}
\left\lbrace
\begin{array}{rcl}
	\beta_x(t) & = & \beta \left(1-e^{-k_xt}\right) + \kappa\alpha_\infty \Xi\left(k_xt,\sigma_x\right)\\
	\beta      & = & \Theta+\kappa\left(1-\alpha_\infty\right)\\
	\sigma_x   & = & \dfrac{\omega_h}{k_x} = \dfrac{\omega_0}{k_x} + \dfrac{k_h}{k_x}\\
\end{array}
\right.
\end{equation}
with
\begin{equation}
\left\lbrace
\begin{array}{rcl}
	\Xi(u,p) & = & \dfrac{e^{-pu}-e^{-u}}{1-p}\\
	\Xi(u,1) & = & ue^{-u}\\
	 u_{max} & = & \dfrac{\ln p}{p-1}\\
	 \Xi_{max} &\approx&\dfrac{1-\tanh\left[0.7+0.4\ln p \right]}{2} 
\end{array}
\right.
\end{equation}
\item $t>t_c$
\begin{equation}
	\beta_x^\ast(t'=t-t_c) = \beta_{x,c} e^{-k_xt'} + \beta^\ast \left(1-e^{-k_xt'}\right) + \kappa^\ast(\alpha_\infty^\ast-\alpha_c) \Xi\left(k_xt',\sigma_x^\ast\right)\\
\end{equation}

\end{itemize}

\section{Rewriting}

\subsection{Boundary Conditions}

\begin{equation}
	\dfrac{\LiOut{7}}{\LiOut{6}} = \rho_s \left(1+10^{-3}\deltaLiOut\right)
\end{equation}
so that
\begin{equation}
\left\lbrace
	\begin{array}{rclcl}
	\LiOut{6} & = & \dfrac{1}{1+\rho_s\left(1+10^{-3}\deltaLiOut\right)} \LiOut{} & = & \sin^2(\varphi) \LiOut{} \\
	\\
	\LiOut{7} & = & \dfrac{\rho_s\left(1+10^{-3}\deltaLiOut\right)}{1+\rho_s\left(1+10^{-3}\deltaLiOut\right)} \LiOut{} & = & \cos^2(\varphi)\LiOut{} \\
	\end{array}
\right.
\end{equation}
$$
	\rho_s \approx 12.0192
$$
\begin{equation}
\left\lbrace
	\begin{array}{rcl}
	\sin^2(\varphi) = s^2 & \simeq & 0.0768 - 7.0910\,10^{-5} \deltaLiOut + \left(2.5586\,10^{-4}\deltaLiOut\right)^2\\
	\cos^2(\varphi) = c^2 & \simeq & 0.9232 + 7.0910\,10^{-5} \deltaLiOut - \left(2.5586\,10^{-4}\deltaLiOut\right)^2\\
	\end{array}
\right.
\end{equation}

\subsection{Delta Lithium}
\begin{equation}
	\deltaLi = 10^3 \left( \left[1+10^{-3}\deltaLiOut\right] \dfrac{\beta_7}{\beta_6} - 1\right)
\end{equation}
%
%\begin{equation}
%	\dfrac{\beta_7}{\beta_6} = 
%	\dfrac
%	{\left(1-e^{-\tau}\right) + \psi \Xi\left(\tau,\sigma\right)}
%	{\left(1-e^{-\lambda\tau}\right) + \psi \Xi\left(\lambda\tau,\sigma/\lambda\right)}
%\end{equation}

\begin{equation}
\left\lbrace
	\begin{array}{rcl}
	\deltaLi_{ini} & = & \dfrac{\deltaLiOut+1000(1-\lambda)}{\lambda}\\
	\\
	\lambda        & = & \dfrac{1000+\deltaLiOut}{1000+\deltaLi_{ini}}\\
	\end{array}
\right.
\end{equation}

\begin{equation}
\left\lbrace
	\begin{array}{rcl}
	\tau & = & k_7 t\\
	k_6  & = & \lambda k_7\\
	\dfrac{k_h}{k_7} & = & \eta \\
	\end{array}
\right.
\end{equation}

\subsection{Short Times}

\begin{equation}
	t \leq t_c,\;\; \beta_x \approx \kappa \alpha_\infty \Xi\left(k_x t, \sigma_x \right)
\end{equation}
so that
\begin{equation}
	\dfrac{\beta_7}{\beta_6} \approx \dfrac{\Xi(\tau,\sigma)}{\Xi(\lambda \tau, \sigma/\lambda)}
\end{equation}
then
\begin{equation}
	\beta_{7,c} = \Xi(\tau_c,\sigma),\;\;\beta_{6,c} = \Xi(\lambda\tau_c,\sigma/\lambda)
\end{equation}

\subsection{Long Times}

\begin{equation}
	t > t_c,\;\; \dfrac{\beta_7}{\beta_6} \approx 
	\dfrac
	{\beta_{7,c}e^{-\tau'}+\beta^\ast \left(1-e^{-\tau'}\right)}
	{\beta_{6,c}e^{-\lambda\tau'}+\beta^\ast \left(1-e^{-\lambda\tau'}\right)}
\end{equation}
with
\begin{equation}
	\beta_{x,c} \simeq \beta \left(1-e^{-k_x t_c}\right)
\end{equation}
for we expect that the catalytic part vanished at that point.
Then
\begin{equation}
\begin{array}{rcl}
	\beta_{7,c} & = & \beta \left(1-e^{-\tau_c}\right)\\
	\beta_{6,c} & = & \beta \left(1-e^{-\lambda\tau_c}\right)\\
\end{array}
\end{equation}
and
\begin{equation}
	t > t_c,\;\; \dfrac{\beta_7}{\beta_6} \approx 
	\dfrac
	{\left(1-e^{-\tau_c}\right)e^{-\tau'} + \gamma \left(1-e^{-\tau'}\right)}
	{\left(1-e^{-\lambda\tau_c}\right)e^{-\lambda\tau'} + \gamma \left(1-e^{-\lambda\tau'}\right)}
\end{equation}

\section{Direct Integration}

\begin{equation}
	\partial_t\alpha  = \omega_0 - \alpha\left\lbrack k_h+\omega_0\right\rbrack 
\end{equation}
and with
$$
	\omega_x =  \dfrac{\bar{p}^2}{\tilde{D}} k_x \proton \LiOut{x}
$$
then
$$
	\omega_0 = \bar{p}^2 k_7 h(t) \dfrac{\left(c^2+\lambda s^2\right) \LiOut{} }{ 1 + \left(c^2J_6+s^2J_7\right) \LiOut{}}.
$$
That we rewrite
$$
	\omega_0 = k_7  \gamma \frac{h(t)}{h_0}
$$
where $\gamma$ is an inverse concentration that depends on the external concentrations in lithium.
Then
\begin{equation}
	\partial_t \alpha = k_7  \left(\gamma \frac{h(t)}{h_0} - \alpha \left[\gamma \frac{h(t)}{h_0} + \eta \right]\right),\;\;\eta=\dfrac{k_h}{k_7}
\end{equation}

\begin{equation}
	\kappa = \dfrac{\bar{p}^2 \proton E_0}{\tilde{D}} = \mu \frac{h(t)}{h_0}
\end{equation}

\begin{equation}
	\partial_t \beta_x =  k_x \left( \Theta + \mu \frac{h(t)}{h_0} (1-\alpha(t)) - \beta_x\right)
\end{equation}


\end{document}
