\documentclass[aps,onecolumn,12pt]{revtex4}
\usepackage{graphicx}
\usepackage{amssymb,amsfonts,amsmath,amsthm}
\usepackage{chemarr}
\usepackage{bm}
\usepackage{pslatex}
\usepackage{xfrac}
\usepackage{xcolor}
\usepackage{bookman}
\usepackage{dsfont}
\usepackage{mathptmx}

\newcommand{\mychem}[1]{\mathtt{#1}}
\newcommand{\myconc}[1]{\left\lbrack{#1}\right\rbrack}

\newcommand{\spLi}[1]{{~^{\mychem{#1}}\mychem{Li}}}
\newcommand{\Li}[1]{\myconc{\spLi{#1}}}

\newcommand{\spEout}{\mychem{E}}
\newcommand{\Eout}{\myconc{\spEout}}

\newcommand{\spLiEin}[1]{\left\lbrace\spLi{#1}\spEout\right\rbrace_{\mathrm{in}}}
\newcommand{\LiEin}[1]{\myconc{\spLiEin{#1}}}

\newcommand{\spLiE}[1]{\left\lbrace\spLi{#1}\spEout\right\rbrace}
\newcommand{\LiE}[1]{\myconc{\spLiE{#1}}}


\newcommand{\spLiEout}[1]{\left\lbrace\spLi{#1}\spEout\right\rbrace_{\mathrm{out}}}
\newcommand{\LiEout}[1]{\myconc{\spLiEout{#1}}}

\newcommand{\spLiIn}[1]{{\spLi{#1}}_{\mathrm{in}}}
\newcommand{\LiIn}[1]{\myconc{\spLiIn{#1}}}

\newcommand{\spLiOut}[1]{{\spLi{#1}}_{\mathrm{out}}}
\newcommand{\LiOut}[1]{\myconc{\spLiOut{#1}}}

\newcommand{\spEHin}{\mychem{EH}}
\newcommand{\EHin}{\myconc{\spEHin}}
\newcommand{\spproton}{\mychem{H}}
\newcommand{\proton}{\myconc{\spproton}}

\newcommand{\mytrn}[1]{{#1}^{\!\mathsf{T}}}
\newcommand{\mymat}[1]{{\bm{#1}}}
\newcommand{\mydet}[1]{{\left|{#1}\right|}}

\newcommand{\ratioLi}{ {\left(\dfrac{\Li{7}}{\Li{6}}\right)} }
\newcommand{\deltaLi}{ {\delta\!\!\!\spLi{7}} }
\newcommand{\deltaLiOut}{{\deltaLi}_{\mathrm{out}}}
\newcommand{\ih}{\ensuremath{\mathbf{H}}}
\newcommand{\ig}{\ensuremath{\mathbf{G}}}

\newcommand{\LiAll}{\Lambda}
\newcommand{\LiAllOut}{{\LiAll}_{\mathrm{out}}}

\begin{document}

\section{Isotopic Separation}
$$
	\deltaLi = \left(
		\dfrac{\left(\dfrac{\Li{7}}{\Li{6}}\right)_{sample}}
		{\left(\dfrac{\Li{7}}{\Li{6}}\right)_{standard}}
		 -1 
	\right) \times 1000
$$

$$
	\left(\dfrac{\Li{7}}{\Li{6}}\right)_{sample} = \left(\dfrac{\Li{7}}{\Li{6}}\right)_{standard} \left[1+10^{-3}\deltaLi\right] = \beta_s \left[1+10^{-3}\deltaLi\right]
$$

\begin{equation}
\left\lbrace
\begin{array}{rcl}
	\LiAll    & = & \Li{6} + \Li{7}\\
	\LiAllOut & = & \LiOut{6} + \LiOut{7}\\
\end{array}
\right.
\end{equation}
and
\begin{equation}
\left\lbrace
\begin{array}{rclcl}
	\LiOut{6} & = & \dfrac{1}{1+\beta_s \left[1+10^{-3}\deltaLi\right] } \LiAllOut & = & \epsilon_6 \LiAllOut  = \epsilon \LiAllOut \\
	\\
	\LiOut{7} & = & \dfrac{\beta_s \left[1+10^{-3}\deltaLi\right]}{1+\beta_s \left[1+10^{-3}\deltaLi\right] } \LiAllOut & = & \epsilon_7 \LiAll,\;\epsilon_7 = 1-\epsilon \\
\end{array}
\right.
\end{equation}
with, for the experiments,
\begin{equation}
	\beta_s = 12.0192
\end{equation}
\begin{equation}
	\epsilon^\mathrm{out} \simeq 0.076, 1-\epsilon^\mathrm{out} \simeq 0.924
\end{equation}

\section{Proposed Mechanism}

\begin{equation}
	 \spLiOut{x} +  \spEout  
	 \xrightleftharpoons[k_x^d]{k_x^a} 
	 \spLiE{x}
	  \xrightleftharpoons[k_x^q]{\mychem{+H},\;k_x^p} \underbrace{\spEHin}_{\xrightarrow[]{k_h} \mychem{E} + \mychem{H}_{\mathrm{out}}} + \underbrace{\spLiIn{x}}_{\xrightleftharpoons[]{k_x} \spLiOut{x}}
\end{equation}

\section{Scheme}

\subsection{Hypothesis}
\begin{itemize}
\item $\proton$ is a  user's function $h(t)$.
\item $\LiOut{6}$ and  $\LiOut{7}$ are parameters.
\item $k_h$ is constant during the whole experiments for NHE is in its saturated mode for the full range of pH
\end{itemize}

\subsection{Kinetics}
We have the phase space described by
\begin{equation}
 \vec{X} = 
        \begin{pmatrix}
        \Eout\\
        \EHin\\
        \LiE{6}\\
        \LiIn{6}\\
        \LiE{7}\\
        \LiIn{7}\\
        \end{pmatrix}
\end{equation}

At any time, we  have
\begin{equation} 
	\label{eq:E0}
	E_0 = \Eout + \EHin +  \LiE{6} + \LiE{7}
\end{equation}

\subsection{Secondary Hypothesis}
We consider that we have the two equations
\begin{equation}
%\left\lbrace
	\begin{array}{rcll}
	 \spLiOut{x} +  \spEout &  \xrightleftharpoons[]{} & \spLiE{x}, & J_x = \dfrac{\LiE{x}}{\LiOut{x} \Eout} = \dfrac{k_x^a}{k_x^d}\\
	 \end{array}
\end{equation}
leading to a constraint vector $\vec{\Gamma}$
with 
\begin{equation}
	\tilde{J}_x = J_x \LiOut{x}
\end{equation}

\begin{equation}
\vec{\Gamma} = 
\begin{pmatrix}
	\tilde{J}_6 \Eout - \LiE{6} \\
	\tilde{J}_7 \Eout - \LiE{7} \\
\end{pmatrix}
\end{equation}
which already simplifies the matter conservation \eqref{eq:E0} into
\begin{equation}
	E_0 = \EHin + \Eout \left(1+\tilde{J}_6+\tilde{J}_7\right).
\end{equation}
And we have the topology for the two equations as
\begin{equation}
	\label{eq:Nu}
	\mymat{\nu}=\begin{pmatrix}-1 & 0 & 1 & 0 & 0 & 0\cr -1 & 0 & 0 & 0 & 1 & 0\end{pmatrix}
\end{equation}
and
\begin{equation}
	\partial_{\vec{X}}\vec{\Gamma} = 
	\begin{pmatrix}\tilde{J}_6 & 0 & -1 & 0 & 0 & 0\cr\tilde{J}_7 & 0 & 0 & 0 & -1 & 0\end{pmatrix}
\end{equation}

\subsection{Rates}

The "slow" rate vector is
\begin{equation}
	\partial_t\vec{X}_{slow} = 
	\begin{pmatrix}
		v_h\\
		p_6-q_6+p_7-q_7-v_h\\
		q_6-p_6\\
		p_6-l_6-q_6\\
		q_7-p_7\\
		p_7-l_7-q_7\\
	\end{pmatrix}
	,\;\;\text{ with }
	\left\lbrace
	\begin{array}{rcll}
	v_h & = & k_h \EHin & \text{(recycling)}\\
	p_x & = & k_x^p \proton \LiE{x} & \text{(forward transfer)}\\
	q_x & = & k_x^q \EHin \LiIn{x} & \text{(reverse transfer)} \\
	l_x & = & k_x  \left(\LiIn{x}-\tilde{\Theta}_x\right) & \text{(leak)}\\
	\end{array}
	\right.
\end{equation}
with (Goldman-Hodgkins-Katz)
\begin{equation}
	\tilde{\Theta}_x = \Theta \LiOut{x}
\end{equation}

\subsection{Semi-Stationary Equations}
We define
\begin{equation}
	\mymat{W} = \mymat{\Phi}\mytrn{\mymat{\nu}} = \begin{pmatrix} -\tilde{J}_6-1 & -\tilde{J}_6 \cr -\tilde{J}_7 & -\tilde{J}_7-1\end{pmatrix}
	,\;\mymat{W}^\ast = \begin{pmatrix} -\tilde{J}_6-1 & \tilde{J}_7 \cr \tilde{J}_6 & -\tilde{J}_7-1\end{pmatrix}
	,\;\; \tilde{D} =\det(\mymat{W})=1+\tilde{J}_6+\tilde{J}_7.
\end{equation}
and
\begin{equation}
	\mymat{\chi} = \tilde{D}\mathds{1}_6-\mytrn{\mymat{\nu}}\mymat{W}^\ast\mymat{\Phi}
\end{equation}

\begin{equation}
	\partial_t\vec{X} = \dfrac{1}{\tilde{D}}
	\mymat{\chi} \partial_t\vec{X}_{slow}
\end{equation}
and we find
\begin{equation}
	\vec{Y} = \begin{pmatrix} \EHin \cr \LiIn{6} \cr \LiIn{7} \end{pmatrix}
	,\;\partial_t \vec{Y} = 
	\begin{pmatrix}
	p_6-q_6+p_7-q_7-v_h\\
	p_6-q_6-l_6\\
	p_7-q_7-l_7
	\end{pmatrix}
\end{equation}
with the expressions
\begin{equation}
\left\lbrace
	\begin{array}{rcl}
	v_h & = & k_h \EHin \\
	q_x & = & k_x^q \EHin \Li{x}  \\
	l_x & = & k_x  \left(\Li{x}- \tilde{\Theta}_x\right)\\
	p_x & = & k_x^p \proton \LiE{x}\\
	\end{array}
\right.
\end{equation}
with
\begin{equation}
	\LiE{x} = \tilde{J}_x \Eout,\;\;\Eout=\dfrac{E_0-\EHin}{\tilde{D}}
\end{equation}
so that
\begin{equation}
	p_x = k_x^p \proton  \tilde{J}_x \dfrac{E_0-\EHin}{\tilde{D}}
\end{equation}

then we get the \underline{three} coupled equations  

\begin{equation}
\boxed{
\left\lbrace
	\begin{array}{rcl}
		\partial_t\EHin & = & -k_h \EHin + \left(E_0- \EHin\right) \dfrac{\proton}{\tilde{D}} \left(\sum_x k_x^p \tilde{J}_x \right)  
		- \EHin \left\lbrack {\sum_x k_x^q \Li{x}} \right\rbrack\\
		& = & 
		-k_h E_0+ \left(E_0- \EHin\right)\left\lbrack k_h+ \dfrac{\proton}{\tilde{D}} \left(\sum_x k_x^p \tilde{J}_x \right)\right] 
		- \EHin \left\lbrack {\sum_x k_x^q \Li{x}} \right\rbrack\\
		\partial_t\Li{x} & = & k_x \left(\tilde{\Theta}_x -\Li{x} \right)  + \left(E_0-\EHin\right) \dfrac{\proton}{\tilde{D}}   k_x^p \tilde{J}_x  - \EHin k_x^q \Li{x}\\
	\end{array}
\right.
}
\end{equation}


\subsection{Looking for steady-state, part I}
We assume that there exists $\EHin_\infty$, then
\begin{equation}
	\Li{x}_\infty = 
	\dfrac{\left\lbrack k_x\tilde{\Theta}_x+ \dfrac{k_x^p \tilde{J}_x}{\tilde{D}} h_\infty\left(E_0-\EHin_\infty\right)\right\rbrack}
	{k_x+ k_x^q \EHin_\infty}
\end{equation}
Since experimentally, there exist ${\beta}$ (observed) and $\Theta$ (GHK) such that
\begin{equation}
	\Li{x}_\infty=\beta\LiOut{x},\;\tilde{\Theta}_x = \Theta \LiOut{x}
\end{equation}
we obtain that
\begin{equation}
		\beta = \dfrac{\left\lbrack k_x{\Theta}_x+ \dfrac{k_x^p {J}_x}{\tilde{D}} h_\infty\left(E_0-\EHin_\infty\right)\right\rbrack}
	{k_x+ k_x^q \EHin_\infty}
\end{equation}
\textit{which must be the same for both species!}

We have
\begin{equation}
\left\lbrace
	\begin{array}{rcll}
	k_x^q    & = & \bar{q}\,k_x,  &  \bar{q}\text{ inverse of concentration}\\
	k_x^pJ_x & = & \bar{p}^2\,k_x, & \bar{p}\text{ inverse of concentration}\\
	\end{array}
\right.
\end{equation}
leading to
\begin{equation}
	\label{steady_beta}
	\beta = \dfrac{\Theta+\dfrac{\bar{p}^2 h_\infty }{1+J_6\LiOut{6}+J_7\LiOut{7}} \left(E_0-\EHin_\infty\right)}{1+\bar{q}\EHin_\infty}
\end{equation}
which read exactly as the GHK level (electroosmotic leaks) shifted by NHE intake and reduced by NHE output.

\centerline{\bf This has some meaning: leak modulated isotopic separation!!!}



\subsection{Unified rewrite}

\begin{equation}
\boxed{
\left\lbrace
	\begin{array}{rcl}
		\partial_t\EHin & = & 
		-k_h E_0
		+ \left(E_0- \EHin\right)\left\lbrack k_h+ \dfrac{\bar{p}^2\proton}{\tilde{D}} \left(\sum_x k_x \LiOut{x} \right)\right] 
		- \bar{q}\EHin \left( {\sum_x k_x \Li{x}} \right)\\
		& = & 
			-k_h \EHin
		+ \left(E_0- \EHin\right)\left\lbrack \dfrac{\bar{p}^2\proton}{\tilde{D}} \left(\sum_x k_x \LiOut{x} \right)\right] 
		- \bar{q}\EHin \left( {\sum_x k_x \Li{x}} \right)\\
		\partial_t\Li{x} & = & k_x \left({\Theta}_x\LiOut{x} -\Li{x} \right) 
		 + k_x \left(E_0-\EHin\right) \dfrac{\bar{p}^2\proton}{\tilde{D}}  \LiOut{x}  -  \bar{q}\EHin k_x \Li{x}\\
	\end{array}
\right.
}
\end{equation}


\begin{equation}
\left\lbrace
\begin{array}{rcl}
	\alpha     & = & \dfrac{\EHin}{E_0}\\
	\\
	\beta_x    & = & \dfrac{\Li{x}}{\LiOut{x}}\\
\end{array}
\right.
\end{equation}

\begin{equation}
\left\lbrace
\begin{array}{rcl}
	\partial_t \alpha & = & -k_h \alpha +
	 \left[
	  (1-\alpha) \dfrac{\bar{p}^2 \proton }{\tilde{D}} \left( \sum_x k_x \epsilon_x \right)
	  - \bar{q} \alpha  \left( \sum_x k_x \epsilon_x \beta_x \right)
	  \right] \LiAllOut\\
	  \\
	  \partial_t \beta_x & = & 
	  k_x \left( \Theta - \beta_x \right)
	  + \left[
	  k_x (1-\alpha) \dfrac{\bar{p}^2\proton}{\tilde{D}}
	  -k_x\bar{q}\alpha\beta_x
	  \right] E_0
	  \\
	  \\
	  \tilde{D} & = & 1 + \left[J_6\epsilon_6+J_7\epsilon_7\right] \LiAllOut = 1 + \left[ J_6 \epsilon + J_7 \left(1-\epsilon\right) \right] \LiAllOut = 1 + J_\epsilon \LiAllOut\\
\end{array}
\right.
\end{equation}




\subsection{Looking for steady-state, part II}
We get the system 
\begin{equation}
\left\lbrace
\begin{array}{rcl}
\beta_\infty & = & \dfrac{\Theta+\dfrac{\bar{p}^2 E_0 h_\infty }{\tilde{D}} \left(1-\alpha_\infty\right)}{1+\bar{q}E_0\alpha_\infty}\\
\\
\alpha_\infty & = &
 \dfrac{ \LiAllOut \dfrac{\bar{p}^2 h_\infty }{\tilde{D}} \left( \sum_x k_x \epsilon_x \right)}
 {k_h+\left[\dfrac{\bar{p}^2 h_\infty }{\tilde{D}} + \bar{q}\beta_\infty\right]\left( \sum_x k_x \epsilon_x \right)\LiAllOut}\\
\end{array}
\right.
\end{equation}

OK, I computed them, $\alpha_\infty$ is $\alpha_0$ with a slight decrease if $\bar{q}$ increases...\\
For many reasons, let's assume $\bar{q}E_0\ll 1$, and see if a correction is necessary ??

\section{First Order Resolution}
\subsection{First Order Equations}
\begin{equation}
\left\lbrace
\begin{array}{rcll}
\partial_t \alpha & = & -k_h \alpha +
	  k_\epsilon (1-\alpha) \Upsilon_\alpha \proton , & \Upsilon_\alpha= \dfrac{\bar{p}^2\LiAllOut}{1+J_\epsilon \LiAllOut}\\
	 \\
\partial_t \beta_x & = & k_x\left(\Theta - \beta_x \right) + k_x (1-\alpha) \Upsilon_\beta \proton, & \Upsilon_\beta = \dfrac{\bar{p}^2E_0}{1+J_\epsilon \LiAllOut}\\
\end{array}
\right.
\end{equation}
and using
\begin{equation}
	\hat\alpha = 1-\alpha
\end{equation}

\begin{equation}
\left\lbrace
\begin{array}{rcl}
	\partial_t \hat\alpha + \left(k_h+k_\epsilon \Upsilon_\alpha \proton \right) \hat\alpha & = & k_h\\
	\partial_t \beta_x  + k_x \beta_x  & = & k_x \Theta + k_x \hat\alpha \Upsilon_\beta \proton \\
\end{array}
\right.
\end{equation}
\subsection{First Order Solutions}
We set
\begin{equation}
	\ig(t) = \int_0^{t} \left[ k_h + k_\epsilon \Upsilon_\alpha h(t) \right] \,\mathrm{d}u
\end{equation}
so that
\begin{equation}
\left\lbrace
\begin{array}{rcl}
	\hat\alpha(t) & = & \displaystyle \left[1+ k_h\int_0^t e^{\ig(u)}\,\mathrm{d} u\right] e^{-\ig(t)} \\
	\\
	\beta_x(t)    & = & \displaystyle \Theta \left[ 1-e^{-k_x t} \right] + k_x \Upsilon_\beta \left[ \int_0^t  \hat\alpha(u) h(u) e^{k_x u} \,\mathrm{d} u\right] e^{-k_xt}\\
\end{array}
\right.
\end{equation}

\subsection{Constant pH}
Let's assume
\begin{equation}
	h(t) = h_0, \;\;\omega_0 = k_h + k_\epsilon \Upsilon_\alpha h_0,\;\; \ig(u) = \omega_0 u
\end{equation}
then
\begin{equation}
	\hat\alpha = \dfrac{1}{\omega_0} \left[ k_h + \left(\omega_0-k_h\right)e^{-\omega_0 t}\right]
\end{equation}
so that
\begin{equation}
\left\lbrace
\begin{array}{rcl}
	\beta_x & = & \displaystyle \Theta \left[ 1-e^{-k_x t} \right] + \dfrac{k_x \Upsilon_\beta h_0 }{\omega_0} e^{-k_xt} 
	\int_0^t \left[ k_h e^{k_xu} + \left(\omega_0-k_h\right)e^{ (k_x-\omega_0) u} \right] \,\mathrm{d} u  \\
	\\
	& = & \displaystyle \Theta \left[ 1-e^{-k_x t} \right] + \dfrac{k_h \Upsilon_\beta h_0 }{\omega_0}\left[ 1-e^{-k_x t} \right]
	+ \dfrac{k_x \Upsilon_\beta h_0 }{\omega_0} \dfrac{\omega_0-k_h}{k_x-\omega_0} \left[ e^{-\omega_0t} - e^{-k_xt}\right]\\
	\\
	& = & \left[\Theta + \Upsilon_\beta h_0 \hat\alpha_\infty \right]\left[ 1-e^{-k_x t} \right] 
	+ \Upsilon_\beta h_0 \left(1-\hat\alpha_\infty\right) \left[ k_x \dfrac{e^{-\omega_0t} - e^{-k_xt}}{k_x-\omega_0}\right]
	 \\
\end{array}
\right.
\end{equation}


\end{document}


