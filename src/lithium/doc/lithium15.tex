\documentclass[aps,onecolumn,12pt]{revtex4}
\usepackage{graphicx}
\usepackage{amssymb,amsfonts,amsmath,amsthm}
\usepackage{chemarr}
\usepackage{bm}
\usepackage{pslatex}
\usepackage{xfrac}
\usepackage{xcolor}
\usepackage{bookman}
\usepackage{dsfont}
\usepackage{mathptmx}

\newcommand{\mychem}[1]{\mathtt{#1}}
\newcommand{\myconc}[1]{\left\lbrack{#1}\right\rbrack}

\newcommand{\spLi}[1]{{~^{\mychem{#1}}\mychem{Li}}}
\newcommand{\Li}[1]{\myconc{\spLi{#1}}}

\newcommand{\spEout}{\mychem{E}}
\newcommand{\Eout}{\myconc{\spEout}}

\newcommand{\spLiEin}[1]{\left\lbrace\spLi{#1}\spEout\right\rbrace_{\mathrm{in}}}
\newcommand{\LiEin}[1]{\myconc{\spLiEin{#1}}}

\newcommand{\spLiE}[1]{\left\lbrace\spLi{#1}\spEout\right\rbrace}
\newcommand{\LiE}[1]{\myconc{\spLiE{#1}}}


\newcommand{\spLiEout}[1]{\left\lbrace\spLi{#1}\spEout\right\rbrace_{\mathrm{out}}}
\newcommand{\LiEout}[1]{\myconc{\spLiEout{#1}}}

\newcommand{\spLiIn}[1]{{\spLi{#1}}_{\mathrm{in}}}
\newcommand{\LiIn}[1]{\myconc{\spLiIn{#1}}}

\newcommand{\spLiOut}[1]{{\spLi{#1}}_{\mathrm{out}}}
\newcommand{\LiOut}[1]{\myconc{\spLiOut{#1}}}

\newcommand{\spEHin}{\mychem{EH}}
\newcommand{\EHin}{\myconc{\spEHin}}
\newcommand{\spproton}{\mychem{H}}
\newcommand{\proton}{\myconc{\spproton}}

\newcommand{\mytrn}[1]{{#1}^{\!\mathsf{T}}}
\newcommand{\mymat}[1]{{\bm{#1}}}
\newcommand{\mydet}[1]{{\left|{#1}\right|}}

\newcommand{\ratioLi}{ {\left(\dfrac{\Li{7}}{\Li{6}}\right)} }
\newcommand{\deltaLi}{ {\delta\!\!\!\spLi{7}} }
\newcommand{\deltaLiOut}{{\deltaLi}_{\mathrm{out}}}
\newcommand{\ih}{\ensuremath{\mathbf{H}}}
\newcommand{\ig}{\ensuremath{\mathbf{G}}}

\begin{document}

\section{Isotopic Separation}
$$
	\deltaLi = \left(
		\dfrac{\left(\dfrac{\Li{7}}{\Li{6}}\right)_{sample}}
		{\left(\dfrac{\Li{7}}{\Li{6}}\right)_{standard}}
		 -1 
	\right) \times 1000
$$

$$
	\left(\dfrac{\Li{7}}{\Li{6}}\right)_{sample} = \left(\dfrac{\Li{7}}{\Li{6}}\right)_{standard} \left[1+10^{-3}\deltaLi\right] = \rho_s \left[1+10^{-3}\deltaLi\right]
$$


\section{Proposed Mechanism}

\begin{equation}
	 \spLiOut{x} +  \spEout  
	 \xrightleftharpoons[k_x^d]{k_x^a} 
	 \spLiE{x}
	  \xrightleftharpoons[k_x^q]{\mychem{+H},\;k_x^p} \underbrace{\spEHin}_{\xrightarrow[]{k_h} \mychem{E} + \mychem{H}_{\mathrm{out}}} + \underbrace{\spLiIn{x}}_{\xrightleftharpoons[]{k_x} \spLiOut{x}}
\end{equation}

\section{Scheme}

\subsection{Hypothesis}
\begin{itemize}
\item $\proton$ is a  user's function $h(t)$.
\item $\LiOut{6}$ and  $\LiOut{7}$ are parameters.
\item $k_h$ is constant during the whole experiments for NHE is in its saturated mode for the full range of pH
\end{itemize}

\subsection{Kinetics}
We have the phase space described by
\begin{equation}
 \vec{X} = 
        \begin{pmatrix}
        \Eout\\
        \EHin\\
        \LiE{6}\\
        \LiIn{6}\\
        \LiE{7}\\
        \LiIn{7}\\
        \end{pmatrix}
\end{equation}

At any time, we  have
\begin{equation} 
	\label{eq:E0}
	E_0 = \Eout + \EHin +  \LiE{6} + \LiE{7}
\end{equation}

\subsection{Secondary Hypothesis}
We consider that we have the two equations
\begin{equation}
%\left\lbrace
	\begin{array}{rcll}
	 \spLiOut{x} +  \spEout &  \xrightleftharpoons[]{} & \spLiE{x}, & J_x = \dfrac{\LiE{x}}{\LiOut{x} \Eout} = \dfrac{k_x^a}{k_x^d}\\
	 \end{array}
\end{equation}
leading to a constraint vector $\vec{\Gamma}$
with 
\begin{equation}
	\tilde{J}_x = J_x \LiOut{x}
\end{equation}

\begin{equation}
\vec{\Gamma} = 
\begin{pmatrix}
	\tilde{J}_6 \Eout - \LiE{6} \\
	\tilde{J}_7 \Eout - \LiE{7} \\
\end{pmatrix}
\end{equation}
which already simplifies the matter conservation \eqref{eq:E0} into
\begin{equation}
	E_0 = \EHin + \Eout \left(1+\tilde{J}_6+\tilde{J}_7\right).
\end{equation}
And we have the topology for the two equations as
\begin{equation}
	\label{eq:Nu}
	\mymat{\nu}=\begin{pmatrix}-1 & 0 & 1 & 0 & 0 & 0\cr -1 & 0 & 0 & 0 & 1 & 0\end{pmatrix}
\end{equation}
and
\begin{equation}
	\partial_{\vec{X}}\vec{\Gamma} = 
	\begin{pmatrix}\tilde{J}_6 & 0 & -1 & 0 & 0 & 0\cr\tilde{J}_7 & 0 & 0 & 0 & -1 & 0\end{pmatrix}
\end{equation}

\subsection{Rates}

The "slow" rate vector is
\begin{equation}
	\partial_t\vec{X}_{slow} = 
	\begin{pmatrix}
		v_h\\
		p_6-q_6+p_7-q_7-v_h\\
		q_6-p_6\\
		p_6-l_6-q_6\\
		q_7-p_7\\
		p_7-l_7-q_7\\
	\end{pmatrix}
	,\;\;\text{ with }
	\left\lbrace
	\begin{array}{rcll}
	v_h & = & k_h \EHin & \text{(recycling)}\\
	p_x & = & k_x^p \proton \LiE{x} & \text{(forward transfer)}\\
	q_x & = & k_x^q \EHin \LiIn{x} & \text{(reverse transfer)} \\
	l_x & = & k_x  \left(\LiIn{x}-\tilde{\Theta}_x\right) & \text{(leak)}\\
	\end{array}
	\right.
\end{equation}
with (Goldman-Hodgkins-Katz)
\begin{equation}
	\tilde{\Theta}_x = \Theta \LiOut{x}
\end{equation}

\subsection{Semi-Stationary Equations}
We define
\begin{equation}
	\mymat{W} = \mymat{\Phi}\mytrn{\mymat{\nu}} = \begin{pmatrix} -\tilde{J}_6-1 & -\tilde{J}_6 \cr -\tilde{J}_7 & -\tilde{J}_7-1\end{pmatrix}
	,\;\mymat{W}^\ast = \begin{pmatrix} -\tilde{J}_6-1 & \tilde{J}_7 \cr \tilde{J}_6 & -\tilde{J}_7-1\end{pmatrix}
	,\;\; \tilde{D} =\det(\mymat{W})=1+\tilde{J}_6+\tilde{J}_7.
\end{equation}
and
\begin{equation}
	\mymat{\chi} = \tilde{D}\mathds{1}_6-\mytrn{\mymat{\nu}}\mymat{W}^\ast\mymat{\Phi}
\end{equation}

\begin{equation}
	\partial_t\vec{X} = \dfrac{1}{\tilde{D}}
	\mymat{\chi} \partial_t\vec{X}_{slow}
\end{equation}
and we find
\begin{equation}
	\vec{Y} = \begin{pmatrix} \EHin \cr \LiIn{6} \cr \LiIn{7} \end{pmatrix}
	,\;\partial_t \vec{Y} = 
	\begin{pmatrix}
	p_6-q_6+p_7-q_7-v_h\\
	p_6-q_6-l_6\\
	p_7-q_7-l_7
	\end{pmatrix}
\end{equation}
with the expressions
\begin{equation}
\left\lbrace
	\begin{array}{rcl}
	v_h & = & k_h \EHin \\
	q_x & = & k_x^q \EHin \Li{x}  \\
	l_x & = & k_x  \left(\Li{x}- \tilde{\Theta}_x\right)\\
	p_x & = & k_x^p \proton \LiE{x}\\
	\end{array}
\right.
\end{equation}
with
\begin{equation}
	\LiE{x} = \tilde{J}_x \Eout,\;\;\Eout=\dfrac{E_0-\EHin}{\tilde{D}}
\end{equation}
so that
\begin{equation}
	p_x = k_x^p \proton  \tilde{J}_x \dfrac{E_0-\EHin}{\tilde{D}}
\end{equation}
We now use
\begin{equation}
	\begin{array}{rcl}
	\alpha    & = &\dfrac{\EHin}{E_0}\\
	%\lambda_x & = & \LiIn{x}\\
	\end{array}
\end{equation}
and we get
\begin{equation}
	\left\lbrace
	\begin{array}{rcl}
	v_h & = & k_h E_0 \alpha \\
	q_x & = & k_x^q E_0 \alpha \Li{x}  \\
	l_x & = & k_x \left(\Li{x} - \tilde{\Theta}_x\right)\\
	p_x & = & \underbrace{k_x^p  \dfrac{\tilde{J}_x}{\tilde{D}}}_{\Upsilon_x} E_0 \proton(1-\alpha) \\
	\Upsilon_0 & = & \Upsilon_6 + \Upsilon_7 \;\;\left(\text{ in M$^{-1}$.s$^{-1}$}\right) \\
	\end{array}
\right.
\end{equation}


%We define the set of reduced constants
%\begin{equation}
%\left\lbrace
%	\begin{array}{rcl}
%	\omega_x & = &  k_x^p \proton \dfrac{J_x}{\tilde{D}}\LiOut{x}\\
%	\omega_0 & = & \omega_6 + \omega_7\\
%	p_x & = & \omega_x E_0 \left(1-\alpha\right)\\
%	\end{array}
%\right.
%\end{equation}
%


then we get the \underline{three} coupled equations (TODO: not using $\Upsilon...$)
%\begin{equation}
%\boxed{
%\left\lbrace
%	\begin{array}{rcl}
%		\partial_t\alpha & = & \Upsilon_0 h - \alpha\left\lbrack k_h+\Upsilon_0 h +{\sum_x k_x^q \Li{x}} \right\rbrack\\
%		\partial_t\Li{x} & = & \left\lbrack k_x\tilde{\Theta}_x+\Upsilon_x h  E_0\right\rbrack
%		-\left\lbrack
%			\Upsilon_x h  E_0\alpha +  \Li{x}  \left(k_x + k_x^q E_0\alpha\right)
%		\right\rbrack\\
%	\end{array}
%\right.
%}
%\end{equation}

\begin{equation}
\boxed{
\left\lbrace
	\begin{array}{rcl}
		\partial_t\alpha & = & -k_h + \left(1- \alpha\right)\left\lbrack k_h+\Upsilon_0 h\right] - \alpha \left\lbrack {\sum_x k_x^q \Li{x}} \right\rbrack\\
		\partial_t\Li{x} & = & k_x \left(\tilde{\Theta}_x -\Li{x} \right)  + \left(1-\alpha\right) \Upsilon_x h  E_0 - \alpha E_0 k_x^q \Li{x}\\
	\end{array}
\right.
}
\end{equation}


\subsection{Looking for steady-state}
\subsubsection{Constraints}
\begin{equation}
	\Li{x}_\infty = \dfrac{\left\lbrack k_x\tilde{\Theta}_x+\Upsilon_x h_\infty E_0 \left(1-\alpha_\infty\right)\right\rbrack}{k_x+ k_x^q E_0\alpha_\infty}
\end{equation}
Since experimentally, there exist ${\beta}$ (observed) and $\Theta$ (GHK) such that
\begin{equation}
	\Li{x}_\infty=\beta\LiOut{x},\;\tilde{\Theta}_x = \Theta \LiOut{x}
\end{equation}
we obtain that
\begin{equation}
		\beta = \dfrac{k_x\Theta + k_x^p h_\infty {J_x} E_0 (1-\alpha_\infty)/\tilde{D}}{k_x+ k_q^x E_0\alpha_\infty}
\end{equation}
\textit{which must be the same for both species!}

We have
\begin{equation}
\left\lbrace
	\begin{array}{rcll}
	k_x^q    & = & \bar{q}\,k_x,  &  \bar{q}\text{ inverse of concentration}\\
	k_x^pJ_x & = & \bar{p}^2\,k_x, & \bar{p}\text{ inverse of concentration}\\
	\end{array}
\right.
\end{equation}
leading to
\begin{equation}
	\label{steady_beta}
	\beta = \dfrac{\Theta+\dfrac{\bar{p}^2 h_\infty E_0}{1+J_6\LiOut{6}+J_7\LiOut{7}} \left(1-\alpha_\infty\right)}{1+\bar{q}E_0\alpha_\infty}
\end{equation}
which read exactly as the GHK level (electroosmotic leaks) shifted by NHE intake and reduced by NHE output.

\centerline{\bf This has some meaning: leak modulated isotopic separation!!!}


\subsubsection{Expression}
OK, I computed them, $\alpha_\infty$ is $\alpha_0$ with a slight decrease if $\bar{q}$ increases...\\
For many reasons, let's assume $\bar{q}E_0\ll 1$, and see if a correction is necessary ??

\subsection{Unified rewrite}
\begin{equation}
\begin{array}{rcl}
	%\Upsilon_x & = & \bar{p}^2 k_x \dfrac{\LiOut{x}}{\tilde{D}}\\
	\beta_x    & = & \dfrac{\Li{x}}{\LiOut{x}}\\
\end{array}
\end{equation}

\begin{equation}
\boxed{
\left\lbrace
	\begin{array}{rcl}
		\partial_t\alpha    & = & 
		-k_h 
		+ \left(1-\alpha\right) \left[ k_h + h \dfrac{\bar{p}^2}{\tilde{D}} \left(\sum_x k_x \LiOut{x}\right) \right]
		- \alpha \bar{q} \left( {\sum_x k_x \Li{x}} \right)\\
		\partial_t\beta_x & = & k_x \left[
		\left(\Theta  - \beta_x \right)
		+ \left(1-\alpha\right) h E_0 \dfrac{\bar{p}^2}{\tilde{D}} 
		- \alpha E_0 \bar{q} \beta_x
		\right]
		\\
	\end{array}
\right.
}
\end{equation}



\section{First Order Resolution}

\subsection{First Order Equations}

\begin{equation}
\left\lbrace
\begin{array}{rcl}
\partial_t \alpha & = & \Upsilon_0 h - \alpha\left\lbrack k_h+\Upsilon_0 h  \right\rbrack\\
\partial_t\Li{x} & = & \left\lbrack k_x\tilde{\Theta}_x+\Upsilon_x h  E_0\right\rbrack
		-\left\lbrack
			\Upsilon_x h  E_0\alpha +  k_x\Li{x}  %\left(k_x + k_x^q E_0\alpha\right)
		\right\rbrack\\
\end{array}
\right.
\end{equation}

and using 
\begin{equation}
	\beta_x = \dfrac{\Li{x}}{\LiOut{x}}
\end{equation}
we get
\begin{equation}
\left\lbrace
\begin{array}{rcl}
\partial_t \alpha & = & \Upsilon_0 h - \alpha\left\lbrack k_h+\Upsilon_0 h  \right\rbrack\\
\partial_t\beta_x & = & k_x \left\lbrack
{\Theta + \underbrace{\dfrac{\bar{p}^2}{\tilde{D}}}_{\kappa} h E_0  \left(1-\alpha\right)}
- \beta_x
\right\rbrack \\
\end{array}
\right.
\end{equation}
and we reckon from \eqref{steady_beta} and the first order assumption
\begin{equation}
	\Theta + \kappa h_\infty E_0  \left(1-\alpha_\infty\right) = \beta 
\end{equation}

\subsection{First Order Solving}

We make a change in equation by defining
\begin{equation}
	\hat{\alpha} = 1-\alpha
\end{equation}
Leading to
\begin{equation}
\displaystyle
\left\lbrace
\begin{array}{rcl}
\partial_t \hat\alpha + \left\lbrack k_h+\Upsilon_0 h  \right\rbrack \hat{\alpha} & = & k_h\\
\partial_t \beta_x + k_x \beta_x & = & k_x \kappa E_0 h \hat\alpha \\
\end{array}
\right.
\end{equation}

We also define
\begin{equation}
%\displaystyle
\left\lbrace
\begin{array}{rcl}
g(t) & = & k_h + \Upsilon_0 h(t) \\
\\
\ih  & = & \displaystyle \Upsilon_0\int_0^t h(u) \, \mathrm{d}u\\
\\
\ig  & = &  \displaystyle \int_0^t g(u) \, \mathrm{d}u = k_h t +  \ih(t)\\
\end{array}
\right.
\end{equation}
We find
\begin{equation}
	\hat\alpha = \left[ 1 + k_h \int_0^t e^{\ig(u)}\,\mathrm{d}u \right] e^{-\ig(t)}
\end{equation}
with
\begin{equation}
	\hat\alpha(0) = 1,\;\;\hat\alpha_\infty = \dfrac{k_h}{k_h+\Upsilon_0 h_\infty}
\end{equation}
and after that
\begin{equation}
\beta_x = k_x\left[\int_0^t \kappa E_0 h(t) \hat\alpha(t) e^{k_xu} \, \mathrm{d}u\right] e^{-k_xt}
\end{equation}

And we remind that
\begin{equation}
\left\lbrace
\begin{array}{rcl}
	\dfrac{\beta_7}{\beta_6} & = & \dfrac{1+10^{-3}\deltaLi}{1+10^{-3}\deltaLi_{out}} \\
	\\
	\deltaLi & = & 10^3 \left( \left[1+10^{-3}\deltaLi_{out}\right] \dfrac{\beta_7}{\beta_6} - 1 \right)\\
\end{array}
\right.
\end{equation}

\subsection{First Order Short Times}
We assume that for a certain period of time, we have about $h(t)\simeq h_0$.
Leading to
\begin{equation}
	\ig_0(t) = (k_h+\Upsilon_0h_0) t = \omega_0 t
\end{equation}
then
\begin{equation}
\hat\alpha_0(t) = \dfrac{1}{\omega_0}\left[ k_h + \left(\omega_0 - k_h\right) e^{-\omega_0t}\right]
\end{equation}
and
\begin{equation}
\left\lbrace
\begin{array}{rcl}
\beta_x(t) & = & \displaystyle
\dfrac{k_x \kappa E_0 h_0}{\omega_0}  e^{-k_xt}  \left(
\int_0^t \left[ k_h e^{k_xu} + \left(\omega_0-k_h\right) e^{\left(k_x-\omega_0\right)u}\right] \, \mathrm{d} u\right) \\
\\
 & = & 
 \displaystyle
\dfrac{k_x \kappa E_0 h_0}{\omega_0}  e^{-k_xt}  \left(
\dfrac{k_h}{k_x} \left[ e^{k_xt} -1 \right] + \left(\omega_0-k_h\right) \dfrac{ e^{\left(k_x-\omega_0\right)t}-1}{k_x-\omega_0}
\right)
 \\
 \\
  & = & \dfrac{\kappa E_0 h_0}{\omega_0} 
  \left( 
  k_h \left[1-e^{-k_xt}\right] 
  + \left(\omega_0-k_h\right) k_x \dfrac{e^{-\omega_0t} - e^{-k_xt}}{k_x-\omega_0}
  \right) \\
  & = & 
 \dfrac{\kappa E_0 h_0 k_h}{\omega_0} 
  \left( 
   \left[1-e^{-k_xt}\right] 
  + \left(\dfrac{\omega_0}{k_h}-1\right) k_x \dfrac{e^{-\omega_0t} - e^{-k_xt}}{k_x-\omega_0}
  \right) \\\end{array}
\right.
\end{equation}
Using
\begin{equation}
\left\lbrace
\begin{array}{rcl}
	R(\tau) & = &1-e^{-\tau} \\
	 B(\sigma,\tau) & = & \dfrac{e^{-\sigma\tau}-e^{-\tau}}{1-\sigma} \;\;\;\; (B(1,\tau)=\tau e^{-\tau})\\
	 %\varphi        & = & \dfrac{\omega_0}{k_h} - 1 \\
\end{array}
\right.
\end{equation}
we get
\begin{equation}
	\beta_x(t) =  \dfrac{\kappa E_0 h_0 k_h}{\omega_0} \left[
	R(k_xt) + \left( \dfrac{\omega_0}{k_h} - 1 \right) B(\omega_0/k_x,k_xt)
	\right]
\end{equation}
We finally define
\begin{equation}
\left\lbrace
\begin{array}{rcl}
	k_6 & = & \lambda k_7\\
	\omega_0 & = & \sigma k_7 \\
	\tau & = & k_7 t\\
	\eta & = & \dfrac{k_7}{k_h} \\
\end{array}
\right.
\end{equation}
to express
\begin{equation}
		\dfrac{\beta_7}{\beta_6}
		 = \dfrac{ R(\tau) + \left(\sigma\eta-1\right) B(\sigma,\tau) }{R(\lambda \tau) + \left(\sigma\eta-1\right) B(\frac{\sigma}{\lambda},\lambda \tau)}
\simeq \dfrac{1}{\lambda} + \dfrac{1}{2} \left(1-\dfrac{1}{\lambda}\right) \tau + \ldots
\end{equation}

\end{document}
