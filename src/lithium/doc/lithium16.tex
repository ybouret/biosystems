\documentclass[aps,onecolumn,12pt]{revtex4}
\usepackage{graphicx}
\usepackage{amssymb,amsfonts,amsmath,amsthm}
\usepackage{chemarr}
\usepackage{bm}
\usepackage{pslatex}
\usepackage{xfrac}
\usepackage{xcolor}
\usepackage{bookman}
\usepackage{dsfont}
\usepackage{mathptmx}

\newcommand{\mychem}[1]{\mathtt{#1}}
\newcommand{\myconc}[1]{\left\lbrack{#1}\right\rbrack}

\newcommand{\spLi}[1]{{~^{\mychem{#1}}\mychem{Li}}}
\newcommand{\Li}[1]{\myconc{\spLi{#1}}}

\newcommand{\spEout}{\mychem{E}}
\newcommand{\Eout}{\myconc{\spEout}}

\newcommand{\spLiEin}[1]{\left\lbrace\spLi{#1}\spEout\right\rbrace_{\mathrm{in}}}
\newcommand{\LiEin}[1]{\myconc{\spLiEin{#1}}}

\newcommand{\spLiE}[1]{\left\lbrace\spLi{#1}\spEout\right\rbrace}
\newcommand{\LiE}[1]{\myconc{\spLiE{#1}}}


\newcommand{\spLiEout}[1]{\left\lbrace\spLi{#1}\spEout\right\rbrace_{\mathrm{out}}}
\newcommand{\LiEout}[1]{\myconc{\spLiEout{#1}}}

\newcommand{\spLiIn}[1]{{\spLi{#1}}_{\mathrm{in}}}
\newcommand{\LiIn}[1]{\myconc{\spLiIn{#1}}}

\newcommand{\spLiOut}[1]{{\spLi{#1}}_{\mathrm{out}}}
\newcommand{\LiOut}[1]{\myconc{\spLiOut{#1}}}

\newcommand{\spEHin}{\mychem{EH}}
\newcommand{\EHin}{\myconc{\spEHin}}
\newcommand{\spproton}{\mychem{H}}
\newcommand{\proton}{\myconc{\spproton}}

\newcommand{\mytrn}[1]{{#1}^{\!\mathsf{T}}}
\newcommand{\mymat}[1]{{\bm{#1}}}
\newcommand{\mydet}[1]{{\left|{#1}\right|}}

\newcommand{\ratioLi}{ {\left(\dfrac{\Li{7}}{\Li{6}}\right)} }
\newcommand{\deltaLi}{ {\delta\!\!\!\spLi{7}} }
\newcommand{\deltaLiOut}{{\deltaLi}_{\mathrm{out}}}
\newcommand{\ih}{\ensuremath{\mathbf{H}}}
\newcommand{\ig}{\ensuremath{\mathbf{G}}}

\newcommand{\LiAll}{\Lambda}
\newcommand{\LiAllOut}{{\LiAll}_{\mathrm{out}}}

\begin{document}

\section{Isotopic Separation}
$$
	\deltaLi = \left(
		\dfrac{\left(\dfrac{\Li{7}}{\Li{6}}\right)_{sample}}
		{\left(\dfrac{\Li{7}}{\Li{6}}\right)_{standard}}
		 -1 
	\right) \times 1000
$$

$$
	\left(\dfrac{\Li{7}}{\Li{6}}\right)_{sample} = \left(\dfrac{\Li{7}}{\Li{6}}\right)_{standard} \left[1+10^{-3}\deltaLi\right] = \beta_s \left[1+10^{-3}\deltaLi\right]
$$

\begin{equation}
\left\lbrace
\begin{array}{rcl}
	\LiAll    & = & \Li{6} + \Li{7}\\
	\LiAllOut & = & \LiOut{6} + \LiOut{7}\\
\end{array}
\right.
\end{equation}
and
\begin{equation}
\left\lbrace
\begin{array}{rclcl}
	\LiOut{6} & = & \dfrac{1}{1+\beta_s \left[1+10^{-3}\deltaLiOut\right] } \LiAllOut & = & \epsilon_6 \LiAllOut  = \epsilon \LiAllOut \\
	\\
	\LiOut{7} & = & \dfrac{\beta_s \left[1+10^{-3}\deltaLi\right]}{1+\beta_s \left[1+10^{-3}\deltaLiOut\right] } \LiAllOut & = & \epsilon_7 \LiAll,\;\epsilon_7 = 1-\epsilon \\
\end{array}
\right.
\end{equation}
with, for the experiments,
\begin{equation}
	\beta_s = 12.0192
\end{equation}
\begin{equation}
	\epsilon^\mathrm{out} \simeq 0.076, 1-\epsilon^\mathrm{out} \simeq 0.924
\end{equation}

\section{Proposed Mechanism}

\begin{equation}
	 \spLiOut{x} +  \spEout  
	 \xrightleftharpoons[k_x^d]{k_x^a} 
	 \spLiE{x}
	  \xrightleftharpoons[k_x^q]{\mychem{+H},\;k_x^p} \underbrace{\spEHin}_{\xrightarrow[]{k_h} \mychem{E} + \mychem{H}_{\mathrm{out}}} + \underbrace{\spLiIn{x}}_{\xrightleftharpoons[]{k_x} \spLiOut{x}}
\end{equation}

\section{Scheme}

\subsection{Hypothesis}
\begin{itemize}
\item $\proton$ is a  user's function $h(t)$.
\item $\LiOut{6}$ and  $\LiOut{7}$ are parameters.
\item $k_h$ is constant during the whole experiments for NHE is in its saturated mode for the full range of pH
\end{itemize}

\subsection{Kinetics}
We have the phase space described by
\begin{equation}
 \vec{X} = 
        \begin{pmatrix}
        \Eout\\
        \EHin\\
        \LiE{6}\\
        \LiIn{6}\\
        \LiE{7}\\
        \LiIn{7}\\
        \end{pmatrix}
\end{equation}

At any time, we  have
\begin{equation} 
	\label{eq:E0}
	E_0 = \Eout + \EHin +  \LiE{6} + \LiE{7}
\end{equation}

\subsection{Secondary Hypothesis}
We consider that we have the two equations
\begin{equation}
%\left\lbrace
	\begin{array}{rcll}
	 \spLiOut{x} +  \spEout &  \xrightleftharpoons[]{} & \spLiE{x}, & J_x = \dfrac{\LiE{x}}{\LiOut{x} \Eout} = \dfrac{k_x^a}{k_x^d}\\
	 \end{array}
\end{equation}
leading to a constraint vector $\vec{\Gamma}$
with 
\begin{equation}
	\tilde{J}_x = J_x \LiOut{x}
\end{equation}

\begin{equation}
\vec{\Gamma} = 
\begin{pmatrix}
	\tilde{J}_6 \Eout - \LiE{6} \\
	\tilde{J}_7 \Eout - \LiE{7} \\
\end{pmatrix}
\end{equation}
which already simplifies the matter conservation \eqref{eq:E0} into
\begin{equation}
	E_0 = \EHin + \Eout \left(1+\tilde{J}_6+\tilde{J}_7\right).
\end{equation}
And we have the topology for the two equations as
\begin{equation}
	\label{eq:Nu}
	\mymat{\nu}=\begin{pmatrix}-1 & 0 & 1 & 0 & 0 & 0\cr -1 & 0 & 0 & 0 & 1 & 0\end{pmatrix}
\end{equation}
and
\begin{equation}
	\partial_{\vec{X}}\vec{\Gamma} = 
	\begin{pmatrix}\tilde{J}_6 & 0 & -1 & 0 & 0 & 0\cr\tilde{J}_7 & 0 & 0 & 0 & -1 & 0\end{pmatrix}
\end{equation}

\subsection{Rates}

The "slow" rate vector is
\begin{equation}
	\partial_t\vec{X}_{slow} = 
	\begin{pmatrix}
		v_h\\
		p_6-q_6+p_7-q_7-v_h\\
		q_6-p_6\\
		p_6-l_6-q_6\\
		q_7-p_7\\
		p_7-l_7-q_7\\
	\end{pmatrix}
	,\;\;\text{ with }
	\left\lbrace
	\begin{array}{rcll}
	v_h & = & k_h \EHin & \text{(recycling)}\\
	p_x & = & k_x^p \proton \LiE{x} & \text{(forward transfer)}\\
	q_x & = & k_x^q \EHin \LiIn{x} & \text{(reverse transfer)} \\
	l_x & = & k_x  \left(\LiIn{x}-\tilde{\Theta}_x\right) & \text{(leak)}\\
	\end{array}
	\right.
\end{equation}
with (Goldman-Hodgkins-Katz)
\begin{equation}
	\tilde{\Theta}_x = \Theta \LiOut{x}
\end{equation}

\subsection{Semi-Stationary Equations}
We define
\begin{equation}
	\mymat{W} = \mymat{\Phi}\mytrn{\mymat{\nu}} = \begin{pmatrix} -\tilde{J}_6-1 & -\tilde{J}_6 \cr -\tilde{J}_7 & -\tilde{J}_7-1\end{pmatrix}
	,\;\mymat{W}^\ast = \begin{pmatrix} -\tilde{J}_6-1 & \tilde{J}_7 \cr \tilde{J}_6 & -\tilde{J}_7-1\end{pmatrix}
	,\;\; \tilde{D} =\det(\mymat{W})=1+\tilde{J}_6+\tilde{J}_7.
\end{equation}
and
\begin{equation}
	\mymat{\chi} = \tilde{D}\mathds{1}_6-\mytrn{\mymat{\nu}}\mymat{W}^\ast\mymat{\Phi}
\end{equation}

\begin{equation}
	\partial_t\vec{X} = \dfrac{1}{\tilde{D}}
	\mymat{\chi} \partial_t\vec{X}_{slow}
\end{equation}
and we find
\begin{equation}
	\vec{Y} = \begin{pmatrix} \EHin \cr \LiIn{6} \cr \LiIn{7} \end{pmatrix}
	,\;\partial_t \vec{Y} = 
	\begin{pmatrix}
	p_6-q_6+p_7-q_7-v_h\\
	p_6-q_6-l_6\\
	p_7-q_7-l_7
	\end{pmatrix}
\end{equation}
with the expressions
\begin{equation}
\left\lbrace
	\begin{array}{rcl}
	v_h & = & k_h \EHin \\
	q_x & = & k_x^q \EHin \Li{x}  \\
	l_x & = & k_x  \left(\Li{x}- \tilde{\Theta}_x\right)\\
	p_x & = & k_x^p \proton \LiE{x}\\
	\end{array}
\right.
\end{equation}
with
\begin{equation}
	\LiE{x} = \tilde{J}_x \Eout,\;\;\Eout=\dfrac{E_0-\EHin}{\tilde{D}}
\end{equation}
so that
\begin{equation}
	p_x = k_x^p \proton  \tilde{J}_x \dfrac{E_0-\EHin}{\tilde{D}}
\end{equation}

then we get the \underline{three} coupled equations  

\begin{equation}
%\boxed{
\left\lbrace
	\begin{array}{rcl}
		\partial_t\EHin & = & -k_h \EHin + \left(E_0- \EHin\right) \dfrac{\proton}{\tilde{D}} \left(\sum_x k_x^p \tilde{J}_x \right)  
		- \EHin \left\lbrack {\sum_x k_x^q \Li{x}} \right\rbrack\\
		& = & 
		-k_h E_0+ \left(E_0- \EHin\right)\left\lbrack k_h+ \dfrac{\proton}{\tilde{D}} \left(\sum_x k_x^p \tilde{J}_x \right)\right] 
		- \EHin \left\lbrack {\sum_x k_x^q \Li{x}} \right\rbrack\\
		\partial_t\Li{x} & = & k_x \left(\tilde{\Theta}_x -\Li{x} \right)  + \left(E_0-\EHin\right) \dfrac{\proton}{\tilde{D}}   k_x^p \tilde{J}_x  - \EHin k_x^q \Li{x}\\
	\end{array}
\right.
%}
\end{equation}
which can be turned into an \textbf{unified} set of equations using
\begin{equation}
\left\lbrace
\begin{array}{rcl}
	\alpha & = & \dfrac{\EHin}{E_0}\\
	\\
	\hat\alpha & = & 1-\alpha\\
	\\
	\beta_x & = & \dfrac{\Li{x}}{\LiOut{x}}\\
	\\
	\Lambda & = & \Li{6} + \Li{7} = \left(\epsilon_6 \beta_6 + \epsilon_7 \beta_7\right) \LiAllOut = \left[\epsilon \beta_6 + (1-\epsilon) \beta_7\right] \LiAllOut\\
\end{array}
\right.
\end{equation}
and obviously
\begin{equation}
	\deltaLi = 1000 \left ( \left[1+10^{-3}\deltaLiOut\right] \dfrac{\beta_7}{\beta_6}-1\right)
\end{equation}
or
\begin{equation}
	\dfrac{\beta_7}{\beta_6} = \dfrac{\left[1+10^{-3}\deltaLi\right]}{\left[1+10^{-3}\deltaLiOut\right]}
\end{equation}

\begin{equation}
%\boxed{
\left\lbrace
\begin{array}{rcl}
\partial_t\alpha & = &
 -k_h \alpha 
 + \left(1-\alpha\right)\proton \dfrac{\left(\sum_x k_x^p \epsilon_x J_x \right)\LiAllOut}{1+\left(\sum_x J_x \epsilon_x \right) \LiAllOut}
 - \alpha \left(\sum_x k_x^q \epsilon_x \beta_x \right) \LiAllOut\\
 \\
 \partial_t \hat\alpha & = & k_h - \hat\alpha\left(k_h+\proton\dfrac{\left(\sum_x k_x^p \epsilon_x J_x \right)\LiAllOut}{1+\left(\sum_x J_x \epsilon_x \right) \LiAllOut}\right) + (1-\hat\alpha)\left(\sum_x k_x^q \epsilon_x \beta_x \right) \LiAllOut
 \\\\
 \partial_t \beta_ x & = & k_x \left( \Theta - \beta_x \right) 
 + \left(1-\alpha\right)\proton \dfrac{k_x^p  J_x E_0 }{1+\left(\sum_x J_x \epsilon_x \right) \LiAllOut} 
 - \alpha \ k_x^q  \beta_x  E_0\\
 \\
 & = &  k_x \left( \Theta - \beta_x \right) 
 + \hat \alpha \proton \dfrac{k_x^p  J_x E_0 }{1+\left(\sum_x J_x \epsilon_x \right) \LiAllOut} 
 - \left(1-\hat\alpha\right)\ k_x^q  \beta_x  E_0\\
\end{array}
\right.
%}
\end{equation}
and we will define the scalings
\begin{equation}
\left\lbrace
\begin{array}{ccll}
	k_x^p J_x & = & k_x \bar{p}_x^2, & \bar{p}_x \text{ in M}^{-1} \\
	k_x^q     & = & k_x \bar{q}_x  , & \bar{q}_x \text{ in M}^{-1} \\
\end{array}
\right.
\end{equation}

\section{Steady State}
TODO?

\section{First Order}
\subsection{First Order $\alpha$}

\subsubsection{Generic $\proton$}
We get
\begin{equation}
	\partial_t \hat\alpha = k_h - \hat\alpha\left( k_h+ \Upsilon_\alpha \proton \right) = k_h - \hat\alpha g(t)
\end{equation}
We define
\begin{equation}
	\ig(t) = \int_0^t g(u) \,\mathrm{d} u
\end{equation}
to get
\begin{equation}
	\hat \alpha_1(t) = \left[ 1 + k_h \int_0^t e^{\ig(u)}  \,\mathrm{d} u \right] e^{-\ig(t)}
\end{equation}
\subsubsection{Constant $\proton$}
\begin{equation}
	\proton=h_0,\;\;g(t) = (k_h+\Upsilon_\alpha h_0)=\omega_0,\;\;\ig(t) = \omega_0 t
\end{equation}
\begin{equation}
	\hat\alpha_0(t) = e^{-\omega_0t} + \dfrac{k_h}{\omega_0}\left( 1 - e^{-\omega_0 t}\right) 
	= \dfrac{1}{\omega_0} \left[k_h + \left(\omega_0-k_h\right) e^{-\omega_0 t} \right] = \hat\alpha_\infty + \left(1-\hat\alpha_\infty\right) e^{-\omega_0 t}
\end{equation}

\subsubsection{Initial and final values}
In any case
\begin{equation}
	\hat\alpha(0) = 1, \;\; \hat\alpha_\infty = \dfrac{k_h}{k_h+h_\infty\Upsilon_\alpha}
\end{equation}


\subsection{First Order $\beta$}
\subsubsection{Generic $\proton$}
\begin{equation}
\partial_t\beta_x + k_x \beta_x = k_x\Theta + \hat{\alpha}(t)h(t) \Upsilon_x
\end{equation}
leading to
\begin{equation}
\left\lbrace
\begin{array}{rcl}
\beta_{x,1} & = & \displaystyle \left( \int_0^t \left[k_x\Theta + \hat{\alpha}_1(u)h(u) \Upsilon_x \right]e^{k_xu}\,\mathrm{d}u \right) e^{-k_x t}\\
\\
 & = & \displaystyle \Theta \left[1-e^{-k_xt}\right] + \left(\Upsilon_x \int_0^t \hat{\alpha}_1(u)h(u) e^{k_xu}\,\mathrm{d}u \right)  e^{-k_x t}
\end{array}
\right.
\end{equation}

\subsubsection{Constant $\proton$}
\begin{equation}
\left\lbrace
\begin{array}{rcl}
\beta_{x,0} & = & \displaystyle \Theta \left[1-e^{-k_xt}\right] +h_0 \Upsilon_x
\left( \int_0^t \left[\hat\alpha_\infty + \left(1-\hat\alpha_\infty\right) e^{-\omega_0 u} \right]e^{k_xu} \,\mathrm{d}u \right) e^{-k_x t} \\
\\
& = & \displaystyle \Theta \left[1-e^{-k_xt}\right] +
h_0 \Upsilon_x 
	\left( 
		\dfrac{\hat\alpha_\infty}{k_x}\left[1-e^{-k_xt}\right]  
		+ \dfrac{1-\hat\alpha_\infty}{k_x - \omega_0}\left[e^{-\omega_0t} - e^{-k_xt}\right]
	\right)\\
\\
& = & \left(\Theta+\dfrac{h_0\Upsilon_x}{k_x} \hat{\alpha}_\infty\right) \left[1-e^{-k_xt}\right] + (1-\hat\alpha_\infty) \dfrac{h_0\Upsilon_x}{k_x} \left[ k_x \dfrac{e^{-\omega_0t} - e^{-k_xt}}{k_x - \omega_0}\right]
\end{array}
\right.
\end{equation}


\subsection{Time Scaling}
\begin{equation}
\left\lbrace
\begin{array}{rcl}
	k_6      & = & \sigma k_7\\
	\omega_0 & = & \Omega k_7\\
	\tau     & = & k_7 t\\
\end{array}
\right.
\end{equation}
We get
\begin{equation}
\left\lbrace
\begin{array}{rcl}
	\hat\alpha(\tau)  & = & \hat\alpha_\infty + \left(1-\hat\alpha_\infty\right) e^{-\Omega \tau}\\
	\\
	\beta_{7,0}(\tau) & = & \left( \Theta + \hat\alpha_\infty A_7 \right)  \left[1-e^{-\tau}\right]
	+ \left(1-\hat\alpha_\infty\right) A_7 \left[\dfrac{e^{-\Omega\tau} - e^{-\tau}}{1-\Omega}\right]\\
	\\
	\beta_{6,0}(\tau) & = & \left( \Theta + \hat\alpha_\infty A_6 \right)  \left[1-e^{-\sigma\tau}\right]
	+ \left(1-\hat\alpha_\infty\right) A_6 \left[\dfrac{e^{-\frac{\Omega}{\sigma}\sigma\tau} - e^{-\sigma\tau}}{1-\dfrac{\Omega}{\sigma}}\right]\\
\end{array}
\right.
\end{equation}

We find
\begin{equation}
\left\lbrace
\begin{array}{ccl}
	J_\epsilon & = & \epsilon J_6 + (1-\epsilon) J_7 \\
	\\
	A_x & = & \dfrac{h_0\Upsilon_x}{k_x} = \dfrac{\dfrac{k_x^p J_x}{k_x}   E_0 h_0}{1+\left(\sum_x J_x \epsilon_x \right) \LiAllOut} =  \dfrac{\bar{p}_x^2}{1+J_\epsilon \LiAllOut} h_0 E_0 \\
	\\
	h_0\Upsilon_\alpha & = & \dfrac{\left(\sum_x k_x^p \epsilon_x J_x \right)h_0\LiAllOut}{1+J_\epsilon\LiAllOut} = 
	 k_7 \dfrac{\left[\sigma \epsilon \bar{p}_6^2 + (1-\epsilon) \bar{p}_7^2 \right] h_0 \LiAllOut }{1+J_\epsilon \LiAllOut}  \\
	 \\
	\hat{\alpha}_\infty & = & \cos^2 \phi\\
\end{array}
\right.
\end{equation}

\subsection{First order Results}

\subsubsection{Passive Separation, valid for any case}
Without enzyme, and for any outer lithium concentration (even extrapolated to 0)
\begin{equation}
	\dfrac{\beta_7^\star}{\beta_6^\star} = \dfrac{  \left[1-e^{-\tau}\right]}{  \left[1-e^{-\sigma\tau}\right]}
\end{equation}

\subsubsection{Constant $\proton$ for initial times}
\begin{equation}
	B(\lambda,\tau) = \dfrac{e^{-\lambda\tau}-e^{-\tau}}{1-\lambda},\;\;B(1,\tau) = \tau e^{-\tau}
\end{equation}

The first order equation is
\begin{equation}
	\dfrac{\beta_{7,0}}{\beta_{6,0}} = 
	\dfrac
	{\left( \Theta + A_7\cos^2\phi \right)  \left[1-e^{-\tau}\right]+ A_7B(\Omega,\tau) \sin^2\phi }
	{\left( \Theta + A_6\cos^2\phi \right)  \left[1-e^{-\sigma\tau}\right]+ A_6B(\frac{\Omega}{\sigma},\sigma\tau) \sin^2\phi }
\end{equation}

\end{document}


