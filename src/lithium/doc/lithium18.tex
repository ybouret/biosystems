\documentclass[aps,onecolumn,12pt]{revtex4}
\usepackage{graphicx}
\usepackage{amssymb,amsfonts,amsmath,amsthm}
\usepackage{chemarr}
\usepackage{bm}
\usepackage{pslatex}
\usepackage{xfrac}
\usepackage{xcolor}
\usepackage{bookman}
\usepackage{dsfont}
\usepackage{mathptmx}

\newcommand{\mychem}[1]{\mathtt{#1}}
\newcommand{\myconc}[1]{\left\lbrack{#1}\right\rbrack}

\newcommand{\spLi}[1]{{~^{\mychem{#1}}\mychem{Li}}}
\newcommand{\Li}[1]{\myconc{\spLi{#1}}}

\newcommand{\spEout}{\mychem{E}}
\newcommand{\Eout}{\myconc{\spEout}}

\newcommand{\spLiEin}[1]{\left\lbrace\spLi{#1}\spEout\right\rbrace_{\mathrm{in}}}
\newcommand{\LiEin}[1]{\myconc{\spLiEin{#1}}}

\newcommand{\spLiE}[1]{\left\lbrace\spLi{#1}\spEout\right\rbrace}
\newcommand{\LiE}[1]{\myconc{\spLiE{#1}}}


\newcommand{\spLiEout}[1]{\left\lbrace\spLi{#1}\spEout\right\rbrace_{\mathrm{out}}}
\newcommand{\LiEout}[1]{\myconc{\spLiEout{#1}}}

\newcommand{\spLiIn}[1]{{\spLi{#1}}_{\mathrm{in}}}
\newcommand{\LiIn}[1]{\myconc{\spLiIn{#1}}}

\newcommand{\spLiOut}[1]{{\spLi{#1}}_{\mathrm{out}}}
\newcommand{\LiOut}[1]{\myconc{\spLiOut{#1}}}

\newcommand{\spEHin}{\mychem{EH}}
\newcommand{\EHin}{\myconc{\spEHin}}
\newcommand{\spproton}{\mychem{H}}
\newcommand{\proton}{\myconc{\spproton}}

\newcommand{\mytrn}[1]{{#1}^{\!\mathsf{T}}}
\newcommand{\mymat}[1]{{\bm{#1}}}
\newcommand{\mydet}[1]{{\left|{#1}\right|}}

\newcommand{\ratioLi}{ {\left(\dfrac{\Li{7}}{\Li{6}}\right)} }
\newcommand{\deltaLi}{ {\delta\!\!\!\spLi{7}} }
\newcommand{\deltaLiOut}{{\deltaLi}_{\mathrm{out}}}
\newcommand{\ih}{\ensuremath{\mathbf{H}}}
\newcommand{\ig}{\ensuremath{\mathbf{G}}}

\newcommand{\LiAll}{\Lambda}
\newcommand{\LiAllOut}{{\LiAll}_{\mathrm{out}}}

\begin{document}
\tableofcontents

\section{Isotopic Separation}
$$
	\deltaLi = \left(
		\dfrac{\left(\dfrac{\Li{7}}{\Li{6}}\right)_{sample}}
		{\left(\dfrac{\Li{7}}{\Li{6}}\right)_{standard}}
		 -1 
	\right) \times 1000
$$

$$
	\left(\dfrac{\Li{7}}{\Li{6}}\right)_{sample} = \left(\dfrac{\Li{7}}{\Li{6}}\right)_{standard} \left[1+10^{-3}\deltaLi\right] = \beta_s \left[1+10^{-3}\deltaLi\right]
$$

\begin{equation}
\left\lbrace
\begin{array}{rcl}
	\LiAll    & = & \Li{6} + \Li{7}\\
	\LiAllOut & = & \LiOut{6} + \LiOut{7}\\
\end{array}
\right.
\end{equation}
and
\begin{equation}
\left\lbrace
\begin{array}{rclcl}
	\LiOut{6} & = & \dfrac{1}{1+\beta_s \left[1+10^{-3}\deltaLiOut\right] } \LiAllOut & = & \epsilon_6 \LiAllOut  = \epsilon \LiAllOut \\
	\\
	\LiOut{7} & = & \dfrac{\beta_s \left[1+10^{-3}\deltaLi\right]}{1+\beta_s \left[1+10^{-3}\deltaLiOut\right] } \LiAllOut & = & \epsilon_7 \LiAll,\;\epsilon_7 = 1-\epsilon \\
\end{array}
\right.
\end{equation}
with, for the experiments,
\begin{equation}
	\beta_s = 12.0192
\end{equation}
\begin{equation}
	\epsilon^\mathrm{out} \simeq 0.076, 1-\epsilon^\mathrm{out} \simeq 0.924
\end{equation}

\section{Proposed Mechanism}

\begin{equation}
	 \spLiOut{x} +  \spEout  
	 \xrightleftharpoons[k_x^d]{k_x^a} 
	 \spLiE{x}
	  \xrightleftharpoons[k_x^q]{\mychem{+H},\;k_x^p} \underbrace{\spEHin}_{\xrightarrow[]{k_h} \mychem{E} + \mychem{H}_{\mathrm{out}}} + \underbrace{\spLiIn{x}}_{\xrightleftharpoons[]{k_x} \spLiOut{x}}
\end{equation}

\section{Scheme}

\subsection{Hypothesis}
\begin{itemize}
\item $\proton$ is a  user's function $h(t)$.
\item $\LiOut{6}$ and  $\LiOut{7}$ are parameters.
\item $k_h$ is constant during the whole experiments for NHE is in its saturated mode for the full range of pH
\end{itemize}

\subsection{Kinetics}
We have the phase space described by
\begin{equation}
 \vec{X} = 
        \begin{pmatrix}
        \Eout\\
        \EHin\\
        \LiE{6}\\
        \LiIn{6}\\
        \LiE{7}\\
        \LiIn{7}\\
        \end{pmatrix}
\end{equation}

At any time, we  have
\begin{equation} 
	\label{eq:E0}
	E_0 = \Eout + \EHin +  \LiE{6} + \LiE{7}
\end{equation}

\subsection{Secondary Hypothesis}
We consider that we have the two equations
\begin{equation}
%\left\lbrace
	\begin{array}{rcll}
	 \spLiOut{x} +  \spEout &  \xrightleftharpoons[]{} & \spLiE{x}, & J_x = \dfrac{\LiE{x}}{\LiOut{x} \Eout} = \dfrac{k_x^a}{k_x^d}\\
	 \end{array}
\end{equation}
leading to a constraint vector $\vec{\Gamma}$
with 
\begin{equation}
	\tilde{J}_x = J_x \LiOut{x}
\end{equation}

\begin{equation}
\vec{\Gamma} = 
\begin{pmatrix}
	\tilde{J}_6 \Eout - \LiE{6} \\
	\tilde{J}_7 \Eout - \LiE{7} \\
\end{pmatrix}
\end{equation}
which already simplifies the matter conservation \eqref{eq:E0} into
\begin{equation}
	E_0 = \EHin + \Eout \left(1+\tilde{J}_6+\tilde{J}_7\right).
\end{equation}
And we have the topology for the two equations as
\begin{equation}
	\label{eq:Nu}
	\mymat{\nu}=\begin{pmatrix}-1 & 0 & 1 & 0 & 0 & 0\cr -1 & 0 & 0 & 0 & 1 & 0\end{pmatrix}
\end{equation}
and
\begin{equation}
	\partial_{\vec{X}}\vec{\Gamma} = 
	\begin{pmatrix}\tilde{J}_6 & 0 & -1 & 0 & 0 & 0\cr\tilde{J}_7 & 0 & 0 & 0 & -1 & 0\end{pmatrix}
\end{equation}

\subsection{Rates}

The "slow" rate vector is
\begin{equation}
	\partial_t\vec{X}_{slow} = 
	\begin{pmatrix}
		v_h\\
		p_6-q_6+p_7-q_7-v_h\\
		q_6-p_6\\
		p_6-l_6-q_6\\
		q_7-p_7\\
		p_7-l_7-q_7\\
	\end{pmatrix}
	,\;\;\text{ with }
	\left\lbrace
	\begin{array}{rcll}
	v_h & = & k_h \EHin & \text{(recycling)}\\
	p_x & = & k_x^p \proton \LiE{x} & \text{(forward transfer)}\\
	q_x & = & k_x^q \EHin \LiIn{x} & \text{(reverse transfer)} \\
	l_x & = & k_x  \left(\LiIn{x}-\tilde{\Theta}_x\right) & \text{(leak)}\\
	\end{array}
	\right.
\end{equation}
with (Goldman-Hodgkins-Katz)
\begin{equation}
	\tilde{\Theta}_x = \Theta \LiOut{x}
\end{equation}

\subsection{Semi-Stationary Equations}
We define
\begin{equation}
	\mymat{W} = \mymat{\Phi}\mytrn{\mymat{\nu}} = \begin{pmatrix} -\tilde{J}_6-1 & -\tilde{J}_6 \cr -\tilde{J}_7 & -\tilde{J}_7-1\end{pmatrix}
	,\;\mymat{W}^\ast = \begin{pmatrix} -\tilde{J}_6-1 & \tilde{J}_7 \cr \tilde{J}_6 & -\tilde{J}_7-1\end{pmatrix}
	,\;\; \tilde{D} =\det(\mymat{W})=1+\tilde{J}_6+\tilde{J}_7.
\end{equation}
and
\begin{equation}
	\mymat{\chi} = \tilde{D}\mathds{1}_6-\mytrn{\mymat{\nu}}\mymat{W}^\ast\mymat{\Phi}
\end{equation}

\begin{equation}
	\partial_t\vec{X} = \dfrac{1}{\tilde{D}}
	\mymat{\chi} \partial_t\vec{X}_{slow}
\end{equation}
and we find
\begin{equation}
	\vec{Y} = \begin{pmatrix} \EHin \cr \LiIn{6} \cr \LiIn{7} \end{pmatrix}
	,\;\partial_t \vec{Y} = 
	\begin{pmatrix}
	p_6-q_6+p_7-q_7-v_h\\
	p_6-q_6-l_6\\
	p_7-q_7-l_7
	\end{pmatrix}
\end{equation}
with the expressions
\begin{equation}
\left\lbrace
	\begin{array}{rcl}
	v_h & = & k_h \EHin \\
	q_x & = & k_x^q \EHin \Li{x}  \\
	l_x & = & k_x  \left(\Li{x}- \tilde{\Theta}_x\right)\\
	p_x & = & k_x^p \proton \LiE{x}\\
	\end{array}
\right.
\end{equation}
with
\begin{equation}
	\LiE{x} = \tilde{J}_x \Eout,\;\;\Eout=\dfrac{E_0-\EHin}{\tilde{D}}
\end{equation}
so that
\begin{equation}
	p_x = k_x^p \proton  \tilde{J}_x \dfrac{E_0-\EHin}{\tilde{D}}
\end{equation}

then we get the \underline{three} coupled equations  

\begin{equation}
%\boxed{
\left\lbrace
	\begin{array}{rcl}
		\partial_t\EHin & = & -k_h \EHin + \left(E_0- \EHin\right) \dfrac{\proton}{\tilde{D}} \left(\sum_x k_x^p \tilde{J}_x \right)  
		- \EHin \left\lbrack {\sum_x k_x^q \Li{x}} \right\rbrack\\
		\\
		& = & 
		-k_h E_0+ \left(E_0- \EHin\right)\left\lbrack k_h+ \dfrac{\proton}{\tilde{D}} \left(\sum_x k_x^p \tilde{J}_x \right)\right] 
		- \EHin \left\lbrack {\sum_x k_x^q \Li{x}} \right\rbrack\\
		\\
		\partial_t\Li{x} & = & k_x \left(\tilde{\Theta}_x -\Li{x} \right)  + \left(E_0-\EHin\right) \dfrac{\proton}{\tilde{D}}   k_x^p \tilde{J}_x  - \EHin k_x^q \Li{x}\\
	\end{array}
\right.
%}
\end{equation}

\section{Solving}

\subsection{First Simplification/Catalyst}
\begin{equation}
\left\lbrace
\begin{array}{rcl}
	\alpha & = & \dfrac{\EHin}{E_0}\\
	\\
	\hat\alpha & = & 1-\alpha\\
\end{array}
\right.
\end{equation}
Leading to
\begin{equation}
\left\lbrace
\begin{array}{rcl}
\partial_t\alpha & = & -k_h \alpha + \left(1- \alpha\right) \dfrac{\proton}{\tilde{D}} \left(\sum_x k_x^p \tilde{J}_x \right)  
		- \alpha \left\lbrack {\sum_x k_x^q \Li{x}} \right\rbrack\\
		\\
		& = & 
		-k_h + \left(1- \alpha\right)\left\lbrack k_h+ \dfrac{\proton}{\tilde{D}} \left(\sum_x k_x^p \tilde{J}_x \right)\right] 
		- \alpha \left\lbrack {\sum_x k_x^q \Li{x}} \right\rbrack\\\end{array}
\right.
\end{equation}
or
\begin{equation}
	\partial_t \hat\alpha = k_h - \hat\alpha \left\lbrack k_h+ \dfrac{\proton}{\tilde{D}} \left(\sum_x k_x^p \tilde{J}_x \right)\right] 
		+ (1-\hat\alpha) \left\lbrack {\sum_x k_x^q \Li{x}} \right\rbrack
\end{equation}
	

\subsection{Second Simplification}
\begin{equation}
\left\lbrace
\begin{array}{rcl}
\partial_t\Li{x} & = & k_x \left(\tilde{\Theta}_x -\Li{x} \right)  + \left(E_0-\EHin\right) \dfrac{\proton}{\tilde{D}}   k_x^p \tilde{J}_x  - \EHin k_x^q \Li{x}\\
 &=&   \LiOut{x} \left( k_x
 	\left[\Theta-\dfrac{\Li{x}}{\LiOut{x}}\right] 
	+ \left(1-\alpha\right) E_0 \proton \dfrac{ k_x^p J_x}{\tilde{D}}
 - \alpha  E_0 k_x^q \dfrac{\Li{x}}{\LiOut{x}} \right) \\
\end{array}
\right.
\end{equation}
We define
\begin{equation}
%\left\lbrace
\begin{array}{rcl}
\beta_x & = & \dfrac{\Li{x}}{\LiOut{x}} \\
\end{array}
%\right.
\end{equation}
Leading to
\begin{equation}
\left\lbrace
\begin{array}{rcl}
\partial_t \beta_x + k_x \beta_x 
& = & k_x \Theta  + \left(1-\alpha\right) \dfrac{ k_x^p J_x}{\tilde{D}} E_0 \proton  - \alpha E_0 k_x^q \beta_x \\
\\
& = & k_x \Theta  +  \hat\alpha \dfrac{ k_x^p J_x}{\tilde{D}} E_0 \proton  -  \left(1-\hat\alpha\right) E_0 k_x^q \beta_x \\
\end{array}
\right.
\end{equation}

and obviously
\begin{equation}
	\deltaLi = 1000 \left ( \left[1+10^{-3}\deltaLiOut\right] \dfrac{\beta_7}{\beta_6}-1\right)
\end{equation}
or
\begin{equation}
	\dfrac{ \beta_7}{\beta_6} = \dfrac{\left[1+10^{-3}\deltaLi\right]}{\left[1+10^{-3}\deltaLiOut\right]}
\end{equation}


\subsection{Unified system}
Using 
\begin{equation}
	E_0 = \eta \LiAllOut
\end{equation}
we get
\begin{equation}
\left\lbrace
\begin{array}{rcl}
\partial_t \beta_x  & = &  k_x \left(\Theta -\beta_x \right) +  \eta \left[ \hat\alpha \dfrac{ k_x^p J_x}{\tilde{D}} \LiAllOut \proton  -  \left(1-\hat\alpha\right) \LiAllOut k_x^q \beta_x \right] \\
\\
	\partial_t \hat\alpha & = & k_h - \hat\alpha \left\lbrack k_h+ \dfrac{\proton}{\tilde{D}} \left(\sum_x \epsilon_x k_x^p J_x \right) \LiAllOut \right] 
		+ (1-\hat\alpha) \left\lbrack {\sum_x k_x^q \epsilon_x \beta_x }  \right\rbrack \LiAllOut \\
\end{array}
\right.
\end{equation}


we define 
\begin{equation}
\left\lbrace
\begin{array}{rcl}
	\Upsilon_x & = & \dfrac{k_x^pJ_x \LiAllOut}{\tilde{D}}
	 = \dfrac{k_x^pJ_x \LiAllOut}{1+ J_\epsilon\LiAllOut}, \;\; J_\epsilon= \epsilon J_6 + (1-\epsilon) J_7\\
	 \\
	 Q_x & = & k_x^q \LiAllOut \\
\end{array}
\right.
\end{equation}
so that
\begin{equation}
\left\lbrace
\begin{array}{rcl}
\partial_t \beta_x  & = &  k_x \left(\Theta -\beta_x \right) +  \eta \left[ \hat\alpha \proton \Upsilon_x  -  \left(1-\hat\alpha\right) Q_x\beta_x \right] \\
\\
	\partial_t \hat\alpha & = & k_h - 
		\hat\alpha \left\lbrack k_h+ \proton \left(\sum_x \Upsilon_x \right)\right] 
		+ (1-\hat\alpha) \left\lbrack {\sum_x  \epsilon_x Q_x \beta_x }  \right\rbrack \\
\end{array}
\right.
\end{equation}

\section{Steady State}
We get the system of equations
\begin{equation}
\left\lbrace
\begin{array}{rcl}
	\beta_x^\infty & = & \dfrac{k_x\Theta + \eta \hat\alpha_\infty h_\infty \Upsilon_x}{k_x + \eta (1-\hat\alpha_\infty) Q_x}\\
	\\
	\hat\alpha_\infty & = & \dfrac{k_h+\sum_x\epsilon_x Q_x\beta_x^\infty}{k_h + \left(\sum_x h_\infty \Upsilon_x \right) + \sum_x\epsilon_x Q_x\beta_x^\infty}\\
\end{array}
\right.
\end{equation}


\end{document}

We define 
\begin{equation}
\left\lbrace
\begin{array}{ccll}
	k_x^p J_x & = & k_x \bar{p}_x^2, & \bar{p}_x \text{ in M}^{-1} \\
	k_x^q     & = & k_x \bar{q}_x  , & \bar{q}_x \text{ in M}^{-1} \\
	\tilde{D} & = & 1+(\sum_x \epsilon_x J_x) \LiAllOut, & J_\epsilon = \epsilon J_6 + (1-\epsilon) J_7 \\
\end{array}
\right.
\end{equation}

\begin{equation}
	\partial_t \beta_x + k_x \beta_x = k_x \left[ \Theta +
	 \eta \left(
	 \hat\alpha\dfrac{\bar{p}_x^2}{\tilde{D}} \LiAllOut \proton
	 -(1-\hat\alpha) \bar{q}_x \LiAllOut \beta_x 
	  \right) \right]
\end{equation}
and we set
\begin{equation}
	\Upsilon_x = \bar{p}_x^2 \dfrac{\LiAllOut}{1+J_\epsilon \LiAllOut} \;\;(\text{in M}^{-1}) , \;\; Q_x = \bar{q}_x \LiAllOut\;\; (\text{a scalar})
\end{equation}
to get
\begin{equation}
\left\lbrace
\begin{array}{rcl}
	\partial_t \beta_x + k_x \beta_x & = & k_x \left[ \Theta + \eta \left( \hat\alpha \Upsilon_x \proton - (1-\hat\alpha)Q_x \beta_x \right) \right] \\
	\\
	\partial_t \hat\alpha & = & k_h - \hat\alpha \left\lbrack k_h+
	 \proton \left(\sum_x \epsilon_x k_x \Upsilon_x \right) \right] 
		+ (1-\hat\alpha) \left\lbrack {\sum_x \epsilon_x k_x Q_x \beta_x }  \right\rbrack \\
\end{array}
\right.
\end{equation}
It is now time to change time using the $\Li{7}$ passive rate as reference
\begin{equation}
 k_6 = \sigma k_7,\;\;\tau=k_7 t, \;\; \partial_t = k_7 \partial_\tau, \;\; \mu = \dfrac{k_h}{k_7}
\end{equation}
to get
\begin{equation}
\left\lbrace
\begin{array}{ccl}
	\partial_\tau \beta_7 + \beta_7 & = & \left[ \Theta + \eta \left( \hat\alpha \Upsilon_7 \proton - (1-\hat\alpha)Q_7 \beta_7 \right) \right]\\
	\\
	\partial_\tau \beta_6 + \sigma \beta_6 & = & \sigma \left[ \Theta + \eta \left( \hat\alpha \Upsilon_6 \proton - (1-\hat\alpha)Q_6 \beta_6 \right) \right]\\
	\\
		\partial_\tau \hat\alpha & = & \mu - \hat\alpha \left\lbrack \mu +
	 \proton \underbrace{\left(\epsilon\sigma\Upsilon_6 + (1-\epsilon)\Upsilon_7 \right)}_{\Upsilon_\alpha} \right] 
		+ (1-\hat\alpha) \left\lbrack { \epsilon \sigma Q_6 \beta_6 + (1-\epsilon) Q_7 \beta_7  }  \right\rbrack \\
\end{array}
\right.
\end{equation}



\section{Steady State}
We have the two equations
\begin{equation}
\begin{array}{rcl}
	\beta_x^\infty & =  & \dfrac{\Theta+\hat\alpha_\infty \eta \Upsilon_x h_\infty }{1+ (1-\hat\alpha_\infty)\eta Q_x }\\
	%\mu \eta (1-\hat\alpha_\infty) & = & \Theta \left[ \sigma\epsilon (\beta_6^\infty-1) + (1-\epsilon)(\beta_7^\infty-1)\right]\\\
\end{array}
\end{equation}
to be injected in the last one to get $\hat\alpha_\infty$.


\section{First Order}
\subsection{First Order $\hat\alpha$}
\subsubsection{Generic $\proton$}
We will integrate
\begin{equation}
	\partial_\tau \hat\alpha = \mu - \hat\alpha\left[\mu+\proton \Upsilon_\alpha\right] = \mu-\hat{\alpha}(\tau) g(\tau)
\end{equation}
We define
\begin{equation}
	\ig(\tau) = \int_0^\tau g(u)\,\mathrm{d}u
\end{equation}
and the solution is
\begin{equation}
	\hat\alpha_1(\tau) = \left[1+\mu\int_0^\tau e^{\ig(u)}\,\mathrm{d} u \right] e^{-\ig(\tau)}
\end{equation}

\subsubsection{Constant $\proton$}
\begin{equation}
\left\lbrace
\begin{array}{rcl}
	g(\tau) & = & \Omega=\mu+\Upsilon_\alpha h_0\\
	\ig(\tau) & = & \Omega \tau\\
	\hat{\alpha}_0 (\tau) & = & \dfrac{1}{\Omega} \left[ \mu + \left(\Omega-\mu\right) e^{-\Omega \tau} \right]\\
	      &=&\hat\alpha_0^\infty+(1-\hat\alpha_0^\infty) e^{-\Omega \tau}
\end{array}
\right.
\end{equation}


\subsection{First order $\beta_x$}


\end{document}


