\documentclass[aps,onecolumn,10pt]{revtex4}
\usepackage{graphicx}
\usepackage{amssymb,amsfonts,amsmath,amsthm}
\usepackage{chemarr}
\usepackage{bm}
\usepackage{pslatex}
\usepackage{xfrac}
\usepackage[dvipsnames]{xcolor}
\usepackage{bookman}
\usepackage{dsfont}
\usepackage{mathptmx}
%\usepackage{hyperref}

\newcommand{\mychem}[1]{\mathtt{#1}}
\newcommand{\myconc}[1]{\left\lbrack{#1}\right\rbrack}

\newcommand{\spLi}[1]{{~^{\mychem{#1}}\mychem{Li}}}
\newcommand{\Li}[1]{\myconc{\spLi{#1}}}

\newcommand{\spEout}{\mychem{E}}
\newcommand{\Eout}{\myconc{\spEout}}

\newcommand{\spLiEin}[1]{\left\lbrace\spLi{#1}\spEout\right\rbrace_{\mathrm{in}}}
\newcommand{\LiEin}[1]{\myconc{\spLiEin{#1}}}

\newcommand{\spLiE}[1]{\left\lbrace\spLi{#1}\spEout\right\rbrace}
\newcommand{\LiE}[1]{\myconc{\spLiE{#1}}}


\newcommand{\spLiEout}[1]{\left\lbrace\spLi{#1}\spEout\right\rbrace_{\mathrm{out}}}
\newcommand{\LiEout}[1]{\myconc{\spLiEout{#1}}}

\newcommand{\spLiIn}[1]{{\spLi{#1}}_{\mathrm{in}}}
\newcommand{\LiIn}[1]{\myconc{\spLiIn{#1}}}

\newcommand{\spLiOut}[1]{{\spLi{#1}}_{\mathrm{out}}}
\newcommand{\LiOut}[1]{\myconc{\spLiOut{#1}}}

\newcommand{\spEHin}{\mychem{EH}}
\newcommand{\EHin}{\myconc{\spEHin}}
\newcommand{\spproton}{\mychem{H}}
\newcommand{\proton}{\myconc{\spproton}}

\newcommand{\mytrn}[1]{{#1}^{\!\mathsf{T}}}
\newcommand{\mymat}[1]{{\bm{#1}}}
\newcommand{\mydet}[1]{{\left|{#1}\right|}}

\newcommand{\ratioLi}{ {\left(\dfrac{\Li{7}}{\Li{6}}\right)} }
\newcommand{\deltaLi}{ {\delta\!\!\!\spLi{7}} }
\newcommand{\deltaLiOut}{{\deltaLi}_{\mathrm{out}}}
\newcommand{\ih}{\ensuremath{\mathbf{H}}}
\newcommand{\ig}{\ensuremath{\mathbf{G}}}

\newcommand{\LiAll}{\Lambda}
\newcommand{\LiAllOut}{{\LiAll}_{\mathrm{out}}}

\begin{document}
%\tableofcontents

\section{Isotopic Separation}
$$
	\deltaLi = \left(
		\dfrac{\left(\dfrac{\Li{7}}{\Li{6}}\right)_{sample}}
		{\left(\dfrac{\Li{7}}{\Li{6}}\right)_{standard}}
		 -1 
	\right) \times 1000
$$

$$
	\left(\dfrac{\Li{7}}{\Li{6}}\right)_{sample} = \left(\dfrac{\Li{7}}{\Li{6}}\right)_{standard} \left[1+10^{-3}\deltaLi\right] = \beta_s \left[1+10^{-3}\deltaLi\right]
$$

\begin{equation}
\left\lbrace
\begin{array}{rcl}
	\LiAll    & = & \Li{6} + \Li{7}\\
	\LiAllOut & = & \LiOut{6} + \LiOut{7}\\
\end{array}
\right.
\end{equation}
and
\begin{equation}
\left\lbrace
\begin{array}{rclcl}
	\LiOut{6} & = & \dfrac{1}{1+\beta_s \left[1+10^{-3}\deltaLiOut\right] } \LiAllOut & = & \epsilon_6 \LiAllOut  = \epsilon \LiAllOut \\
	\\
	\LiOut{7} & = & \dfrac{\beta_s \left[1+10^{-3}\deltaLi\right]}{1+\beta_s \left[1+10^{-3}\deltaLiOut\right] } \LiAllOut & = & \epsilon_7 \LiAllOut,\;\epsilon_7 = 1-\epsilon \\
\end{array}
\right.
\end{equation}
with, for the experiments,
\begin{equation}
	\beta_s = 12.0192
\end{equation}
\begin{equation}
	\epsilon^\mathrm{out} \simeq 0.076, 1-\epsilon^\mathrm{out} \simeq 0.924
\end{equation}

\section{Proposed Mechanism}

\begin{equation}
	 \spLiOut{x} +  \spEout  
	 \xrightleftharpoons[k_x^d]{k_x^a} 
	 \spLiE{x}
	  \xrightleftharpoons[k_x^q]{\mychem{+H},\;k_x^p} 
	  \underbrace{\spEHin}_{\xrightarrow[]{k_h} \mychem{E} + \mychem{H}_{\mathrm{out}}} + \underbrace{\spLiIn{x}}_{\xrightleftharpoons[]{k_x} \spLiOut{x}}
\end{equation}

\section{Scheme}

\subsection{Hypothesis}
\begin{itemize}
\item $\proton$ is a  user's function $h(t)$.
\item $\LiOut{6}$ and  $\LiOut{7}$ are parameters.
\item $k_h$ is constant during the whole experiments for NHE is in its saturated mode for the full range of pH
\end{itemize}

\subsection{Kinetics}
We have the phase space described by
\begin{equation}
 \vec{X} = 
        \begin{pmatrix}
        \Eout\\
        \EHin\\
        \LiE{6}\\
        \LiIn{6}\\
        \LiE{7}\\
        \LiIn{7}\\
        \end{pmatrix}
\end{equation}

At any time, we  have
\begin{equation} 
	\label{eq:E0}
	E_0 = \Eout + \EHin +  \LiE{6} + \LiE{7}
\end{equation}

\subsection{Secondary Hypothesis}
We consider that we have the two equations
\begin{equation}
%\left\lbrace
	\begin{array}{rcll}
	 \spLiOut{x} +  \spEout &  \xrightleftharpoons[]{} & \spLiE{x}, & J_x = \dfrac{\LiE{x}}{\LiOut{x} \Eout} = \dfrac{k_x^a}{k_x^d}\\
	 \end{array}
\end{equation}
leading to a constraint vector $\vec{\Gamma}$
with 
\begin{equation}
	\tilde{J}_x = J_x \LiOut{x}
\end{equation}

\begin{equation}
\vec{\Gamma} = 
\begin{pmatrix}
	\tilde{J}_6 \Eout - \LiE{6} \\
	\tilde{J}_7 \Eout - \LiE{7} \\
\end{pmatrix}
\end{equation}
which already simplifies the matter conservation \eqref{eq:E0} into
\begin{equation}
	E_0 = \EHin + \Eout \left(1+\tilde{J}_6+\tilde{J}_7\right).
\end{equation}
And we have the topology for the two equations as
\begin{equation}
	\label{eq:Nu}
	\mymat{\nu}=\begin{pmatrix}-1 & 0 & 1 & 0 & 0 & 0\cr -1 & 0 & 0 & 0 & 1 & 0\end{pmatrix}
\end{equation}
and
\begin{equation}
	\partial_{\vec{X}}\vec{\Gamma} = 
	\begin{pmatrix}\tilde{J}_6 & 0 & -1 & 0 & 0 & 0\cr\tilde{J}_7 & 0 & 0 & 0 & -1 & 0\end{pmatrix}
\end{equation}

\subsection{Rates}

The "slow" rate vector is
\begin{equation}
	\partial_t\vec{X}_{slow} = 
	\begin{pmatrix}
		v_h\\
		p_6-q_6+p_7-q_7-v_h\\
		q_6-p_6\\
		p_6-q_6-l_6\\
		q_7-p_7\\
		p_7-q_7-l_7\\
	\end{pmatrix}
	,\;\;\text{ with }
	\left\lbrace
	\begin{array}{rcll}
	v_h & = & k_h \EHin & \text{(recycling)}\\
	p_x & = & k_x^p \proton \LiE{x} & \text{(forward transfer)}\\
	q_x & = & k_x^q \EHin \LiIn{x} & \text{(reverse transfer)} \\
	l_x & = & k_x  \left(\LiIn{x}-\tilde{\Theta}_x\right) & \text{(leak)}\\
	\end{array}
	\right.
\end{equation}
with (Goldman-Hodgkins-Katz)
\begin{equation}
	\tilde{\Theta}_x = \Theta \LiOut{x}
\end{equation}

\subsection{Semi-Stationary Equations}
We define
\begin{equation}
	\mymat{W} = \mymat{\Phi}\mytrn{\mymat{\nu}} = \begin{pmatrix} -\tilde{J}_6-1 & -\tilde{J}_6 \cr -\tilde{J}_7 & -\tilde{J}_7-1\end{pmatrix}
	,\;\mymat{W}^\ast = \begin{pmatrix} -\tilde{J}_6-1 & \tilde{J}_7 \cr \tilde{J}_6 & -\tilde{J}_7-1\end{pmatrix}
	,\;\; \tilde{D} =\det(\mymat{W})=1+\tilde{J}_6+\tilde{J}_7.
\end{equation}
and
\begin{equation}
	\mymat{\chi} = \tilde{D}\mathds{1}_6-\mytrn{\mymat{\nu}}\mymat{W}^\ast\mymat{\Phi}
\end{equation}

\begin{equation}
	\partial_t\vec{X} = \dfrac{1}{\tilde{D}}
	\mymat{\chi} \partial_t\vec{X}_{slow}
\end{equation}
and we find
\begin{equation}
	\vec{Y} = \begin{pmatrix} \EHin \cr \LiIn{6} \cr \LiIn{7} \end{pmatrix}
	,\;\partial_t \vec{Y} = 
	\begin{pmatrix}
	p_6-q_6+p_7-q_7-v_h\\
	p_6-q_6-l_6\\
	p_7-q_7-l_7
	\end{pmatrix}
\end{equation}
with the expressions
\begin{equation}
\left\lbrace
	\begin{array}{rcl}
	v_h & = & k_h \EHin \\
	q_x & = & k_x^q \EHin \Li{x}  \\
	l_x & = & k_x  \left(\Li{x}- \tilde{\Theta}_x\right)\\
	p_x & = & k_x^p \proton \LiE{x}\\
	\end{array}
\right.
\end{equation}
with
\begin{equation}
	\LiE{x} = \tilde{J}_x \Eout,\;\;\Eout=\dfrac{E_0-\EHin}{\tilde{D}}
\end{equation}
so that
\begin{equation}
	p_x = k_x^p \proton  \tilde{J}_x \dfrac{E_0-\EHin}{\tilde{D}}
\end{equation}

then we get the \underline{three} coupled equations  

\begin{equation}
%\boxed{
\left\lbrace
	\begin{array}{rcl}
		\partial_t\EHin & = & -k_h \EHin + \left(E_0- \EHin\right) \dfrac{\proton}{\tilde{D}} \left(\sum_x k_x^p \tilde{J}_x \right)  
		- \EHin \left\lbrack {\sum_x k_x^q \Li{x}} \right\rbrack
		\\
		\\
		& = & 
		-k_h E_0+ \left(E_0- \EHin\right)\left\lbrack k_h+ \dfrac{\proton}{\tilde{D}} \left(\sum_x k_x^p \tilde{J}_x \right)\right] 
		- \EHin \left\lbrack {\sum_x k_x^q \Li{x}} \right\rbrack
		\\
		\\
		\partial_t\Li{x} & = & k_x \left(\tilde{\Theta}_x -\Li{x} \right)  + \left(E_0-\EHin\right) \dfrac{\proton}{\tilde{D}}   k_x^p \tilde{J}_x  
		- \EHin k_x^q \Li{x}
		\\
	\end{array}
\right.
%}
\end{equation}

\section{Solving}

\subsection{First Simplification/Catalyst}
\begin{equation}
\left\lbrace
\begin{array}{rcl}
	\alpha & = & \dfrac{\EHin}{E_0}\\
	\\
	\hat\alpha & = & 1-\alpha\\
\end{array}
\right.
\end{equation}
Leading to
\begin{equation}
\label{eq:alpha}
\left\lbrace
\begin{array}{rcl}
\partial_t\alpha & = & -k_h \alpha + \left(1- \alpha\right) \dfrac{\proton}{\tilde{D}} \left(\sum_x k_x^p \tilde{J}_x \right)  
		- \alpha \left\lbrack {\sum_x k_x^q \Li{x}} \right\rbrack
		\\
		\\
		& = & 
		-k_h + \left(1- \alpha\right)\left\lbrack k_h+ \dfrac{\proton}{\tilde{D}} \left(\sum_x k_x^p \tilde{J}_x \right)\right] 
		- \alpha \left\lbrack {\sum_x k_x^q \Li{x}} \right\rbrack
		\\
		\\
	\partial_t \hat\alpha & = & k_h - \hat\alpha \left\lbrack k_h+ \dfrac{\proton}{\tilde{D}} \left(\sum_x k_x^p \tilde{J}_x \right)\right] 
		+ (1-\hat\alpha) \left\lbrack {\sum_x k_x^q \Li{x}} \right\rbrack
		\end{array}
\right.
\end{equation}
where $\hat\alpha$ is the \underline{fraction of not-protonated NHE}.

\subsection{Second Simplification}
\begin{equation}
\left\lbrace
\begin{array}{rcl}
\partial_t\Li{x} & = & k_x \left(\tilde{\Theta}_x -\Li{x} \right)  + \left(E_0-\EHin\right) \dfrac{\proton}{\tilde{D}}   k_x^p \tilde{J}_x  - \EHin k_x^q \Li{x}
\\
 &=&   \LiOut{x} \left( k_x
 	\left[\Theta-\dfrac{\Li{x}}{\LiOut{x}}\right] 
	+ \left(1-\alpha\right) E_0 \proton \dfrac{ k_x^p J_x}{\tilde{D}}
 - \alpha  E_0 k_x^q \dfrac{\Li{x}}{\LiOut{x}}
  \right) \\
\end{array}
\right.
\end{equation}
We define
\begin{equation}
%\left\lbrace
\begin{array}{rcl}
\beta_x & = & \dfrac{\Li{x}}{\LiOut{x}} \\
\end{array}
%\right.
\end{equation}
Leading to
\begin{equation}
\left\lbrace
\begin{array}{rcl}
\partial_t \beta_x 
& = & k_x (\Theta-\beta_x)  + \left(1-\alpha\right) \dfrac{k_x^p J_x}{\tilde{D}} E_0 \proton  - \alpha E_0 k_x^q \beta_x
 \\
\\
& = & k_x (\Theta-\beta_x)  +  \hat\alpha \dfrac{ k_x^p J_x}{\tilde{D}} E_0 \proton  -  \left(1-\hat\alpha\right) E_0 k_x^q \beta_x
 \\
\end{array}
\right.
\end{equation}

and obviously
\begin{equation}
	\deltaLi = 1000 \left ( \left[1+10^{-3}\deltaLiOut\right] \dfrac{\beta_7}{\beta_6}-1\right)
\end{equation}
or
\begin{equation}
	\dfrac{ \beta_7}{\beta_6} = \dfrac{\left[1+10^{-3}\deltaLi\right]}{\left[1+10^{-3}\deltaLiOut\right]}
\end{equation}

We inject $\beta_x$ into the \eqref{eq:alpha} equation to get
\begin{equation}
	\partial_t \hat\alpha  =  k_h - \hat\alpha \left\lbrack k_h+ \dfrac{\proton}{\tilde{D}} \left(\sum_x k_x^p \tilde{J}_x \right)\right] 
		+ (1-\hat\alpha) \left\lbrack {\sum_x k_x^q \epsilon_x \beta_x } \right\rbrack \LiAllOut
\end{equation}


\subsection{Unified system for variables}
Using 
\begin{equation}
\left\lbrace
\begin{array}{rcl}
	E_0 &= & \eta \LiAllOut\\
	J_\epsilon & = & \epsilon_6 J_6 + \epsilon_7 J_7\\
\end{array}
\right.
\end{equation}
we get
\begin{equation}
\left\lbrace
\begin{array}{rcl}
\partial_t \beta_x  & = &  k_x \left(\Theta -\beta_x \right) + \eta \left[ \hat\alpha \dfrac{ k_x^p J_x}{1+J_\epsilon \LiAllOut} \LiAllOut \proton -  \left(1-\hat\alpha\right) \LiAllOut k_x^q \beta_x \right] \\
\\
	\partial_t \hat\alpha & = & k_h - \hat\alpha \left\lbrack k_h+ \dfrac{\proton}{1+J_\epsilon \LiAllOut} \left(\sum_x \epsilon_x k_x^p J_x \right) \LiAllOut \right] 
		+ (1-\hat\alpha) \left\lbrack {\sum_x k_x^q \epsilon_x \beta_x }  \right\rbrack \LiAllOut 
		\\
\end{array}
\right.
\end{equation}


\subsection{Rescaling time}
Since we are concerned with the catalytic aspect, we use
\begin{equation}
	\tau = k_h t, \;\;\partial_t = k_h \partial_\tau
\end{equation}

We define 
\begin{equation}
\left\lbrace
\begin{array}{rcll}
	\mu_x      & = & \dfrac{k_x}{k_h} & (\text{a scalar})\\
	\\
	\Upsilon_x & = & \dfrac{k_x^pJ_x \LiAllOut}{k_h\left(1+ J_\epsilon\LiAllOut\right)} & (\text{in M}^{-1})
	 \\
	 \\
	Q_x & = & \dfrac{k_x^q \LiAllOut}{k_h} & (\text{a scalar})\\
\end{array}
\right.
\end{equation}
so that
\begin{equation}
\left\lbrace
\begin{array}{rcl}
\partial_\tau \beta_x  & = &  \mu_x \left(\Theta -\beta_x \right) + \eta \left[ \hat\alpha \proton \Upsilon_x  -  \left(1-\hat\alpha\right) Q_x\beta_x \right] \\
\\
	\partial_\tau \hat\alpha & = & 1 - 
		\hat\alpha \left\lbrack 1+ \proton \left(\sum_x \epsilon_x \Upsilon_x \right)\right] + (1-\hat\alpha) \left\lbrack {\sum_x  \epsilon_x Q_x \beta_x }  \right\rbrack \\
\end{array}
\right.
\end{equation}

\section{Specific Values}
\subsection{Short times}
Since $\hat\alpha\to1$,
\begin{equation}
\left\lbrace
\begin{array}{rcl}
\beta_{x,0} & \simeq & \left(\mu_x \Theta + \eta h_0 \Upsilon_x)\right)\tau\\
\hat\alpha_{0} & \simeq & 1 - \left[1+h_0 \left(\sum_x \epsilon_x \Upsilon_x\right)\right] \tau\\
\end{array}
\right.
\end{equation}
so that
\begin{equation}
	\label{eq:r0}
	r_0 = \dfrac{\beta_{7,0}}{\beta_{6,0}} = \dfrac{1+10^{-3}\deltaLi_0}{1+10^{-3}\deltaLiOut}
	 = \dfrac{\mu_7\Theta+\eta h_0 \Upsilon_7}{\mu_6\Theta+\eta h_0 \Upsilon_6}
\end{equation}
only depends on the linear part...

\subsection{Steady State}
We get the system of equations
\begin{equation}
\label{eq:steady}
\left\lbrace
\begin{array}{rcl}
	\beta_x^\infty & = & \dfrac{\mu_x\Theta + \eta \hat\alpha_\infty h_\infty \Upsilon_x}{\mu_x + \eta (1-\hat\alpha_\infty) Q_x}\\\
	\\
	\hat\alpha_\infty & = & \dfrac{1+\sum_x\epsilon_x Q_x\beta_x^\infty
	}{1 + \left(\sum_x \epsilon_x h_\infty \Upsilon_x \right) + \sum_x\epsilon_x Q_x\beta_x^\infty
	}, \;\; \dfrac{1}{1+\left(\sum_x \epsilon_x h_\infty \Upsilon_x \right)}\leq\hat\alpha_\infty \leq 1\\\
	\\
	& = & 1 - \dfrac{1}{\eta}\left( \sum_x \epsilon_x \mu_x \left[\beta_x^\infty - \Theta\right]\right) \;\; \text{by combination } Q_x\\
\end{array}
\right.
\end{equation}

\begin{equation}
\text{parameters} : k_h, \mu_6, \mu_7, \Theta, \epsilon_6, \epsilon_7, Q_6, Q_7, \eta, \Upsilon_6, \Upsilon_7, h_0, h_\infty (13)
\end{equation}

\begin{itemize}
	\item We fix $r_0$ from the initial $\deltaLi_0$
	\item We fix $r_\infty$ from the final $\deltaLi_\infty$
	\item We fix $\epsilon_6$ and $\epsilon_7$ from the initial composition
	\item We set $\Theta$ from the $GHK$ equation
	$$
		\Theta = e^{ -\frac{FV_m}{RT} },\;\;V_m\simeq -40\text{mV}
	$$
	\item We fix $h_0$ and $h_\infty$, which shall depend and the initial condition and $\LiAllOut$.
	\item and we have three steady state relations...
\end{itemize}

\subsection{Choosing Parameters}

\begin{equation}
\left\lbrace
\begin{array}{rcl}
		\mu_6     & = & \sigma \mu_7\\
		\sigma    & \simeq & 1.00229 \\
		\\
		\hline
		\\
		%\Upsilon_6 & = & \kappa \Upsilon_7\\
		\\
		\hline
		\\
		\Theta     & \simeq  & e^{\frac{-F V_m}{RT} } \simeq 4.47\\
		\\
\end{array}
\right.
\end{equation}

\begin{itemize}
\item We choose $\mu_7$ so that $\mu_6=\sigma\mu_7$, $\Theta$ from the electroosmotic gradient
\item Since we must satisfy \eqref{eq:r0},
\begin{equation}
	r_0 = \dfrac{\mu_7\Theta+\eta h_0 \Upsilon_7}{\mu_6\Theta+\eta h_0 \Upsilon_6}
\end{equation}
Without catalytic process, $r_0$ cannot be lower than $\dfrac{1}{\sigma}$,
%the catalytic ratio is
%\begin{equation}
%	\eta = \Theta\dfrac{\mu_7-r_0\mu_6}{r_0 h_0 \Upsilon_6-h_0\Upsilon_7} 
%	= \dfrac{\Theta\mu_7}{h_0\Upsilon_7} \underbrace{\dfrac{1-\sigma r_0}{r_0\kappa-1}}_{f_0}
%\end{equation}

\underline{The catalytic process is observed if}
\begin{equation}
\boxed{
\left\lbrace
\begin{array}{rcl}
	r_0 & = & \dfrac{1+10^{-3}\deltaLi_0}{1+10^{-3}\deltaLiOut} \leq \dfrac{1}{\sigma} = 0.99772\\
	%    & \Leftrightarrow & \Upsilon_6 \geq \sigma \Upsilon_7\\
\end{array}
\right.
}
\end{equation}
%which is quite normal, the intake speedup of $\spLi{6}$ must be itself greater than its leakage speedup!
We define
\begin{equation}
	\Upsilon_6  = \kappa \Upsilon_7
\end{equation}
leading to
\begin{equation}
	r_0 =  \dfrac{\mu_7\Theta+\eta h_0 \Upsilon_7}{\mu_6\Theta+\eta h_0 \Upsilon_6} = \dfrac{1}{\sigma} \dfrac{\mu_7\Theta+\eta h_0 \Upsilon_7}{\mu_7\Theta+\eta \kappa h_0 \Upsilon_7}
\end{equation}
and
\begin{equation}
	\boxed{
	\eta = \dfrac{\mu_7\Theta}{h_0\Upsilon_7} \dfrac{1-r_0\sigma}{\kappa r_0-1} = \dfrac{\mu_7\Theta}{h_0\Upsilon_7} f_0
	,\;\; r_0 < \frac{1}{\sigma}, \;\; \kappa > \frac{1}{r_0} > \sigma
	}
\end{equation}
which are the conditions arising from a catalytic speedup...

\item Since we can't have a higher $\beta_x$ than the linear case,
\begin{equation}
\left\lbrace
\begin{array}{rcl}
\beta_x^\infty & \leq & \Theta + \eta \dfrac{h_\infty\Upsilon_x\cos^2\Omega}{\mu_x}\;\;\left(\text{with }\cos^2\Omega=\hat\alpha_\infty^1=\dfrac{1}{1+\sum_x\epsilon_x h_\infty \Upsilon_x}\right)\\
\\
&\leq& \Theta\left(1+ f_0 \dfrac{h_\infty}{h_0} \dfrac{\Upsilon_x}{\Upsilon_7}\dfrac{\mu_7}{\mu_x}\cos^2 \Omega \right)
\end{array}
\right.
\end{equation}
So that
\begin{equation}
	\beta_x = \Theta + \eta \dfrac{h_\infty\Upsilon_x\cos^2\Omega}{\mu_x} \cos^2 \varphi_x,\;\;Q_x\to0\text{ when }\varphi_x\to0
\end{equation}
and we notice
\begin{equation}
	\tan^2 \Omega = \sum_x h_\infty \epsilon_x \Upsilon_x = h_\infty \Upsilon_7 \left[\epsilon\kappa+(1-\epsilon)\right]
\end{equation}
We can write
\begin{equation}
	\beta_x^\infty = \Theta \left(1 + f_0 \dfrac{h_\infty}{h_0} \dfrac{\Upsilon_x}{\Upsilon_7}\dfrac{\mu_7}{\mu_x}\cos^2 \Omega\cos^2 \varphi_x\right)
	%= \Theta \left(1 + f_0 \dfrac{h_\infty}{h_0} \dfrac{\Upsilon_x}{\Upsilon_7}\dfrac{\mu_7}{\mu_x}
	%\dfrac{\cos^2 B_x}{1+\left[\epsilon\kappa+(1-\kappa)\right] h_\infty \Upsilon_7 }
	%\cos^2\Omega \cos^2\varphi_x
	%\right)
\end{equation}
so that
\begin{equation}
	\beta_7^\infty = \Theta \left(1 + 
	f_0 \dfrac{h_\infty}{h_0} 	\cos^2\Omega \cos^2\varphi_7
	\right)
\end{equation}

\begin{equation}
	\beta_6^\infty = \Theta \left(1 + f_0 \dfrac{h_\infty}{h_0} \dfrac{\kappa}{\sigma}
	\cos^2\Omega \cos^2\varphi_6 
	\right)
\end{equation}


\item The system imposes $\beta_x\geq\Theta$, shown by the two expressions of $\hat\alpha_\infty$, and because the catalytic process imposes a boost with respect to the $\Theta$ level.
If we set
\begin{equation}
	\hat\alpha_\infty = \cos^2\Omega \cos^2\psi + \sin^2\psi, 
	\;\;\cos^2\Omega  = \dfrac{1}{1+\sum_x\epsilon_x h_\infty \Upsilon_x},\;\;1-\hat\alpha_\infty = \sin^2\Omega  \cos^2\psi
\end{equation}
then $\psi$ is the quadratic coupling factor, since
\begin{equation}
	\sum_x \epsilon_x Q_x \beta_x^\infty = \dfrac{\tan^2\psi}{\cos^2\Omega}
\end{equation}

We deduce from each choice of $B_x$
\begin{equation}
\left\lbrace
\begin{array}{rcl}
	\eta (1-\hat\alpha_\infty) \beta_x^\infty Q_x & = & \eta \hat\alpha_\infty h_\infty \Upsilon_x - \mu_x\left(\beta_x^\infty - \Theta \right)\\
	\\
	 (1-\hat\alpha_\infty) \beta_x^\infty Q_x & = & \hat\alpha_\infty h_\infty \Upsilon_x -   {h_\infty\Upsilon_x\cos^2\Omega}\sin^2 B_x\\
	 & = &  h_\infty \Upsilon_x \left( \underbrace{\cos^2\Omega \cos^2\psi + \sin^2\psi}_{\geq \cos^2\Omega} - \underbrace{\cos^2\Omega \cos^2 B_x}_{\leq\cos^2\Omega} \right)\\
\end{array}
\right.
\end{equation}

Then
\begin{equation}
\left\lbrace
\begin{array}{rcl}
 (1-\hat\alpha_\infty) \sum_x \epsilon_x \beta_x^\infty Q_x & = & 
 \left(\cos^2\Omega \cos^2\psi + \sin^2\psi\right) \left(\sum_x \epsilon_x h_\infty \Upsilon_x\right) - \cos^2 \Omega \left(\sum_x \epsilon_x h_\infty \Upsilon_x \cos^2 \varphi_x \right)\\
 \\
\sin^2\Omega  \cos^2\psi \dfrac{\tan^2\psi}{\cos^2\Omega} & = & \left(\cos^2\Omega \cos^2\psi + \sin^2\psi\right) \tan^2 \Omega
 - \cos^2 \Omega \left(\sum_x \epsilon_x h_\infty \Upsilon_x \cos^2 \varphi_x \right)\\
 \\
 \cos^2\psi & = & \dfrac{\sum_x \epsilon_x \Upsilon_x \cos^2 \varphi_x }{\sum_x \epsilon_x \Upsilon_x} 
 = \dfrac{\epsilon\kappa\cos^2B_\varphi+(1-\epsilon)\cos^2\varphi_7}{\epsilon\kappa+(1-\epsilon)}\\
\end{array}
\right.
\end{equation}

\begin{equation}
	\boxed
	{
	\cos^2\psi = \dfrac{\epsilon\kappa\cos^2\varphi_6+(1-\epsilon)\cos^2\varphi_7}{\epsilon\kappa+(1-\epsilon)} 
	 \in  \left[\min\left(\cos^2\varphi_6,\cos^2\varphi_7\right):\max\left(\cos^2\varphi_6,\cos^2\varphi_7\right)\right] \\
	}
\end{equation}

\end{itemize}

\section{Building a simulation}

\subsection{Constants}

\begin{itemize}
	\item We choose $\sigma$ from the diffusion coefficient ratio
	\item We choose $\epsilon$ from the external solution
	\item We choose $\Theta$ from the cellular potential
\end{itemize}


\subsection{Parameters To Choose}
\begin{itemize}
	\item We choose $r_0$ from the isotopic data
	
	\item We choose 
	\begin{equation}
		\rho_\infty =  \dfrac{h_\infty}{h_0}
	\end{equation}
	\textbf{and there is an heuristic coupling with $\LiAllOut$}
	
	\item We choose $\beta_7^\infty>\Theta$ so that
	\begin{equation}
		\dfrac{\beta_7^\infty}{\Theta}-1 = \tan^2 \gamma_7 % = f_0 	\rho_\infty \cos^2\Omega \cos^2\varphi_7
	\end{equation}
	
	\item We choose $\cos^2\Omega$ (\textbf{parameter}), when $\Omega=0$ means no intake 
	%and we remember
%		$$
%			\tan^2\Omega = h_\infty \Upsilon_7 \left[\epsilon\kappa+(1-\epsilon)\right]
%		$$

	\item We choose a linear level $\cos^2 \phi_7$
	
\end{itemize}

\subsection{Parameters to compute}


\subsection{older}

\color{blue}

\begin{itemize}
\item We choose $\sigma$, $\epsilon$ and $\Theta$ from physical values
\item We choose $\mu_7$ (\textbf{parameter})
\item We choose $\rho_\infty=h_\infty/h_0$ (from the experiments, it would depends on $\LiAllOut$, \textbf{heuristic})
\item We choose $r_0<1/\sigma$  (from the experiments)
\item We choose $\kappa>1/r_0>\sigma$ (\textbf{parameter})
\item We choose $\cos^2\Omega$ (\textbf{parameter}), when $\Omega=0$ means no intake
\item We compute 
$$
	h_0\Upsilon_7 = \dfrac{1}{\rho_\infty}\left(\dfrac{1}{\cos^2\Omega}-1\right)\dfrac{1}{\epsilon\kappa+(1-\epsilon)}
	= \dfrac{1}{\epsilon\kappa+(1-\epsilon)} \dfrac{\tan^2\Omega}{\rho_\infty}
$$
\item We compute
$$
	f_0 = \dfrac{1-r_0\sigma}{\kappa r_0-1}
$$
\item We compute/choose
$$
	\beta_7^\infty = \Theta \left[ 1 + f_0 \rho_\infty \cos^2\Omega \cos^2 B_7 \right]
$$

$$	
	\beta_6^\infty = \Theta \left[ 1 + f_0 \dfrac{\kappa}{\sigma} \rho_\infty \cos^2\Omega \cos^2 B_6 \right]
$$
\begin{itemize}
\item
We have $\kappa/\sigma>1$, but we check that we can adjust $\cos^2 B_6$ to retrieve, for example $\beta_6^\infty=\beta_7^\infty$, which imposes
$Q_6>Q_7$, which is expected!
\item $\kappa$ can be adjusted to have the desired amplitude on $\beta_7^\infty$
\item We have
$$
	r_\infty = \dfrac{\beta_7^\infty}{\beta_6^\infty} = \dfrac{ 1 + f_0 \rho_\infty \cos^2\Omega \cos^2 B_7}{1 + f_0 \dfrac{\kappa}{\sigma} \rho_\infty \cos^2\Omega \cos^2 B_6}
$$
so that, without second order kinetics, we would always have a fixed ratio $<1$ (a.k.a $\deltaLi_\infty<\deltaLiOut$)
\end{itemize}
\end{itemize}

\end{document}


