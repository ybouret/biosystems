\documentclass[aps,onecolumn,12pt]{revtex4}
\usepackage{graphicx}
\usepackage{amssymb,amsfonts,amsmath,amsthm}
\usepackage{chemarr}
\usepackage{bm}
\usepackage{pslatex}
\usepackage{mathptmx}
\usepackage{xfrac}
\usepackage{xcolor}

\newcommand{\mychem}[1]{\mathtt{#1}}
\newcommand{\myconc}[1]{\left\lbrack{#1}\right\rbrack}
\newcommand{\LiEin}[1]{\myconc{\left\lbrace\mychem{Li}_{#1}\mychem{E}\right\rbrace_{in}}}
\newcommand{\LiEout}[1]{\myconc{\left\lbrace\mychem{Li}_{#1}\mychem{E}\right\rbrace_{out}}}
\newcommand{\LiIn}[1]{\myconc{\mychem{Li}_{#1}^{in}}}
\newcommand{\LiOut}[1]{\myconc{\mychem{Li}_{#1}}}
\newcommand{\EHin}{\myconc{\mychem{EH}}}
\newcommand{\Eout}{\myconc{\mychem{E}}}
\newcommand{\Hin}{\myconc{\mychem{H}}}

\begin{document}

\begin{equation}
	 \mychem{Li}_{x} +  \mychem{E}  
	 \xrightleftharpoons[k_d^x]{k_a^x} 
	 \left\lbrace\mychem{Li}_{x}\mychem{E}\right\rbrace_{out} 
	  \xrightleftharpoons[k_r^x]{k_f^x} 
	  \left\lbrace\mychem{Li}_{x}\mychem{E}\right\rbrace_{in}  
	  \xrightleftharpoons[k_q^x]{\mychem{H}_i,\;k_p^x} \underbrace{\mychem{EH}_{in}}_{\xrightarrow[]{k_h} \mychem{E} + \mychem{H}_{out}} + \mychem{Li}_{x}^{in}
\end{equation}

\begin{equation}
\displaystyle
\left\lbrace
\begin{array}{rcl}
\partial_t\LiIn{x}   & = & k_p^x \myconc{\mychem{H}}_{in} \LiEin{x} - k_q^x \EHin \LiIn{x} \\
\\
\partial_t \EHin     & = & \sum_{x=6,7}\left( k_p^x \Hin \LiEin{x} - k_q^x \EHin \LiIn{x}\right) - k_h \EHin \\
\\
\partial_t \LiEin{x} & = & -k_p^x \Hin \LiEin{x} + k_q^x \EHin \LiIn{x}
+ k_f^x \LiEout{x} - k_r^x \LiEin{x}\\
\\
\partial_t \LiEout{x} & = & k_r^x \LiEin{x} - (k_f^x+k_d^x) \LiEout{x} + k_a \LiOut{x} \Eout \\
\\
E_0 & = & \Eout + \EHin + \LiEin{6} + \LiEout{6} + \LiEin{7}+\LiEout{7}\\
\end{array}
\right.
\end{equation}

The steady state concentrations are given by
\begin{equation}
\begin{pmatrix}
	1 & 1 & 1 &1 &1 &1 \\
	0 & 0 & 0 & k_p^6\Hin & k_p^7\Hin &-(k_h+k_q^6\LiIn{6}+k_q^7\LiIn{7}) \\
	0 & 0 & 0 & 0 & 0 &0 \\
	0 & 0 & 0 & 0 & 0 &0 \\
	0 & 0 & 0 & 0 & 0 &0 \\
	0 & 0 & 0 & 0 & 0 &0 \\
\end{pmatrix}
\cdot
	\begin{pmatrix}
	\Eout\\
	\LiEout{6}\\
	\LiEout{7}\\
	\LiEin{6}\\
	\LiEin{7}\\
	\EHin\\
	\end{pmatrix}
	=
	\begin{pmatrix}
	E_0\\
	0\\
	0\\
	0\\
	0\\
	0\\
	\end{pmatrix}
\end{equation}

\end{document}

