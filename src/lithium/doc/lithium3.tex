\documentclass[aps,onecolumn,11pt]{revtex4}
%\documentclass[11pt]{article}
%\usepackage[cm]{fullpage}
\usepackage{graphicx}
\usepackage{amssymb,amsfonts,amsmath,amsthm}
\usepackage{chemarr}
\usepackage{bm}
\usepackage{pslatex}
\usepackage{mathptmx}
\usepackage{xfrac}
\usepackage{xcolor}

\newcommand{\mychem}[1]{\mathtt{#1}}
\newcommand{\myconc}[1]{\left\lbrack{#1}\right\rbrack}
\newcommand{\LiEin}[1]{\myconc{\left\lbrace\mychem{Li}_{#1}\mychem{E}\right\rbrace_{in}}}
\newcommand{\LiEout}[1]{\myconc{\left\lbrace\mychem{Li}_{#1}\mychem{E}\right\rbrace_{out}}}
\newcommand{\LiIn}[1]{\myconc{\mychem{Li}_{#1}^{in}}}
\newcommand{\LiOut}[1]{\myconc{\mychem{Li}_{#1}^{out}}}
\newcommand{\EHin}{\myconc{\mychem{EH}}}
\newcommand{\Eout}{\myconc{\mychem{E}}}
\newcommand{\Hin}{\myconc{\mychem{H}}}
\newcommand{\mytrn}[1]{{#1}^{\mathsf{T}}}

\begin{document}
\section{Mechanism}
\begin{equation}
	 \mychem{Li}_{x} +  \mychem{E}  
	 \xrightleftharpoons[k_d^x]{k_a^x} 
	 \left\lbrace\mychem{Li}_{x}\mychem{E}\right\rbrace_{out} 
	  \xrightleftharpoons[k_r^x]{k_f^x} 
	  \left\lbrace\mychem{Li}_{x}\mychem{E}\right\rbrace_{in}  
	  \xrightleftharpoons[k_q^x]{\mychem{H}_i,\;k_p^x} \underbrace{\mychem{EH}_{in}}_{\xrightarrow[]{k_h} \mychem{E} + \mychem{H}_{out}} + \mychem{Li}_{x}^{in}
\end{equation}

\section{Full System}
\subsection{Description}
\begin{equation}
\displaystyle
\left\lbrace
\begin{array}{rcl}
\partial_t\LiIn{x}   & = & k_p^x \Hin \LiEin{x} - k_q^x \EHin \LiIn{x} \\
\\
\partial_t \EHin     & = & \sum_{x=6,7}\left( k_p^x \Hin \LiEin{x} - k_q^x \EHin \LiIn{x}\right) - k_h \EHin \\
\\
\partial_t \LiEin{x} & = & -\left(k_p^x \Hin + k_r^x\right) \LiEin{x} + k_q^x \EHin \LiIn{x}
+ k_f^x \LiEout{x} \\
\\
\partial_t \LiEout{x} & = & k_r^x \LiEin{x} - (k_f^x+k_d^x) \LiEout{x} + k_a^x \LiOut{x} \Eout \\
\\
E_0 & = & \Eout + \EHin + \LiEin{6} + \LiEout{6} + \LiEin{7}+\LiEout{7}\\
\end{array}
\right.
\end{equation}

We have the intrinsic part that dispatch $\LiEin{x}$ and $\LiEout{x}$

\begin{equation}
\underbrace{
\begin{pmatrix}
	\left(k_f^x+k_d^x\right) & -k_r^x\\
	-k_f^x & \left(k_r^x+k_p^x\Hin\right)\\
\end{pmatrix}
}_{M_x}
\begin{pmatrix}
	\LiEout{x}\\
	\LiEin{x}\\
\end{pmatrix}
=
	\begin{pmatrix}
	k_a^x & 0 \\
	0     & k_q^x \\
	\end{pmatrix}
	\begin{pmatrix}
	\LiOut{x} & 0 \\
	0     & \LiIn{x} \\
	\end{pmatrix}
	\begin{pmatrix}
	\Eout\\
	\EHin\\
	\end{pmatrix}
\end{equation}
so that
\begin{equation}
	\begin{pmatrix}
	\LiEout{x}\\
	\LiEin{x}\\
\end{pmatrix}
= \dfrac{1}{\delta_x} 
\underbrace{
\begin{pmatrix}
	\left(k_r^x+k_p^x\Hin\right) & k_r^x\\
	k_f^x & \left(k_f^x+k_d^x\right)\\
\end{pmatrix}}_{S_x}
\underbrace{
\begin{pmatrix}
	k_a^x & 0 \\
	0     & k_q^x \\
	\end{pmatrix}
		}_{\kappa_x}
\underbrace{
	\begin{pmatrix}
	\LiOut{x} & 0 \\
	0     & \LiIn{x} \\
	\end{pmatrix}
}_{\Lambda_x}
	\begin{pmatrix}
	\Eout\\
	\EHin\\
	\end{pmatrix}
\end{equation}
and
\begin{equation}
\delta_x = k_d^x k_r^x + k_p^x \Hin \left(k_f^x+k_d^x\right)
\end{equation}
We write
\begin{equation}
	\begin{pmatrix}
	\LiEout{x}\\
	\LiEin{x}\\
\end{pmatrix}
= 
\Theta_x 
\begin{pmatrix}
	\Eout\\
	\EHin\\
\end{pmatrix}
=\Theta_x \vec{\mathcal{E}}
\end{equation}
We use 
\begin{equation}
	\vec{\sigma} = 
	\begin{pmatrix}
	1\\
	1\\
	\end{pmatrix}, \;\; 
	\vec{p} = 
	\begin{pmatrix}
	0\\
	1\\
	\end{pmatrix}
\end{equation}
and the matter conservation becomes
\begin{equation}
	E_0 = \mytrn{\vec{\sigma}}\left(I_2+\Theta_6+\Theta_7\right) \vec{\mathcal{E}}
\end{equation}

and using
\begin{equation}
	B = 
	\begin{pmatrix}
	0 & 0\\
	0 & k_h\\
	\end{pmatrix}
\end{equation}

we get
\begin{equation}
	\begin{array}{rcl}
		0 & = & \sum_{x=6,7}\left( k_p^x \Hin \LiEin{x} - k_q^x \EHin \LiIn{x}\right) - k_h \EHin\\
		  & = &
			\Hin \left(
			\mytrn{\vec{p}} 
			\left(
			k_p^6\Theta_6 + k_p^7 \Theta_7
			\right)
			\vec{\mathcal{E}}
			\right)
			-\left(k_h + k_q^6 \LiIn{6} + k_q^7 \LiIn{7} \right) \EHin
		  \\
		  & = & \Hin \left(
			\mytrn{\vec{p}} 
			\left(
			k_p^6\Theta_6 + k_p^7 \Theta_7
			\right)
			\vec{\mathcal{E}}
			\right)
			-\mytrn{\vec{p}} \left(B+\kappa_6\Lambda_6+\kappa_7\Lambda_7\right) \vec{\mathcal{E}}
\\
 & = & 
 \mytrn{\vec{p}} \left\lbrack
  \dfrac{1}{\delta_6} \left(k_p^6 \Hin S_6 -\delta_6I_2\right)\kappa_6\Lambda_6 
 +\dfrac{1}{\delta_7} \left(k_p^7 \Hin S_7 -\delta_7I_2\right)\kappa_7\Lambda_7 
 - B
 \right\rbrack \vec{\mathcal{E}}
 \\
	\end{array}
\end{equation}

We are left with
\begin{equation}
\begin{array}{rcl}
	0 & = & \left\lbrack \dfrac{1}{\delta_6}
	\begin{pmatrix} 
	k_f^6 k_p^6 \Hin &
	-k_d^6 k_r^6
	\end{pmatrix}
	\kappa_6\Lambda_6 
	+ 
	\dfrac{1}{\delta_7}
	\begin{pmatrix} 
	k_f^7 k_p^7 \Hin &
	-k_d^7 k_r^7
	\end{pmatrix}
	\kappa_7\Lambda_7 
	-
	\begin{pmatrix} 
	0 &  k_h
	\end{pmatrix}
	\right\rbrack \vec{\mathcal{E}}
	\\
	E_0 & = & \mytrn{\vec{\sigma}}
	\left(I_2+\dfrac{1}{\delta_6}S_6\kappa_6\Lambda_6 
	+\dfrac{1}{\delta_7}S_7\kappa_7\Lambda_7 \right) \vec{\mathcal{E}}\\
\end{array}
\end{equation}


\end{document}

The steady state concentrations are given by
\begin{equation}
\begin{pmatrix}
	1 & 1 & 1 &1 &1 &1 \\
	0 & 0 & 0 & k_p^6\Hin & k_p^7\Hin & -(k_h+k_q^6\LiIn{6}+k_q^7\LiIn{7}) \\
	0 & k_f^6 & 0 & -(k_p^6\Hin+k_r^6) & 0 & k_q^6\LiIn{6} \\
	0 & 0 & k_f^7 & 0 & -(k_p^7\Hin+k_r^7) & k_q^7\LiIn{7} \\
	k_a^6\LiOut{6} & -(k_f^6+k_d^6) & 0 & k_r^6 & 0 & 0 \\
	k_a^7\LiOut{7} & 0 & -(k_f^7+k_d^7) & 0 & k_r^7 & 0 \\
\end{pmatrix}
\end{equation}
\begin{equation}
\cdot
	\begin{pmatrix}
	\Eout\\
	\LiEout{6}\\
	\LiEout{7}\\
	\LiEin{6}\\
	\LiEin{7}\\
	\EHin\\
	\end{pmatrix}
	=
	\begin{pmatrix}
	E_0\\
	0\\
	0\\
	0\\
	0\\
	0\\
	\end{pmatrix}
\end{equation}



\section{Resolution}
As it is an intermediate
\begin{equation}
	\LiEout{x} = \dfrac{k_r^x}{k_f^x+k_d^x} \LiEin{x} + \dfrac{k_a^x\LiOut{x}}{k_f^x+k_d^x} \Eout
	= A_x  \LiEin{x} + B_x \Eout
\end{equation}
Then
\begin{equation}
\EHin = \dfrac{k_p^6 \LiEin{6} + k_p^7 \LiEin{7} }{k_h+k_q^6 \LiIn{6} + k_q^7 \LiIn{7} } \Hin
= C_6 \LiEin{6} + C_7 \LiEin{7}
\end{equation}
We now have three equations using $F_x=k_q^x\LiIn{x},\;B'_x=k_f^xB_x,\;A'_x=k_f^xA_x$
\begin{equation}
\begin{array}{rcl}
0   & = & B'_6 \Eout + \left(A'_6-D_6+F_6C_6\right) \LiEin{6} + F_6C_7 \LiEin{7} \\
0   & = & B'_7 \Eout + \left(A'_7-D_7+F_7C_7\right) \LiEin{7} + F_7C_6 \LiEin{6} \\
E_0 & = & \left(1+B_6+B_7\right)\Eout + \left(1+A_6+C_6\right) \LiEin{6} + \left(1+A_7+C_7\right) \LiEin{7}\\
\end{array}
\end{equation}

\begin{equation}
\begin{pmatrix}
\left(A'_6-D_6+F_6C_6\right) & F_6C_7 & B'_6\\
F_7C_6 & \left(A'_7-D_7+F_7C_7\right) & B'_7\\
\left(1+A_6+C_6\right) & \left(1+A_7+C_7\right) & \left(1+B_6+B_7\right)\\
\end{pmatrix}
\cdot
\begin{pmatrix}
	\LiEin{6}\\
	\LiEin{7}\\
	\Eout\\
\end{pmatrix} 
= 
\begin{pmatrix}
	0\\
	0\\
	E_0\\
\end{pmatrix}
\end{equation}

\section{One Lithium Only}
\begin{equation}
\displaystyle
\left\lbrace
\begin{array}{rcl}
\partial_t\LiIn{\nu}   & = & k_p^\nu \Hin \LiEin{\nu} - k_q^x \EHin \LiIn{\nu} \\
\\
\partial_t \EHin     & = & \left( k_p^\nu \Hin \LiEin{\nu} - k_q^\nu \EHin \LiIn{\nu}\right) - k_h \EHin \\
\\
\partial_t \LiEin{\nu} & = & -\left(k_p^\nu \Hin+k_r^\nu\right) \LiEin{\nu} + k_q^\nu \EHin \LiIn{\nu}
+ k_f^\nu \LiEout{\nu} \\ % - k_r^\nu \LiEin{\nu}\\
\\
\partial_t \LiEout{\nu} & = & k_r^\nu \LiEin{\nu} - (k_f^\nu+k_d^\nu) \LiEout{\nu} + k_a^\nu \LiOut{\nu} \Eout \\
\\
E_0 & = & \Eout + \EHin + \LiEin{\nu} + \LiEout{\nu}\\
\end{array}
\right.
\end{equation}


We use the vector of unknowms
\begin{equation}
\vec{Y}_\nu = 
	\begin{pmatrix}
	\Eout\\
	\EHin\\
	\LiEin{\nu}\\
	\LiEout{\nu}
	\end{pmatrix}
\end{equation}
We have the matrix such that
\begin{equation}
	M_\nu \cdot \vec{Y}_\nu = 
	\begin{pmatrix}
	E_0\\
	0\\
	0\\
	0
	\end{pmatrix}
\end{equation}

\begin{equation}
	M_\nu = 
	\begin{pmatrix}
	1 & 1 & 1 & 1\\
	0 & -(k_h+k_q^\nu\LiIn{\nu}) & k_p^\nu \Hin  & 0\\
	0 & k_q^\nu\LiIn{\nu} & -(k_p^\nu \Hin+k_r^\nu) & k_f^\nu\\
	k_a^\nu\LiOut{\nu} & 0 & k_r^\nu & - (k_f^\nu+k_d^\nu)
	\end{pmatrix}
\end{equation}





\end{document}

