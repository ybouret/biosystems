\documentclass[aps,onecolumn,11pt]{revtex4}
%\documentclass[11pt]{article}
%\usepackage[cm]{fullpage}
\usepackage{graphicx}
\usepackage{amssymb,amsfonts,amsmath,amsthm}
\usepackage{chemarr}
\usepackage{bm}
\usepackage{pslatex}
\usepackage{mathptmx}
\usepackage{xfrac}
\usepackage{xcolor}

\newcommand{\mychem}[1]{\mathtt{#1}}
\newcommand{\myconc}[1]{\left\lbrack{#1}\right\rbrack}
\newcommand{\LiEin}[1]{\myconc{\left\lbrace\mychem{Li}_{#1}\mychem{E}\right\rbrace_{in}}}
\newcommand{\LiEout}[1]{\myconc{\left\lbrace\mychem{Li}_{#1}\mychem{E}\right\rbrace_{out}}}
\newcommand{\LiIn}[1]{\myconc{\mychem{Li}_{#1}^{in}}}
\newcommand{\LiOut}[1]{\myconc{\mychem{Li}_{#1}^{out}}}
\newcommand{\EHin}{\myconc{\mychem{EH}}}
\newcommand{\Eout}{\myconc{\mychem{E}}}
\newcommand{\Hin}{\myconc{\mychem{H}}}
\newcommand{\mytrn}[1]{{#1}^{\mathsf{T}}}

\begin{document}
\section{Mechanism}
\begin{equation}
	 \mychem{Li}_{x} +  \mychem{E}  
	 \xrightleftharpoons[k_d^x]{k_a^x} 
	 \left\lbrace\mychem{Li}_{x}\mychem{E}\right\rbrace_{out} 
	  \xrightleftharpoons[k_r^x]{k_f^x} 
	  \left\lbrace\mychem{Li}_{x}\mychem{E}\right\rbrace_{in}  
	  \xrightleftharpoons[k_q^x]{\mychem{H}_i,\;k_p^x} \underbrace{\mychem{EH}_{in}}_{\xrightarrow[]{k_h} \mychem{E} + \mychem{H}_{out}} + \mychem{Li}_{x}^{in}
\end{equation}

\section{Full System}
\subsection{Description}
\begin{equation}
\displaystyle
\left\lbrace
\begin{array}{rcl}
\partial_t\LiIn{x}   & = & k_p^x \Hin \LiEin{x} - k_q^x \EHin \LiIn{x} \\
\\
\partial_t \EHin     & = & \displaystyle\sum_{x=6,7}\left( k_p^x \Hin \LiEin{x} - k_q^x \EHin \LiIn{x}\right) - k_h \EHin \\
\\
\partial_t \LiEin{x} & = & -\left(k_p^x \Hin + k_r^x\right) \LiEin{x} + k_q^x \EHin \LiIn{x}
+ k_f^x \LiEout{x} \\
\\
\partial_t \LiEout{x} & = & k_r^x \LiEin{x} - (k_f^x+k_d^x) \LiEout{x} + k_a^x \LiOut{x} \Eout \\
\\
E_0 & = & \Eout + \EHin + \LiEin{6} + \LiEout{6} + \LiEin{7}+\LiEout{7}\\
\end{array}
\right.
\end{equation}
That we rewrite as
\begin{equation}
\displaystyle
\left\lbrace
\begin{array}{rcl}
\partial_t\LiIn{x}   & = & k_{p'}^x \LiEin{x} - k_{q'}^x \EHin  \\
\\
\partial_t \EHin     & = & \displaystyle\sum_{x=6,7}\left( k_{p'}^x \LiEin{x} - k_{q'}^x \EHin \right) - k_h \EHin \\
\\
\partial_t \LiEin{x} & = & -\left(k_{p'}^x + k_r^x\right) \LiEin{x} + k_{q'}^x \EHin
+ k_f^x \LiEout{x} \\
\\
\partial_t \LiEout{x} & = & k_r^x \LiEin{x} - (k_f^x+k_d^x) \LiEout{x} + k_{a'}^x  \Eout \\
\\
E_0 & = & \Eout + \EHin + \LiEin{6} + \LiEout{6} + \LiEin{7}+\LiEout{7}\\
\end{array}
\right.
\end{equation}

\subsection{Forward Resolution}
We have the intrinsic part that dispatch $\LiEin{x}$ and $\LiEout{x}$

\begin{equation}
\underbrace{
\begin{pmatrix}
	\left(k_f^x+k_d^x\right) & -k_r^x\\
	-k_f^x & \left(k_r^x+k_{p'}^x\right)\\
\end{pmatrix}
}_{M_x}
\begin{pmatrix}
	\LiEout{x}\\
	\LiEin{x}\\
\end{pmatrix}
=
	\begin{pmatrix}
	k_{a'}^x & 0 \\
	0     & k_{q'}^x \\
	\end{pmatrix}
	\begin{pmatrix}
	\Eout\\
	\EHin\\
	\end{pmatrix}
\end{equation}
so that
\begin{equation}
	\begin{pmatrix}
	\LiEout{x}\\
	\LiEin{x}\\
\end{pmatrix}
= \dfrac{1}{\delta_x} 
\underbrace{
\begin{pmatrix}
	\left(k_r^x+k_{p'}^x\right) & k_r^x\\
	k_f^x & \left(k_f^x+k_d^x\right)\\
\end{pmatrix}}_{S_x}
\underbrace{
\begin{pmatrix}
	k_{a'}^x & 0 \\
	0     & k_{q'}^x \\
	\end{pmatrix}
		}_{\kappa_x}
	\begin{pmatrix}
	\Eout\\
	\EHin\\
	\end{pmatrix}
\end{equation}
and
\begin{equation}
\delta_x = k_d^x k_r^x + k_{p'}^x\left(k_f^x+k_d^x\right) 
\end{equation}
We use 
\begin{equation}
	\vec{\sigma} = 
	\begin{pmatrix}
	1\\
	1\\
	\end{pmatrix}, \;\; 
	\vec{p} = 
	\begin{pmatrix}
	0\\
	1\\
	\end{pmatrix}
\end{equation}
and the matter conservation becomes
\begin{equation}
	E_0 = \mytrn{\vec{\sigma}}\left(I_2+\dfrac{1}{\delta_6}S_6\kappa_6+\dfrac{1}{\delta_7}S_7\kappa_7\right) \vec{\mathcal{E}}
\end{equation}

We also get
\begin{equation}
	\begin{array}{rcl}
	0 & = & \displaystyle\sum_{x=6,7}\left( k_{p'}^x \LiEin{x} - k_{q'}^x \EHin \right) - k_h \EHin\\
	\\
	  & = & \displaystyle\sum_{x=6,7}\left(\dfrac{1}{\delta_x} \mytrn{\vec{p}} 
	  \left(
	  	(k_{p'}^xS_x - \delta_x I_2) \kappa_x
	  \right) \vec{\mathcal{E}} \right) - k_h \EHin\\
	  \\
	  & = & \displaystyle\sum_{x=6,7}\left( \dfrac{1}{\delta_x} 
	  \begin{pmatrix}
	  k_{p'}^x k_f^x & -k_d^x k_r^x\\
	  \end{pmatrix}
	  \kappa_x\vec{\mathcal{E}} 
	  \right)- k_h \EHin\\
	  \\
	  & = & \displaystyle\sum_{x=6,7} \left\lbrack \dfrac{1}{\delta_x}
	  \begin{pmatrix}
	  k_{p'}^x k_f^x k_{a'}^x & -k_d^x k_r^x k_{q'}^x\\
	  \end{pmatrix} 
	  \begin{pmatrix}
	  \Eout\\
	  \EHin\\
	  \end{pmatrix}
	  \right\rbrack - k_h \EHin
	\end{array}
\end{equation}
Using
\begin{equation}
	\alpha_x = k_{p'}^x k_f^x k_{a'}^x,\;\;\beta_x=k_d^x k_r^x k_{q'}^x
\end{equation}
we get
\begin{equation}
	\EHin = \Eout
	\dfrac{\dfrac{\alpha_6}{\delta_6}+\dfrac{\alpha_7}{\delta_7}}
	{k_h+\dfrac{\beta_6}{\delta_6}+\dfrac{\beta_7}{\delta_7}}
	= \rho \Eout
\end{equation}

We note that
\begin{equation}
	\mytrn{\sigma}{S_x} =
	\begin{pmatrix}
	k_r^x + k_f^x + k_{p'}^x & k_r^x + k_f^x + k_d^x\\
	\end{pmatrix}
	=
	\begin{pmatrix}
		\gamma_x & \eta_x\\
	\end{pmatrix}
\end{equation}
and that
\begin{equation}
	\mytrn{\sigma}{S_x}\kappa_x = \begin{pmatrix}
		\gamma'_x = \gamma_x k_{a'}^x & \eta'_x = \eta_x k_{q'}^x\\
	\end{pmatrix}
\end{equation}
we get
\begin{equation}
	E_0 = \left\lbrack
	1 + \dfrac{\gamma'_6}{\delta_6} + \dfrac{\gamma'_7}{\delta_7}
	\right\rbrack \Eout
	+
	\left\lbrack
	1 + \dfrac{\eta'_6}{\delta_6} + \dfrac{\eta'_7}{\delta_7}
	\right\rbrack \EHin
	= \lambda\Eout + \mu\EHin = \left(\lambda + \mu \rho\right) \Eout
\end{equation}

\subsection{Reverse Resolution}
\begin{equation}
	\Eout = \dfrac{E_0}{\lambda + \mu \rho}
\end{equation}
\begin{equation}
	\EHin = E_0 \dfrac{\rho}{\lambda + \mu \rho}
\end{equation}
then
\begin{equation}
\begin{array}{rcl}
	\partial_t\LiIn{x} & = & \dfrac{1}{\delta_x}\left(\alpha_x \Eout - \beta_x \EHin \right)\\
	\\
	 & = & \dfrac{E_0}{\delta_x\left(\lambda + \mu \rho\right)}
	 \left(\alpha_x - \rho \beta_x\right)
	 \\
\end{array}
\end{equation}

\end{document}

