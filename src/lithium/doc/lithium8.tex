\documentclass[aps,onecolumn,11pt]{revtex4}
%\documentclass[11pt]{article}
%\usepackage[cm]{fullpage}
\usepackage{graphicx}
\usepackage{amssymb,amsfonts,amsmath,amsthm}
\usepackage{chemarr}
\usepackage{bm}
\usepackage{pslatex}
\usepackage{mathptmx}
\usepackage{xfrac}
\usepackage{xcolor}

\newcommand{\mychem}[1]{\mathtt{#1}}
\newcommand{\myconc}[1]{\left\lbrack{#1}\right\rbrack}

\newcommand{\spLi}[1]{{~^{\mychem{#1}}\mychem{Li}}}
\newcommand{\Li}[1]{\myconc{\spLi{#1}}}

\newcommand{\spEout}{\mychem{E}}
\newcommand{\Eout}{\myconc{\spEout}}

\newcommand{\spLiEin}[1]{\left\lbrace\spLi{#1}\spEout\right\rbrace_{\mathrm{in}}}
\newcommand{\LiEin}[1]{\myconc{\spLiEin{#1}}}

\newcommand{\spLiEout}[1]{\left\lbrace\spLi{#1}\spEout\right\rbrace_{\mathrm{out}}}
\newcommand{\LiEout}[1]{\myconc{\spLiEout{#1}}}

\newcommand{\spLiIn}[1]{{\spLi{#1}}_{\mathrm{in}}}
\newcommand{\LiIn}[1]{\myconc{\spLiIn{#1}}}

\newcommand{\spLiOut}[1]{{\spLi{#1}}_{\mathrm{out}}}
\newcommand{\LiOut}[1]{\myconc{\spLiOut{#1}}}

\newcommand{\spEHin}{\mychem{EH}}
\newcommand{\EHin}{\myconc{\spEHin}}
\newcommand{\spproton}{\mychem{H}}
\newcommand{\proton}{\myconc{\spproton}}

\begin{document}

\section{Mechanism}
$$
	\delta^7Li = \left(
		\dfrac{\left(\dfrac{Li^7}{Li^6}\right)_{sample}}
		{\left(\dfrac{Li^7}{Li^6}\right)_{standard}}
		 -1 
	\right) \times 1000
$$


\begin{equation}
	 \spLiOut{x} +  \spEout  
	 \xrightleftharpoons[k_d^x]{k_a^x} 
	 \spLiEout{x}
	  \xrightleftharpoons[k_r^x]{k_f^x} 
	  \spLiEin{x} 
	  \xrightleftharpoons[k_q^x]{\mychem{H},\;k_p^x} \underbrace{\spEHin}_{\xrightarrow[]{k_h} \mychem{E} + \mychem{H}_{\mathrm{out}}} + \spLiIn{x}
\end{equation}

\section{Hypothesis}
\begin{itemize}
\item The flip stage is slower than the enzyme interaction with lithium.
\item $\proton$ is a parameter
\item $\LiOut{6}$ and  $\LiOut{7}$ are parameters
\end{itemize}

Accordingly, we consider that we have the four equations
\begin{equation}
%\left\lbrace
	\begin{array}{rcll}
	 \spLiOut{x} +  \spEout &  \xrightleftharpoons[]{} & \spLiEout{x}, & J_x = \dfrac{\LiEout{x}}{\LiOut{x} \Eout} = \dfrac{k_a^x}{k_d^x}\\
	 \\
	 \spLiEin{x} & \xrightleftharpoons[k_q^x]{\mychem{H},\;k_p^x} & {\spEHin}+ \spLiIn{x}, &K_x = \dfrac{\EHin \LiIn{x}}{\LiEin{x}\proton} = \dfrac{k_p^x}{k_q^x} \\
	\end{array}
	%\right.
\end{equation}

We have the vector of concentrations
\begin{equation}
	\vec{X} = 
	\begin{pmatrix}
	%\proton\\
	\Eout\\
	\EHin\\
	\LiEout{6}\\
	\LiEin{6}\\
	\LiIn{6}\\
	\LiEout{7}\\
	\LiEin{7}\\
	\LiIn{7}\\
	\end{pmatrix}
\end{equation}
The chemical coupling 4-vector is
\begin{equation}
\vec{\Gamma} = 
\begin{pmatrix}
J_6 \LiOut{6} \Eout - \LiEout{6}\\
J_7 \LiOut{7} \Eout - \LiEout{7}\\
K_6 \LiEin{6}\proton - \EHin \LiIn{6}\\
K_7 \LiEin{7}\proton - \EHin \LiIn{7}\\
\end{pmatrix}
\end{equation}

\end{document}