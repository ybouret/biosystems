\documentclass[aps,onecolumn,11pt]{revtex4}
%\documentclass[11pt]{article}
%\usepackage[cm]{fullpage}
\usepackage{graphicx}
\usepackage{amssymb,amsfonts,amsmath,amsthm}
\usepackage{chemarr}
\usepackage{bm}
\usepackage{pslatex}
\usepackage{mathptmx}
\usepackage{xfrac}
\usepackage{xcolor}

\newcommand{\mychem}[1]{\mathtt{#1}}
\newcommand{\myconc}[1]{\left\lbrack{#1}\right\rbrack}

\newcommand{\spLi}[1]{{~^{\mychem{#1}}\mychem{Li}}}
\newcommand{\Li}[1]{\myconc{\spLi{#1}}}

\newcommand{\spEout}{\mychem{E}}
\newcommand{\Eout}{\myconc{\spEout}}

\newcommand{\spLiEin}[1]{\left\lbrace\spLi{#1}\spEout\right\rbrace_{\mathrm{in}}}
\newcommand{\LiEin}[1]{\myconc{\spLiEin{#1}}}

\newcommand{\spLiEout}[1]{\left\lbrace\spLi{#1}\spEout\right\rbrace_{\mathrm{out}}}
\newcommand{\LiEout}[1]{\myconc{\spLiEout{#1}}}

\newcommand{\spLiIn}[1]{{\spLi{#1}}_{\mathrm{in}}}
\newcommand{\LiIn}[1]{\myconc{\spLiIn{#1}}}

\newcommand{\spLiOut}[1]{{\spLi{#1}}_{\mathrm{out}}}
\newcommand{\LiOut}[1]{\myconc{\spLiOut{#1}}}

\newcommand{\spEHin}{\mychem{EH}}
\newcommand{\EHin}{\myconc{\spEHin}}
\newcommand{\spproton}{\mychem{H}}
\newcommand{\proton}{\myconc{\spproton}}

\begin{document}

\section{Mechanism}
$$
	\delta^7Li = \left(
		\dfrac{\left(\dfrac{Li^7}{Li^6}\right)_{sample}}
		{\left(\dfrac{Li^7}{Li^6}\right)_{standard}}
		 -1 
	\right) \times 1000
$$


\begin{equation}
	 \spLiOut{x} +  \spEout  
	 \xrightleftharpoons[k_d^x]{k_a^x} 
	 \spLiEout{x}
	  \xrightleftharpoons[k_r^x]{k_f^x} 
	  \spLiEin{x} 
	  \xrightleftharpoons[k_q^x]{\mychem{H},\;k_p^x} \underbrace{\spEHin}_{\xrightarrow[]{k_h} \mychem{E} + \mychem{H}_{\mathrm{out}}} + \spLiIn{x}
\end{equation}

\section{Hypothesis}
\begin{itemize}
\item The flip stage is slower than the enzyme interaction with lithium.
\item $\proton$ is a parameter
\item $\LiOut{6}$ and  $\LiOut{7}$ are parameters
\end{itemize}

Accordingly, we consider that we have the four equations
\begin{equation}
%\left\lbrace
	\begin{array}{rcll}
	 \spLiOut{x} +  \spEout &  \xrightleftharpoons[]{} & \spLiEout{x}, & J_x = \dfrac{\LiEout{x}}{\LiOut{x} \Eout} = \dfrac{k_a^x}{k_d^x}\\
	 \\
	 \spLiEin{x} & \xrightleftharpoons[k_q^x]{\mychem{H},\;k_p^x} & {\spEHin}+ \spLiIn{x}, &K_x = \dfrac{\EHin \LiIn{x}}{\LiEin{x}\proton} = \dfrac{k_p^x}{k_q^x} \\
	\end{array}
	%\right.
\end{equation}

We have the vector of concentrations
\begin{equation}
	\vec{X} = 
	\begin{pmatrix}
	\Eout\\
	\EHin\\
	\LiEout{6}\\
	\LiEin{6}\\
	\LiIn{6}\\
	\LiEout{7}\\
	\LiEin{7}\\
	\LiIn{7}\\
	\end{pmatrix}
\end{equation}
The chemical coupling 4-vector is
\begin{equation}
\vec{\Gamma} = 
\begin{pmatrix}
J_6 \LiOut{6} \Eout - \LiEout{6}\\
J_7 \LiOut{7} \Eout - \LiEout{7}\\
K_6 \LiEin{6}\proton - \EHin \LiIn{6}\\
K_7 \LiEin{7}\proton - \EHin \LiIn{7}\\
\end{pmatrix}
\end{equation}

We will substitute
\begin{equation}
\left\lbrace
\begin{array}{rcl}
	\LiEout{x} & = & J_x \LiOut{x} \Eout\\
	\\
	\LiEin{x}  & = & \dfrac{\EHin \LiIn{x}}{K_x\proton}\\
\end{array}
\right.
\end{equation}
in the matter conservation
\begin{equation}
\left\lbrace
\begin{array}{rcl}
	E_0 & = &\displaystyle \Eout + \EHin + \sum_x \left(\LiEout{x}+\LiEin{x}\right)\\
	\\
	& = & \Eout \left\lbrack1+J_6 \LiOut{6} + J_7 \LiOut{7}\right\rbrack + 
	\EHin 
	\left\lbrack
		1+\dfrac{1}{h}\left(\dfrac{\LiIn{6}}{K_6}+\dfrac{\LiIn{7}}{K_7}\right)
	\right\rbrack
	\\
\end{array}
\right.
\end{equation}


at $t=0$,
\begin{equation}
	\vec{X}_0 = 
	\begin{pmatrix}
	\Eout_0    & = & \frac{E_0}{1+J_6 \LiOut{6} + J_7 \LiOut{7}}  \\
	\EHin_0    & = & 0 \\
	\LiEout{6} & = & \frac{J_6 \LiOut{6} E_0} {1+J_6 \LiOut{6} + J_7 \LiOut{7}} \\
	\LiEin{6}  & = & 0 \\
	\LiIn{6}   & = & 0 \\
	\LiEout{7} & = & \frac{J_7 \LiOut{7} E_0} {1+J_6 \LiOut{6} + J_7 \LiOut{7}}   \\
	\LiEin{7}  & = & 0 \\
	\LiIn{7}   & = & 0 \\
	\end{pmatrix}
\end{equation}

Firstly, we obtain
\begin{equation}
	\vec{Y} = 
	\begin{pmatrix}
	\EHin\\
	\LiIn{6}\\
	\LiIn{7}\\
	\end{pmatrix}
	,\;\;
	\partial_t \vec{Y} = \vec{F}\left(\vec{Y}\right) 
\end{equation}

\section{First Order Approximation}
\subsection{Expression}
We use a first order approximation
\begin{equation}
		\partial_t \vec{Y} \simeq \vec{F}_0 + \Omega\cdot\vec{Y}
\end{equation}
which admits a non trivial steady state solution
\begin{equation}
	\vec{Y}^\star = \begin{pmatrix}
	0\\
	K_6' \rho_6\\
	K_7'\rho_7\\
	\end{pmatrix}
\end{equation}
when we define
\begin{equation}
	\begin{array}{rcl}
	\omega_6 & = & \frac{J_6' k_f^6}{1+J_6'+J_7'}\\
	\omega_7 & = & \frac{J_7' k_f^7}{1+J_6'+J_7'}\\
	\omega_0 & = & \omega_6+\omega_6\\
	\rho_6   & = & \frac{\omega_6}{\omega_0}\\
	\rho_7   & = & \frac{\omega_7}{\omega_0}\\
	1        & = & \rho_6 + \rho_7\\
	\end{array}
\end{equation}
So we rescale the concentration using
\begin{equation}
	S = 
	\begin{pmatrix}
	\frac{1}{E_0} & 0 & 0 \\
	0 & \frac{1}{\alpha_6}&0\\
	0 & 0 & \frac{1}{\alpha_7}\\
	\end{pmatrix}
\end{equation}
and
\begin{equation}
	\vec{Z} = S \vec{Y}%,\;\;\vec{Z}^\star = \begin{pmatrix} 0\\1\\1\\ \end{pmatrix}
\end{equation}
With
\begin{equation}
\begin{array}{rcl}
	\omega_6 & = & \frac{J_6'k_f^6}{1+J_6'+J7'}\\
	\omega_7 & = & \frac{J_7'k_f^7}{1+J_6'+J7'}\\
	\omega_0 & = & \omega_6+\omega_7\\
\end{array}
\end{equation}

\begin{equation}
	\frac{1}{\omega_0}\partial_t \vec{Z} = 
	\underbrace{
	\begin{pmatrix}
	1\\
	\frac{E_0}{\alpha_6}\rho_6\\
	\frac{E_0}{\alpha_7}\rho_7\\
	\end{pmatrix}}_{\vec{\lambda}}
	-
	\underbrace{
	\begin{pmatrix}
	1+\rho_h & \frac{\alpha_6}{K_6'} & \frac{\alpha_7}{K_7'}\\
	\frac{E_0}{\alpha_6}\rho_6 & \frac{E_0}{K_6'} & 0 \\
	\frac{E_0}{\alpha_7}\rho_7 & 0      & \frac{E_0}{K_7'}\\
	\end{pmatrix}
	}_{\mu}
	\vec{Z}
	 ,\;\; \rho_h = \frac{k_h}{\omega_0}
\end{equation}
and we now rescale in time
\begin{equation}
\tau = \omega_0 t, \;\; \partial_\tau \vec{Z} = \vec{\lambda} - \mu \vec{Z}
\end{equation}

\subsection{Choices}
If we use $\alpha_6=\LiIn{6}^\star$ and $\alpha_7=\LiIn{7}^\star$ then we get
\begin{equation}
\vec{\lambda} =
\begin{pmatrix}
1\\
\lambda_6\\
\lambda_7\\
\end{pmatrix}
,\;\;
\mu=
\begin{pmatrix}
1+\rho_h & \rho_6 & \rho_7\\
\lambda_6 & \lambda_6 & 0 \\
\lambda_7 & 0 & \lambda_7\\
\end{pmatrix}
\end{equation}

\section{"Real" Steady States}
\begin{equation}
	\begin{array}{rcl}
	\partial_t  \Eout & = & \sigma_h  + \sum_x\left(\sigma_d^x - \sigma_a^x\right)\\
	\partial_t  \LiEout{x} & = & \sigma_a^x+\sigma_r^x - \sigma_d^x-\sigma_f^x\\
	\partial_t  \LiEin{x}  & = & \sigma_f^x + \sigma_q^x - \sigma_r^x - \sigma_p^x\\
	\partial_t  \LiIn{x}   & = &  \sigma_p^x - \sigma_q^x\\
	\partial_t  \EHin & = & -\sigma_h + \sum_x\left(\sigma_p^x - \sigma_q^x\right)\\
	\end{array}
\end{equation}
and
\begin{equation}
	E_0 = \Eout + \EHin + \sum_x \left(\LiEout{x} + \LiEin{x} \right)
\end{equation}

\end{document}