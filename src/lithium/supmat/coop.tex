\documentclass[aps,onecolumn,12pt]{revtex4}
\usepackage{graphicx}
\usepackage{amssymb,amsfonts,amsmath,amsthm}
\usepackage{chemarr}
\usepackage{bm}
\usepackage{pslatex}
\usepackage{xfrac}
\usepackage[dvipsnames]{xcolor}
%\usepackage{bookman}
\usepackage{dsfont}
\usepackage{mathptmx}
\usepackage{hyperref}
%\usepackage{rotating}
\usepackage{fancybox}

\newcommand{\mychem}[1]{\mathtt{#1}}
\newcommand{\myconc}[1]{\left[#1\right]}
\newcommand{\spx}{\mychem{X}}
\newcommand{\spproton}{\mychem{H^+}}
\newcommand{\proton}{\myconc{\spproton}}

\newcommand{\myrotate}[2]{\rotatebox[origin=c]{#1}{#2}}

\newcommand{\mytrn}[1]{{#1}^{\!\mathsf{T}}}
\newcommand{\mymat}[1]{{\bm{#1}}}
\newcommand{\mydet}[1]{{\left|{#1}\right|}}

\begin{document}
\title{Cooperativity}
\maketitle
\tableofcontents

\section{Passive Behaviour}
We have two species $\spx_1$ and $\spx_2$ that may leak throw the membrane
\begin{equation}
	\partial_t \left.\myconc{\spx_u}\right\vert_{leak} = k_u \left( e^{\dfrac{-z_u F V_m }{RT}} C_u - \myconc{\spx_u}\right)
\end{equation}

\section{One Layer}

\subsection{Scheme}
\begin{equation}
\boxed{
\begin{array}{ccccc}
 & & E_{01}  &  & \\
 &  \myrotate{45}{$\xrightleftharpoons[d_{01}]{+\spx_1^{out},\;a_{01}}$} &   & \myrotate{-45}{$\xrightarrow[-\spx_1^{in}]{+ \spproton, \; k^p_1}$} &  \\
E_{00}  &  & \xleftarrow{\text{ recycling } k_h } &   & E_{0H} \\
  & \myrotate{-45}{$\xrightleftharpoons[d_{02}]{+\spx_2^{out},\;a_{02}}$} &   & \myrotate{+45}{$\xrightarrow[-\spx_2^{in}]{+ \spproton, \; k^p_2}$} & \\
 & & E_{02} & & \\
 \end{array}
 }
\end{equation}

\subsection{Equations}

\begin{equation}
\left\lbrace
\begin{array}{rcl}
\partial_t E_{00} & = & D_{01}-A_{01} + D_{02}-A_{02} + v_h\\
\partial_t E_{01} & = & -D_{01}+A_{01} -v^p_1 \\
\partial_t E_{02} & = & -D_{02}+A_{02} -v^p_2\\
\partial_t E_{0H} & = & v^p_1 + v^p_2 - v_h\\
\mathfrak{E}       & = & E_{00}+E_{01}+E_{02} + E_{0H} = E_{0H} + {\displaystyle \sum_{x\leq y\leq 1} E_{xy}}\\
\end{array}
\right.
\end{equation}
with
\begin{equation}
\left\lbrace
\begin{array}{rcl}
C_1    & = & \myconc{\spx_1^{out}}\\
C_2    & = & \myconc{\spx_2^{out}}\\
A_{01} &= &a_{01} E_{00} C_1\\
A_{02} &= &a_{02} E_{00} C_2\\
D_{01} &= &d_{01} E_{01}\\
D_{02} &= &d_{02} E_{02}\\
v^p_1  &=& k^p_1 E_{01} \proton\\
v^p_2  &=& k^p_2 E_{02} \proton\\
v_h    &=& k_x  E_{0H}
\end{array}
\right.
\end{equation}

\subsection{Steady-State}
\subsubsection{Raw Equations}
We assume that the internal components are at a steady-state:
\begin{equation}
\left\lbrace
\begin{array}{rcll}
	E_{01}^\star & = & \dfrac{a_{01}}{d_{01}+k^p_1 \proton } E_{00} C_1 & = J_1 E_{00} C_1\\
	\\
	E_{02}^\star & = & \dfrac{a_{02}}{d_{02}+k^p_2 \proton } E_{00} C_2 & = J_2 E_{00} C_2\\
\end{array}
\right.
\end{equation}

\centerline{\shadowbox{\it We expect a simplification if $d_{01},d_{02} \gg k^p_1 \proton, k^p_2 \proton$.}}

Finally:
\begin{equation}
\mathfrak{E} = E_{00}\left[1+J_1C_1+J_2C_2\right] + E_{0H} \Leftrightarrow E_{00} = \dfrac{\mathfrak{E}-E_{0H}}{{1+J_1C_1+J_2C_2}}
\end{equation}
and:
\begin{equation}
\left\lbrace
\begin{array}{rcl}
	E_{01}^\star & = & \dfrac{J_1C_1}{1+J_1C_1+J_2C_2} \left(\mathfrak{E}-E_{0H}\right)\\
	\\
	E_{02}^\star & = & \dfrac{J_2C_2}{1+J_1C_1+J_2C_2} \left(\mathfrak{E}-E_{0H}\right)
\end{array}
\right.
\end{equation}
so that we have the three equations:
\begin{equation}
\left\lbrace
\begin{array}{rcl}
	\partial_t \myconc{X_u}  & = & v^p_u +\partial_t \left.\myconc{\spx_u}\right\vert_{leak}  = \dfrac{k^p_u J_uC_u}{1+J_1C_1+J_2C_2} \left(\mathfrak{E}-E_{0H}\right) \proton + k_u \left( e^{\dfrac{-z_u F V_m }{RT}} C_u - \myconc{\spx_u}\right) \\
	\\
	\partial_t E_{0H} & = & -k_h E_{0H} + \dfrac{k^p_1 J_1C_1 + k^p_2 J_2 C_2 }{1+J_1C_1+J_2C_2} \left(\mathfrak{E}-E_{0H}\right) \proton \\
\end{array}
\right.
\end{equation}
\subsubsection{Normalisation}
\begin{equation}
\left\lbrace
\begin{array}{rcl}
	C_u & = & \epsilon_u C_0\\
	\\
	\beta_u & = & \dfrac{\myconc{X_u}}{C_u}\\
	\\
	\alpha  & = & \dfrac{E_{0H}}{\mathfrak{E}}\\
	\\
	J_0 & = & \epsilon_1 J_1  + \epsilon_2 J_2 \\
\end{array}
\right.
\end{equation}

Leading to the three equations:
\begin{equation}
\left\lbrace
\begin{array}{rcl}
	\partial_t \beta_u & = & \mathfrak{E} \left[\dfrac{k^p_u J_u}{1+J_0 C_0}\right] \left(1-\alpha\right) \proton
	 + k_u \left( e^{\dfrac{-z_u F V_m }{RT}}- \beta_u\right)\\
	\\
	\partial_t \alpha  & = &  -k_h\alpha + C_0 \left[\dfrac{k^p_1 \epsilon_1 J_1  + k^p_2 \epsilon_2 J_2   }{1+J_0C_0}\right] \left(1-\alpha\right) \proton\\
\end{array}
\right.
\end{equation}

\section{Two Layers}
\subsection{Scheme}

\begin{equation}
\boxed{
\begin{array}{ccccccc}
 & &        &                                                  & E_{11} & & \\
 & &        & \myrotate{45}{$\xrightleftharpoons[d_{11}]{+\spx_1^{out},\;a_{11}}$} & &  \myrotate{-45}{$\xrightarrow[-\spx_1^{in}]{+ \spproton, \; k^p_{11}}$}& \\
 & & E_{01} &  & \xleftarrow{\text{ recycling } k_h } & & E_{1H}\\
 &  \rotatebox[origin=c]{45}{$\xrightleftharpoons[d_{01}]{+\spx_1^{out},\;a_{01}}$} &   & \myrotate{-45}{$\xrightleftharpoons[d_{12}]{+\spx_2^{out},\;a_{12}}$} & & \rotatebox[origin=c]{45}{$\xrightarrow[-\spx_2^{in}]{+ \spproton, \; k^p_{12}}$}&\\
E_{00} & &  & & E_{12} & & \\ 
  & \myrotate{-45}{$\xrightleftharpoons[d_{02}]{+\spx_2^{out},\;a_{02}}$} &  & \myrotate{45}{$\xrightleftharpoons[d_{21}]{+\spx_1^{out},\;a_{21}}$} & & \myrotate{-45}{$\xrightarrow[-\spx_1^{in}]{+ \spproton, \; k^p_{21}}$} & \\
  & & E_{02} &   & \xleftarrow{\text{ recycling } k_h } & & E_{2H}\\
  & &  & \myrotate{-45}{$\xrightleftharpoons[d_{22}]{+\spx_2^{out},\;a_{22}}$} & & \myrotate{45}{$\xrightarrow[-\spx_2^{in}]{+ \spproton, \; k^p_{22}}$} &\\
  & &  &  & E_{22} & &\\

 \end{array}
 }
\end{equation}

\subsection{Equations}
\begin{equation}
\left\lbrace
\begin{array}{rcl}
\partial_t E_{00} & = & D_{01}-A_{01} + D_{02}-A_{02}\\
\partial_t E_{01} & = & -D_{01}+A_{01} - A_{11} + D_{11} - A_{12} + D_{12} + v_{1H}\\
\partial_t E_{02} & = & -D_{02}+A_{02} - A_{22} + D_{22} - A_{21} + D_{21} + v_{2H}\\
\partial_t E_{11} & = & A_{11}-D_{11} -v^p_{11}\\
\partial_t E_{12} & = & A_{12}-D_{12} + A_{21}-D_{21} - (v^p_{12}+v^p_{21})\\
\partial_t E_{22} & = & A_{22}-D_{22} - v^p_{22}\\
\partial_t E_{1H} & = & v^p_{11}+v^p_{12} - v_{1H}\\
\partial_t E_{2H} & = & v^p_{22}+v^p_{21} - v_{2H}\\
\mathfrak{E}      & = & {\displaystyle \sum_{x\leq y\leq 2} E_{xy}}+E_{1H}+E_{2H}\\
\end{array}
\right.
\end{equation}
with
\begin{equation}
\left\lbrace
\begin{array}{rcl}
A_{01} &= &a_{01} E_{00} C_1\\
A_{02} &= &a_{02} E_{00} C_2\\
D_{01} &= &d_{01} E_{01}\\
D_{02} &= &d_{02} E_{02}\\
A_{11} &= & a_{11} E_{01} C_1 \\
D_{11} &= &d_{11} E_{11}\\
A_{22} &= &a_{22} E_{02} C_2 \\
D_{22} &= &d_{22} E_{22}\\
A_{12} & = & a_{12} E_{01} C_2\\
D_{12} & = & d_{12} E_{12}\\
A_{21} & = & a_{21} E_{02} C_1\\
D_{21} & = & d_{21} E_{12}\\
v_{1H} & = & k_h E_{1H}\\
v_{2H} & = & k_h E_{2H}\\
v^p_{11} & = & k^p_{11} E_{11} \proton \\
v^p_{22} & = & k^p_{22} E_{22} \proton \\
v^p_{12} & = & k^p_{12} E_{12} \proton \\
v^p_{21} & = & k^p_{21} E_{12} \proton \\
\end{array}
\right.
\end{equation}

And
\begin{equation}
	\partial_t \myconc{\spx_1} = v^p_{11}+v^p_{21} + \partial_t \myconc{\spx_1}\vert_{leak},\;\;
	\partial_t \myconc{\spx_2} = v^p_{22}+v^p_{12} + \partial_t \myconc{\spx_2}\vert_{leak}
\end{equation}

\subsection{Steady State}

\subsubsection{Second layer as a function of the first layer}


Using the three equations describing the second layer and assuming that the internal pre-equilibrium is fast, we express:
\begin{equation}
\boxed{
E_{1\leq x \leq y \leq 2 } = <{\vec{C}} \vert \mymat{F_{xy}} \vert \vec{E}_1 >
}
\end{equation}
with
\begin{equation}
\left\lbrace
\begin{array}{rcl}
\vec{C}   & = & \begin{bmatrix} C_1 \\ C2 \end{bmatrix}\\
\\
\vec{E}_1 & = & \begin{bmatrix} E_{01} \\ E_{02} \end{bmatrix}\\
\\
\mymat{F}_{11} & = & 
\begin{bmatrix}
	f_{11} & 0 \\
	0 & 0\\
\end{bmatrix}, \; f_{11} = \dfrac{a_{11}}{d_{11}}\\
\\
\mymat{F}_{22} & = & 
\begin{bmatrix}
	0 & 0 \\
	0 & f_{22}\\
\end{bmatrix}, \; f_{22} = \dfrac{a_{22}}{d_{22}}\\
\\
\mymat{F}_{12} & = & 
\begin{bmatrix}
	0 & f_{21}\\
	f_{12} & 0\\
\end{bmatrix}, \; f_{12} = \dfrac{a_{12}}{d_{21}+d_{12}},\; f_{21} = \dfrac{a_{21}}{d_{21}+d_{12}}\\
\end{array}
\right.
\end{equation}
and the second layer mass is
\begin{equation}
E_{11} + E_{12} + E_{22} = <{\vec{C}} \vert \mymat{F} \vert \vec{E}_1 >, \;\;
 \mymat{F} 
 = \begin{bmatrix}
	f_{11} & f_{21}\\
	f_{12} & f_{22}\\
\end{bmatrix}
\end{equation}

\subsubsection{First layer as a function of the external components}
We define:
\begin{equation}
\left\lbrace
\begin{array}{rcl}
K_1 & = & d_{12} f_{21} \\
\\
K_2 & = &  d_{21} f_{12}\\
\\
\mathfrak{D} & = & d_{01} d_{02}+ C_2 K_2 d_{02}+C_1K_1 d_{01}\\
\\
\mymat{M}_1  & = &  \begin{bmatrix}
   d_{02} + K_1 C_1 & K_1 C_1 \\
   K_2C_2 & d_{01} + K_2 C_2 \\
 \end{bmatrix}
\\
\\
\vec{E}_H & = & \begin{bmatrix}
 E_{1H}\\
 E_{2H}\\
 \end{bmatrix}\\
 \\
 \mymat{A}_0 & = & \begin{bmatrix}
 a_{01} & 0 \\
 0 & a_{02}\\
 \end{bmatrix}\\
\end{array}
\right.
\end{equation}
to write:
\begin{equation}
\boxed{
	\vec{E}_1 =  \dfrac{1}{ \mathfrak{D} }
	\mymat{M}_1	
 \left(
k_h 
 \vec{E}_H
 +E_{00} 
 \mymat{A}_0
 \vec{C}
\right)
}
\end{equation}

\subsubsection{Coupling by mass conservation}
We use the vectors
\begin{equation}
	\vec{P}_1 = 
	\begin{bmatrix}
	1\\
	1\\
	\end{bmatrix},
	\;\;
\vec{Q}_1  =  \vec{P}_1 + \mytrn{\mymat{F}}\vec{C} \\
\end{equation}

\begin{equation}
\left\lbrace
\begin{array}{rcl}
\mathfrak{E} & = & E_{00} + \underbrace{E_{01} + E_{02}}_{<\vec{P}_1|\vec{E}_1>} + \underbrace{E_{11} + E_{12} + E_{22}}_{<\vec{C}|\mymat{F}|\vec{E}_1>} + E_{1H} + E_{2H} \\
\\
\mathfrak{E} & = & E_{00} + <\vec{Q}_1|\vec{E}_1> +  <\vec{P}_1|\vec{E}_H>\\
\\
\mathfrak{D}\mathfrak{E} & = & 
\mathfrak{D} E_{00} + 
<\vec{Q}_1|\mymat{M}_1|\left(
	k_h  \vec{E}_H +E_{00} \mymat{A}_0 \vec{C} \right)> +  \mathfrak{D} <\vec{P}_1|\vec{E}_H>\\
	\\
 & = & E_{00} \left( \mathfrak{D} + <\vec{Q}_1|\mymat{M}_1|\mymat{A}_0\vec{C}>\right) + 
\mathfrak{D} <\vec{P}_1|\vec{E}_H> + k_h <\vec{Q}_1|\mymat{M}_1|\vec{E}_H>\\
\\
& = & E_{00} \left( \mathfrak{D} + <\mytrn{\mymat{M}}_1\vec{Q}_1|\mymat{A}_0\vec{C}>\right) + < \mathfrak{D} \vec{P}_1 + k_h\mytrn{\mymat{M}}_1 \vec{Q_1} | \vec{E}_H>\\
\end{array}
\right.
\end{equation}

\end{document}

\end{document}
