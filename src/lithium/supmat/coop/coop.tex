\documentclass[aps,onecolumn,11pt]{revtex4}
\usepackage{graphicx}
\usepackage{amssymb,amsfonts,amsmath,amsthm}
\usepackage{chemarr}
\usepackage{bm}
\usepackage{pslatex}
\usepackage{xfrac}
\usepackage[dvipsnames]{xcolor}
\usepackage{bookman}
\usepackage{dsfont}
%\usepackage{mathptmx}
%\usepackage{hyperref}
%\usepackage{rotating}
\usepackage{fancybox}

\newcommand{\mychem}[1]{\mathtt{#1}}
\newcommand{\myconc}[1]{\big[#1\big]}

\newcommand{\Faraday}{\mathcal{F}}
\newcommand{\spLi}[1]{{\!~^{#1}\mychem{Li}}}
\newcommand{\Li}[1]{\myconc{\spLi{#1}}}
\newcommand{\spproton}{\mychem{H}}
\newcommand{\proton}{\myconc{\spproton}}

\newcommand{\deltaLi}{\delta\!\!\spLi{7}}
\newcommand{\myleak}[1]{\left.{#1}\right\vert_{\mathrm{leak}}}
\newcommand{\myout}[1]{{#1}_{\mathrm{out}}}
\newcommand{\LiOut}[1]{\myout{\Li{#1}}}
\newcommand{\spLiOut}[1]{\myout{\spLi{#1}}}

\newcommand{\myrotate}[2]{\rotatebox[origin=c]{#1}{#2}}

\newcommand{\mytrn}[1]{{#1}^{\!\mathsf{T}}}
\newcommand{\mymat}[1]{{\bm{#1}}}
\newcommand{\mydet}[1]{{\left|{#1}\right|}}

\usepackage{ifthen}


\newcommand{\LiAll}{\Lambda}
\newcommand{\LiAllOut}{\myout{\LiAll}}
\newcommand{\deltaLiOut}{\myout{\deltaLi}}
\newcommand{\inpLi}[1]{\text{+}\spLiOut{#1}}
\newcommand{\outLi}[1]{\text{-}\spLiOut{#1}}

\newcommand{\mycolor}[2]{\ifthenelse{\equal{#1}{6}}{{\color{Red}#2}}{{\color{Green}#2}}}



\begin{document}
\title{Lithium Isotopic Separation:\\ From Electroosmosis to NHE Cooperativity}
\maketitle
%\tableofcontents

\section{Description}
\subsection{Isotopic Separation}
We have two species $\spLi{6}$ and $\spLi{7}$ and a solution can be defined by the isotopic 
separation:
\begin{equation}
	\deltaLi = 10^3 \left[ 
	\dfrac{
	\left(\dfrac{\Li{7}}{\Li{6}}\right)
	}
	{
	\left(\dfrac{\Li{7}}{\Li{6}}\right)_{standard}
	}
	- 1 \right]
\end{equation}
and alternatively:
\begin{equation}
	\left(\dfrac{\Li{7}}{\Li{6}}\right) = \lambda_s \left[ 1+10^{-3} \deltaLi \right]
\end{equation}
where we are using:
\begin{equation}
	\lambda_s = \left(\dfrac{\Li{7}}{\Li{6}}\right)_{standard} %=  12.0192
\end{equation}
We define:
\begin{equation}
	\LiAll = \Li{6}+\Li{7}
\end{equation}
and we obtain the relations:
\begin{equation}
\left\lbrace
\begin{array}{rcl}
\Li{6} & = & \dfrac{1}{1+\lambda_s[1+10^{-3}\deltaLi]} \LiAll \\
\\
\Li{7} & = &  \dfrac{\lambda_s[1+10^{-3}\deltaLi]}{1+\lambda_s[1+10^{-3}\deltaLi]}\LiAll\\
\end{array}
\right.
\end{equation}
\subsection{Numerical Values}
For the current experiments:
\begin{equation}
\left\lbrace
\begin{array}{rcl}
	\lambda_s   & = & 12.0192\\
	\deltaLiOut & = & 14.57\\
	\LiOut{6}   & = & \epsilon_6 \LiAllOut, \;\;\epsilon_6\simeq 0.076\\
	\LiOut{7}   & = & \epsilon_7 \LiAllOut, \;\;\epsilon_7\simeq 0.924\\
\end{array}
\right. 
\end{equation}
If we use the normalised values:
\begin{equation}
\label{eq:beta}
\left\lbrace
\begin{array}{rcl}
	\beta_6 & = & \dfrac{\Li{6}}{\LiOut{6}}\\
	\\
	\beta_7 & = & \dfrac{\Li{7}}{\LiOut{7}}\\
	\\
	\dfrac{\beta_7}{\beta_6} & = & \dfrac{\left[1+10^{-3}\deltaLi\right]}{\left[1+10^{-3}\deltaLiOut\right]}\\
	\\
	\deltaLi & = & 10^3 \left( \left[1+10^{-3}\deltaLiOut\right]\dfrac{\beta_7}{\beta_6} -1 \right)\\
\end{array}
\right.
\end{equation}


\section{Electro-osmotic Intake}
\subsection{Goldman-Hodgkins-Katz (GHK) Flux}
The two species $\spLi{6}$ and $\spLi{7}$ that may leak through the membrane by borrowing some ionic channels:
\begin{equation}
\left\lbrace
\begin{array}{rcl}
	\Theta    & = & e^{\dfrac{-\Faraday V_m }{RT}} \\
	\partial_t \myleak{\Li{u}} & = & k_u \left( \Theta \LiOut{u} - \Li{u}\right)\\
\end{array}
\right.
\end{equation}
According to the GHK theory, both $k_6$ and $k_7$ are the whole cell permeability to $\spLi{6}$ and $\spLi{7}$ respectively,
and should depend on the potential. But in any case, this permeability arises from a diffusive process, and we expect the ratio to be constant:
\begin{equation}
	\dfrac{k_6}{k_7} \simeq \sigma = 1.00229
\end{equation}

\subsection{Potential Regulation}
One the one hand, since this intake is not electroneutral, it shall produces a drift in the cell polarisation. But on the other hand, the cell is able to regulate its potential 
by a mechanism that we shall simply model by a first order return function:
\begin{equation}
	\partial_t V_m = \gamma \left( \partial_t \myleak{\Li{6}} + \partial_t \myleak{\Li{7}} \right) + k_m \left(V_0-V_m\right)
\end{equation}

\section{One Layer}

\subsection{Kinetic Scheme}
{
\Large
\begin{equation}
\boxed{
\begin{array}{ccccc}
 & & E_{06}  &  & \\
 &  \mycolor{6}{\myrotate{45}{$\xrightleftharpoons[\outLi{6},\;d_{06}]{+\spLiOut{6},\;a_{06}}$}} &   & \mycolor{6}{\myrotate{-45}{$\xrightarrow[-\spLi{6}]{+ \spproton, \; k^p_6}$}} &  \\
E_{00}  &  & \xleftarrow{\text{ recycling } k_h } &   & E_{0H} \\
  & \mycolor{7}{\myrotate{-45}{$\xrightleftharpoons[\outLi{7},\;d_{07}]{+\spLiOut{7},\;a_{02}}$}} &   & \mycolor{7}{\myrotate{+45}{$\xrightarrow[-\spLi{7}]{+ \spproton, \; k^p_7}$}} & \\
 & & E_{07} & & \\
 \end{array}
 }
\end{equation}
}

\subsection{Differential Equations}

\begin{equation}
\left\lbrace
\begin{array}{rcl}
\partial_t E_{00} & = & D_{06}-A_{06} + D_{07}-A_{07} + v_h\\
\partial_t E_{06} & = & -D_{06}+A_{06} -v^p_6 \\
\partial_t E_{07} & = & -D_{07}+A_{07} -v^p_7\\
\partial_t E_{0H} & = & v^p_6 + v^p_7 - v_h\\
\mathsf{E}       & = & E_{00}+E_{01}+E_{02} + E_{0H} = E_{0H} + {\displaystyle \sum_{x\leq y\leq 1} E_{xy}}\\
\end{array}
\right.
\end{equation}
with
\begin{equation}
\left\lbrace
\begin{array}{rcl}
A_{06} &= &a_{06} E_{00} \LiOut{6}\\
A_{07} &= &a_{07} E_{00} \LiOut{7}\\
D_{06} &= &d_{06} E_{06}\\
D_{07} &= &d_{07} E_{07}\\
v^p_6  &=& k^p_6 E_{06} \proton\\
v^p_7  &=& k^p_7 E_{07} \proton\\
v_h    &=& k_h   E_{0H}
\end{array}
\right.
\end{equation}

\subsection{Pre-Equilibria Hypothesis}
\subsubsection{Lithium Intake Rate}
We assume that the reversible forms some pre-equilibria before allowing the irreversible reactions to occur.
\begin{equation}
\left\lbrace
\begin{array}{rcll}
	E_{06}^\star & = & \dfrac{a_{06}}{d_{06} } E_{00} \LiOut{6} & = J_6 E_{00} \LiOut{6}\\
	\\
	E_{07}^\star & = & \dfrac{a_{07}}{d_{07}  } E_{00} \LiOut{7} & = J_7 E_{00} \LiOut{7}\\
\end{array}
\right.
\end{equation}


Finally:
\begin{equation}
\mathsf{E} = E_{00}\left(1+J_6\LiOut{6}+J_7\LiOut{7}\right) + E_{0H} \Leftrightarrow E_{00} = \dfrac{\mathsf{E}-E_{0H}}{1+J_6\LiOut{6}+J_7\LiOut{7}}
\end{equation}
and:
\begin{equation}
\left\lbrace
\begin{array}{rcl}
	E_{06}^\star & = & \dfrac{J_6\LiOut{6}}{1+J_6\LiOut{6}+J_7\LiOut{7}} \left(\mathsf{E}-E_{0H}\right)\\
	\\
	E_{07}^\star & = & \dfrac{J_7\LiOut{7}}{1+J_6\LiOut{6}+J_7\LiOut{7}} \left(\mathsf{E}-E_{0H}\right)\\
\end{array}
\right.
\end{equation}
so that we have the three equations:
\begin{equation}
\left\lbrace
\begin{array}{rcl}
	\partial_t \Li{u}  & = & v^p_u +\partial_t \myleak{\Li{u}}  = \dfrac{k^p_u J_u \LiOut{u}}{1+\sum_v J_v \LiOut{v}} \left(\mathsf{E}-E_{0H}\right) \proton + k_u \left( \Theta \LiOut{u} - \Li{u}\right) \\
	\\
	\partial_t E_{0H} & = & -k_h E_{0H} + \dfrac{\sum_v k^p_u J_u \LiOut{u}}{1+\sum_v J_v \LiOut{v}} \left(\mathsf{E}-E_{0H}\right) \proton \\
\end{array}
\right.
\end{equation}
\subsubsection{Normalisation}
We use the definitions of Eq \eqref{eq:beta} and the following:
\begin{equation}
\left\lbrace
\begin{array}{rcl}
%	\LiOut{u} & = & \epsilon_u \LiAllOut\\
%	\\
%	\beta_u & = & \dfrac{\Li{u}}{\LiOut{u}}\\
%	\\
	\alpha  & = & \dfrac{E_{0H}}{\mathsf{E}}\\
	\\
	J_0 & = & \epsilon_6 J_6  + \epsilon_7 J_7 \\
\end{array}
\right.
\end{equation}

We are left with the four equations:
\begin{equation}
\left\lbrace
\begin{array}{rcl}
\partial_t V_m & = & \gamma\LiAllOut \left[\sum_x \epsilon_x k_x \left( \exp\left[ -\frac{\Faraday V_m}{RT}\right] -\beta_x \right)  \right] + k_m\left(V_0-V_m\right)\\
\\
\partial_t \beta_u & = & \mathsf{E} \left[\dfrac{k^p_u J_u}{1+J_0 \LiAllOut}\right] \left(1-\alpha\right) \proton
	 + k_u \left( \exp\left[-\frac{\Faraday V_m}{RT}\right] - \beta_u\right)\\
\\
\partial_t \alpha  & = &  -k_h\alpha + \LiAllOut \left[\dfrac{ \sum_v k^p_v \epsilon_v J_v}{1+J_0\LiAllOut}\right] \left(1-\alpha\right) \proton\\
\end{array}
\right.
\end{equation}

\subsubsection{Initial Isotopic Separation}
The evaluation of the initial yields:
\begin{equation}
\label{eq:level1}
	\left(\dfrac{\beta_7}{\beta_6}\right)^\varnothing \simeq \left(\dfrac{\partial_t \beta_7}{\partial_t\beta_6}\right)^\varnothing
	= \dfrac{\mathsf{E} \left[\dfrac{k^p_7 J_7}{1+J_0 \LiAllOut}\right] \proton_0 + k_7  \Theta_0
	}
	{
	\mathsf{E} \left[\dfrac{k^p_6 J_6}{1+J_0 \LiAllOut}\right] \proton_0 + k_6  \Theta_0
	}
\end{equation}
Let us define:
\begin{equation}
	\kappa = \dfrac{k^p_6 J_6}{k^p_7 J_7}, \;\; \eta = \dfrac{ \mathsf{E} \proton_0 }{ k_7  \Theta_0 } \left[\dfrac{k^p_7 J_7}{1+J_0 \LiAllOut}\right] 
\end{equation}
to obtain:
\begin{equation}
	\left(\dfrac{\beta_7}{\beta_6}\right)^\varnothing = \dfrac{1+\eta}{\eta\kappa+\sigma}
\end{equation}
We see that for a catalytic speedup $\kappa$ greater than the diffusive speedup $\sigma$, we reach some higher isotopic separations (lower $\deltaLi$), depending
on the $\eta$ factor, which is experimentally much greater than one.
Nonetheless, the value of $\eta$ is {\bf decreasing} with $\LiAllOut$, meaning that the isotopic separation is decreasing with respect to the total lithium concentration.
Accordingly, we have to derive a higher order model.

\section{Two Layers Model : Cooperativity}
\subsection{Full Kinetic Scheme}
{
\Large
\begin{equation}
\boxed{
\begin{array}{ccccccc}
 & &        &                                                  & E_{66} & & \\
 & &        & \myrotate{45}{\mycolor{6}{$\xrightleftharpoons[\outLi{6},\;d_{66}]{\inpLi{6},\;a_{66}}$}} & &  \myrotate{-45}{\mycolor{6}{$\xrightarrow[-\spLi{6}]{+ \spproton, \; k^p_{66}}$}}& \\
 & & E_{06} &  & \xleftarrow{\text{ recycling } k^h_6 } & & E_{6H}\\
 &  \myrotate{45}{\mycolor{6}{$\xrightleftharpoons[\outLi{6},\;d_{06}]{\inpLi{6},\;a_{06}}$}} &   & \myrotate{-45}{\mycolor{7}{$\xrightleftharpoons[\outLi{7},\;d_{67}]{\inpLi{7},\;a_{67}}$}} & & \myrotate{45}{\mycolor{7}{$\xrightarrow[-\spLi{7}]{+ \spproton, \; k^p_{67}}$}}&\\
E_{00} & &  & & E_{67}(=E_{76}) & & \\ 
  & \myrotate{-45}{\mycolor{7}{$\xrightleftharpoons[\outLi{7},\;d_{07}]{\inpLi{7},\;a_{07}}$}} &  & \myrotate{45}{\mycolor{6}{$\xrightleftharpoons[\outLi{6},\;d_{76}]{\inpLi{6},\;a_{76}}$}} & & \myrotate{-45}{\mycolor{6}{$\xrightarrow[-\spLi{6}]{+ \spproton, \; k^p_{76}}$}} & \\
  & & E_{07} &   & \xleftarrow{\text{ recycling } k^h_7 } & & E_{7H}\\
  & &  & \myrotate{-45}{\mycolor{7}{$\xrightleftharpoons[\outLi{7},\;d_{77}]{\inpLi{7},\;a_{77}}$}} & & \myrotate{45}{\mycolor{7}{$\xrightarrow[-\spLi{7}]{+ \spproton, \; k^p_{77}}$}} &\\
  & &  &  & E_{77} & &\\

 \end{array}
 }
\end{equation}
}
with \textit{a priori} no restriction on microscopic values.


\subsection{Differential Equations}
\begin{equation}
\left\lbrace
\begin{array}{rcl}
\partial_t E_{00} & = & D_{06}-A_{06} + D_{07}-A_{07}\\
\partial_t E_{06} & = & -D_{06}+A_{06} - A_{66} + D_{66} - A_{67} + D_{67} + v_{6H}\\
\partial_t E_{07} & = & -D_{07}+A_{07} - A_{77} + D_{77} - A_{76} + D_{76} + v_{7H}\\
\partial_t E_{66} & = & A_{66}-D_{66} -v^p_{66}\\
\partial_t E_{67} & = & A_{67}-D_{67} + A_{76}-D_{76} - (v^p_{67}+v^p_{76})\\
\partial_t E_{77} & = & A_{77}-D_{77} - v^p_{77}\\
\partial_t E_{6H} & = & v^p_{66}+v^p_{67} - v_{6H}\\
\partial_t E_{7H} & = & v^p_{77}+v^p_{76} - v_{7H}\\
\mathsf{E}      & = & {\displaystyle \sum_{x\leq y} E_{xy}}+E_{6H}+E_{7H}\\
\end{array}
\right.
\end{equation}

with
\begin{equation}
\left\lbrace
\begin{array}{rcl}%|rcl}
A_{06}   & = & a_{06} E_{00} \LiOut{6}  \\% & A'_{06}   & = & a_{06}    \alpha_{00} \LiOut{6}\\
A_{07}   & = & a_{07} E_{00} \LiOut{7}  \\%& A'_{07}   & = & a_{07}    \alpha_{00} \LiOut{7}\\
D_{06}   & = & d_{06} E_{06}            \\%& D'_{06}   & = & d_{06}    \alpha_{06}\\
D_{07}   & = & d_{07} E_{07}            \\%& D'_{07}   & = & d_{07}    \alpha_{07}\\
A_{66}   & = & a_{66} E_{06} \LiOut{6}  \\%& A'_{66}   & = & a_{66}    \alpha_{06} \LiOut{6} \\
D_{66}   & = & d_{66} E_{66}            \\%& D'_{66}   & = & d_{66}    \alpha_{66}\\
A_{77}   & = & a_{77} E_{07} \LiOut{7}  \\%& A'_{77}   & = & a_{77}    \alpha_{07} \LiOut{7}\\
D_{77}   & = & d_{77} E_{77}            \\%& D'_{77}   & = & d_{77}    \alpha_{77}\\
A_{67}   & = & a_{67} E_{06} \LiOut{7}  \\%& A'_{67}   & = & a_{67}    \alpha_{06} \LiOut{7}\\
D_{67}   & = & d_{67} E_{67}            \\%& D'_{67}   & = & d_{67}    \alpha_{67}\\
A_{76}   & = & a_{76} E_{07} \LiOut{6}  \\%& A'_{76}   & = & a_{76}    \alpha_{07} \LiOut{6}\\
D_{76}   & = & d_{76} E_{67}            \\%& D'_{76}   & = & d_{76}    \alpha_{67}\\
v_{6H}   & = & k^h_6 E_{6H}             \\%& v'_{6H}   & = & k^h_6    \alpha_{6H} \\
v_{7H}   & = & k^h_7 E_{7H}             \\%& v'_{7H}   & = & k^h_7    \alpha_{7H} \\
v^p_{66} & = & k^p_{66} E_{66} \proton  \\%& v'^p_{66} & = & k^p_{66} \alpha_{66} \proton\\
v^p_{77} & = & k^p_{77} E_{77} \proton  \\%& v'^p_{77} & = & k^p_{77} \alpha_{77} \proton\\
v^p_{67} & = & k^p_{67} E_{67} \proton  \\%& v'^p_{67} & = & k^p_{67} \alpha_{67}\\
v^p_{76} & = & k^p_{76} E_{67} \proton  \\%& v'^p_{76} & = & k^p_{76} \alpha_{67} \proton\\
\end{array}
\right.
\end{equation}

And
\begin{equation}
\left\lbrace
\begin{array}{rcl}
	\partial_t \Li{6} & = & v^p_{66}+v^p_{76} + \partial_t \myleak{\Li{6}} \\ %& = & \partial_t \myleak{\Li{6}} + \mathsf{E} \left( v'^p_{66}+v'^p_{76}\right) \\
	\\
	\partial_t \Li{7} & = & v^p_{77}+v^p_{67} + \partial_t \myleak{\Li{7}} \\ %& = & \partial_t \myleak{\Li{7}} + \mathsf{E} \left( v'^p_{66}+v'^p_{76}\right)\\
\end{array}
\right.
\end{equation}

\subsection{Algebraic Notations}



We define:
\begin{itemize}
\item the first layer variables:
\begin{equation}
	\vec{E}_1 = \begin{bmatrix}
	E_{06}\\
	E_{07}\\
	\end{bmatrix}
\end{equation}
\item and the second layer variables:
\begin{equation}
	\vec{E}_2 = \begin{bmatrix}
	E_{66}\\
	E_{67}\\
	E_{77}\\
	\end{bmatrix}
\end{equation}
\item the outer concentrations:
\begin{equation}
\label{eq:C}
	\vec{C} = 
	\begin{bmatrix}
	\LiOut{6}\\
	\LiOut{7}\\
	\end{bmatrix}
	=
	\LiAllOut
	\begin{bmatrix}
	\epsilon_6\\
	\epsilon_7\\
	\end{bmatrix}
	=
	\LiAllOut\vec{\epsilon},\;\;
	\mymat{\Delta}_C = 
	\begin{bmatrix}
	C_6 & 0\\
	 0& C_7\\
	\end{bmatrix}
\end{equation}
\item the protonated enzymes:
\begin{equation}
	\vec{E}_H = 
	\begin{bmatrix}
	E_{6H}\\
	E_{7H}\\
	\end{bmatrix}
\end{equation}
\item the algebraic projection terms:
\begin{equation}
	\vec{L}_1 = 
	\begin{bmatrix}
	1\\
	1\\
	\end{bmatrix},\;\;
	\mymat{Q}_1 = 
	\begin{bmatrix}
	1&1\\
	1&1\\
	\end{bmatrix}
\end{equation}

\end{itemize}

\subsection{Second layer as a function of the first layer}

Using the three equations describing the second layer and assuming that the internal pre-equilibria are reached, we find
that the second layer components are bilinear combinations of the first layer components:
\begin{equation}
\boxed{
E_{xy} = <{\vec{C}} \vert \mymat{F_{xy}} \vert \vec{E}_1 > 
}
\end{equation}
with:
\begin{equation}
\left\lbrace
\begin{array}{rcl}
\mymat{F}_{66} & = & 
\begin{bmatrix}
	\dfrac{a_{66}}{d_{66} } & 0 \\
	0 & 0\\
\end{bmatrix} \\
\\
\mymat{F}_{77} & = & 
\begin{bmatrix}
	0 & 0 \\
	0 & \dfrac{a_{77}}{d_{77} }\\
\end{bmatrix}  \\
\\
\mymat{F}_{67} & = & 
\begin{bmatrix}
	0 &\dfrac{a_{76}}{d_{76}+d_{67}}\\
	\dfrac{a_{67}}{d_{76}+d_{67}} & 0\\
\end{bmatrix} \\
\end{array}
\right.
\end{equation}
and the second layer mass as a function of the first layer values is:
\begin{equation}
E_{66} + E_{67} + E_{77} = <{\vec{C}} \vert \mymat{F} \vert \vec{E}_1 >, \;\;
 \mymat{F} 
 =  \mymat{F}_{66} + \mymat{F}_{67}  + \mymat{F}_{77}
\end{equation}

We also get a compact writing of the lithium intake:
\begin{equation}
\left\lbrace
\begin{array}{rcl}
	\partial_t \Li{6} & = & <\vec{C}|\mymat{G}_6|\vec{E}_1> \proton+ k_6\left(\Theta \LiOut{6} - \Li{6}\right)\\
	\mymat{G}_6 &= & k^p_{66} \mymat{F}_{66} + k^p_{67}\mymat{F}_{67}	\\
	\\
	\partial_t \Li{7} & = & <\vec{C}|\mymat{G}_7|\vec{E}_1> \proton + k_7\left(\Theta \LiOut{7} - \Li{7}\right)\\
	\mymat{G}_7 & = & k^p_{77} \mymat{F}_{77} + k^p_{76}\mymat{F}_{67} \\
\end{array}
\right.
\end{equation}


\subsection{First layer as a function of the external components}
We define the following constants:
\begin{equation}
\left\lbrace
\begin{array}{rcl}
\mymat{\Delta}_A &= &\begin{bmatrix}
a_{06} & 0 \\
0 & a_{07} \\
\end{bmatrix}\\
\\
\mymat{\Delta}_D & = & 
\begin{bmatrix}
d_{06} & 0 \\
0 & d_{07} \\
\end{bmatrix} \\
\\
K_6 & = & \dfrac{1}{d_{07}d_{06}}  \dfrac{a_{76}d_{67}}{d_{67}+d_{76}} \\
\\
K_7 & = &  \dfrac{1}{d_{06}d_{07}} \dfrac{a_{67}d_{76}}{d_{67}+d_{76}} \\
\\
\mymat{\Delta}_K & = & \begin{bmatrix} K_6 & 0 \\ 0 & K7 \\ \end{bmatrix}
\end{array}
\right.
\end{equation}
Leading to the algebraic expressions:
\begin{equation}
\left\lbrace
\begin{array}{rcl}
W_1 & = & 1 +  <\vec{L}_1| \mymat{\Delta}_K \mymat{\Delta}_D | \vec{C}> = 1+ <\vec{L}_w|\vec{C}> \\
\\
\mymat{M_1} & = & \mymat{\Delta}_D^{-1} + \mymat{\Delta}_C \mymat{\Delta}_K \mymat{Q}_1 \\
\end{array}
\right.
\end{equation}
Thus, the first layer expression as a function of the entry layer is
\begin{equation}
\boxed
{
W_1 \vec{E}_1  =  E_{00} \mymat{M}_1  \mymat{\Delta}_A \vec{C}
}
\end{equation}

\subsection{Using Mass Conservation to compute the entry layer}
We expand and factorise:
\begin{equation}
\left\lbrace
\begin{array}{rcl}
\mathsf{E}    & = & E_{00} 
+ \underbrace{E_{11}+E_{12}+E_{22}}_{<\vec{C}|\mymat{F}|\vec{E}_1>} 
+ \underbrace{E_{01}+E_{02}}_{<\vec{L}_1|\vec{E}_1>} 
+ \underbrace{E_{6H} + E_{7H}}_{<\vec{L}_1|\vec{E}_H>}\\
\\
W_1 \mathsf{E} & = & W_1 E_{00} + < \vec{L}_1 + \mytrn{\mymat{F}}\vec{C} | W_1 \vec{E}_1 > + <W_1 \vec{L}_1|\vec{E}_H> \\
& = & W_1 E_{00} + < \vec{L}_1 + \mytrn{\mymat{F}}\vec{C} | E_{00} \mymat{M}_1  \mymat{\Delta}_A \vec{C} > + <W_1 \vec{L}_1|\vec{E}_H> \\
& = & E_{00} \left[ W_1 + < \vec{L}_1 + \mytrn{\mymat{F}}\vec{C} | \mymat{M}_1  \mymat{\Delta}_A \vec{C} > \right] + W_1 <\vec{L}_1 | \vec{E}_H>
\end{array}
\right.
\end{equation}
We define the cubic form:
\begin{equation}
\left\lbrace
\begin{array}{rcl}
	W_3 & = & < \vec{L}_1 + \mytrn{\mymat{F}}\vec{C} | \mymat{M}_1  \mymat{\Delta}_A \vec{C} >\\
	\\
	& = & \displaystyle <\vec{L}_3|\vec{C}> + <\vec{C}|\mymat{Q}_3|\vec{C}> + \sum_{i+j=3}C_6^i C_7^j Z_{ij}  \\
\end{array}
\right.
\end{equation}
Thus, the entry concentration amounts to:
\begin{equation}
\boxed{
 E_{00}  = \dfrac{W_1}{W_1 + W_3} \left[\mathsf{E}-(E_{6H}+E_{7H})\right]
 }
\end{equation}

\subsection{Expressing the Rates}
We deduce the rate for the for involved species. By factorising the terms by $\LiOut{x}$, we write the result using some vectors 
$\vec{L}_x$ and some matrix $\mymat{Q}_x$:
\begin{equation}
\left\lbrace
\begin{array}{rcl}
\partial_t \Li{x} & = & \dfrac{E_{00}}{W_1}<\vec{C}|\mymat{G}_x|\mymat{M}_1\mymat{\Delta}_A|\vec{C}> \proton+ k_x\left(\Theta \LiOut{x} - \Li{x}\right)\\
\\
 & = & \dfrac{E_{00}}{W1} \proton \left( <\vec{L}_x|\vec{C}> + <\vec{C}|\mymat{Q}_x|\vec{C}>\right) \LiOut{x}+ k_x\left(\Theta \LiOut{x} - \Li{x}\right)\\
 \\
 \partial_t E_{xH} & = & <\vec{C}|\mymat{G}_6|\vec{E}_1> \proton - k_x^h E_{xH}\\
 \\
 & = & \dfrac{E_{00}}{W_1} \proton \left( <\vec{L}_x|\vec{C}> + <\vec{C}|\mymat{Q}_x|\vec{C}>\right) \LiOut{x} - k_x^h E_{xH} \\
\end{array}
\right.
\end{equation}
We now write the adimensional variable rates, using the natural definitions:
\begin{equation}
\alpha_x = \dfrac{E_{xH}}{\mathsf{E}}
\end{equation}
and the almost final equations for the concentration ratios are:
\begin{equation}
\left\lbrace
\begin{array}{rcl}
	\partial_t \beta_x & = & \proton \left( <\vec{L}_x|\vec{C}> + <\vec{C}|\mymat{Q}_x|\vec{C}> \right)  \dfrac{E_{00}}{W_1} + k_x \left(\Theta - \beta_x\right)\\
	\\
	\partial_t \alpha_{x} & = &  \dfrac{E_{00}}{\mathsf{E}W_1} \proton \left( <\vec{L}_x|\vec{C}> + <\vec{C}|\mymat{Q}_x|\vec{C}>\right) \LiOut{x} - k_x^h \alpha_{x}\\
\end{array}
\right.
\end{equation}
and finally:
\begin{equation}
\left\lbrace
\begin{array}{rcl}
\partial_t V_m & = & \gamma\LiAllOut \left[\sum_x \epsilon_x k_x \left( \exp\left[ -\frac{\Faraday V_m}{RT}\right] -\beta_x \right)  \right] + k_m\left(V_0-V_m\right)\\
\\
	\partial_t \beta_x & = & 
	\dfrac{\mathsf{E}\proton}{W_1+W_3} \left[1-(\alpha_{6}+\alpha_{7})\right] \left( <\vec{L}_x|\vec{\epsilon}> \LiAllOut + <\vec{\epsilon}|\mymat{Q}_x|\vec{\epsilon}>\LiAllOut^2\right) 
	+ k_x \left(\exp\left[-\frac{\Faraday V_m}{RT}\right] - \beta_x\right)\\
	\\
	\partial_t \alpha_{x} & = &  \epsilon_x  \dfrac{\LiAllOut \proton}{W_1+W_3} \left[1-(\alpha_{6}+\alpha_{7})\right] \left( <\vec{L}_x|\vec{\epsilon}> \LiAllOut + <\vec{\epsilon}|\mymat{Q}_x|\vec{\epsilon}>\LiAllOut^2\right) 
	- k_x^h \alpha_{x}
	\\
	\\
	W_1 & = & 1 +  <\vec{L}_w | \vec{\epsilon}> \LiAllOut \\
	\\
	W_3 & = & \displaystyle <\vec{L}_3|\vec{\epsilon}> \LiAllOut + <\vec{\epsilon}|\mymat{Q}_3|\vec{\epsilon}> \LiAllOut^2+ \left(\sum_{i+j=3}\epsilon_6^i \epsilon_7^j Z_{ij}\right) \LiAllOut^3 \\
\end{array}
\right.
\end{equation}

\subsection{Conclusion on boundary values}
We observe that:
\begin{equation}
\alpha_x^\infty \propto \epsilon_x
\end{equation}
so, morphologically, the two-layers model shall produce the same curves than the one-layer model.
But if we evaluate:
\begin{equation}
	\left.\dfrac{\beta_7}{\beta_6}\right\vert_\varnothing = 
	\dfrac{ \dfrac{\mathsf{E}\proton\mathsf{N}_7(\LiAllOut)}{1+\mathsf{D}(\LiAllOut)}+k_7 \Theta}
	{  \dfrac{\mathsf{E}\proton\mathsf{N}_6(\LiAllOut)}{1+\mathsf{D}(\LiAllOut)} +k_6 \Theta}
\end{equation}
where
\begin{itemize}
\item ${\mathsf{N}_6}$ and  ${\mathsf{N}_7}$ are  {\bf second} order polynomials without constant term
\begin{equation}
\mathsf{N}_x = <\vec{L}_x|\vec{\epsilon}> \LiAllOut + <\vec{\epsilon}|\mymat{Q}_x|\vec{\epsilon}>\LiAllOut^2
\end{equation}
\item ${\mathsf{D}}$ is a {\bf third } order polynomial without constant term
\end{itemize}
there shall exist a set of parameters which allows the initial $\delta_7$ to decrease with $\LiAllOut$, which is
strictly impossible for the one-layer model.
We have the compact form:
\begin{equation}
\boxed{
\left\lbrace
\begin{array}{rcl}
	\partial_t V_m & = & \gamma\LiAllOut \left[\sum_x \epsilon_x k_x \left( \Theta -\beta_x \right)  \right] + k_m\left(V_0-V_m\right)\\
	\\
	\partial_t \beta_x  & = & \mathsf{E}\proton \dfrac{\mathsf{N}_x}{1+\mathsf{D}} \left[ 1-\alpha_6-\alpha_7\right] + k_x \left(\Theta-\beta_x\right)\\
	\\
	\partial_t \alpha_x & = & \epsilon_x \LiAllOut \proton \dfrac{\mathsf{N}_x}{1+\mathsf{D}} \left[ 1-\alpha_6-\alpha_7\right] - k^h_x \alpha_x \\
\end{array}
\right.
}
\end{equation}

\section{Independency hypothesis}
\subsection{Simplification}

If we assume that the microscopic rates depends only on the "naked" lithium and not on the protein state:
\begin{equation}
\label{eq:indep}
\left\lbrace
\begin{array}{rcl}
k^h_6    & = & k^h_7 = k_h\\
\\
a_{76}   & = & a_{66} = a_{06} = a^r_6\\
d_{76}   & = & d_{66} = d_{06} = d^r_6\\
k^p_{76} & = & k^p_{66} = i_6\\
\\
a_{67}   & = & a_{77} = a_{07} = a^r_7 \\
d_{67}   & = & d_{77} = d_{07} = d^r_7\\
k^p_{67} & = & k^p_{77} = i_7\\
\end{array}
\right.
\end{equation}
We also assume that:
\begin{itemize}
\item the {\bf detachment} speed up is the same, meaning that
\begin{equation}
\left\lbrace
\begin{array}{rcl}
	i_6 & = & \rho \; i_7\\
	d^r_6 & = & \rho \; d^r_7\\
\end{array}
\right.
\end{equation}
\item the {\bf formation} ratio are
\begin{equation}
\left\lbrace
\begin{array}{rcl}
	a^r_6 & = & u_6 \; d^r_6\\
	a^r_7 & = & u_7 \; d^r_7 \\
	u_6   & = & \kappa u_7 \\
\end{array}
\right.
\end{equation}
\end{itemize}

\subsection{New expressions}
We obtain:
\begin{equation}
\vec{L}_7 = i_7 u_7^2 
\underbrace{
\begin{bmatrix}
	\kappa \rho \\
	1\\
\end{bmatrix}}_{\vec{L}'} ,\;\;
\mymat{Q_7} = \dfrac{i_7u^3}{1+\rho} |\vec{L}'><\vec{L}'|,\;\; \vec{L}_6 = \kappa \vec{L}_7, \;\; \mymat{Q}_6 = \kappa \mymat{Q}_7
\end{equation}
and
\begin{equation}
	\vec{L}_w = \dfrac{u_7}{1+\rho} \vec{L}',\;\;
	\vec{L}_3 = u_7 \underbrace{\begin{bmatrix} \kappa \\ 1 \\ \end{bmatrix}}_{\vec{\tilde{L}}},\;\;
	\mymat{Q}_3 = u_7^2
	\underbrace{ 
	\begin{bmatrix}
	\dfrac{2\rho+1}{\rho+1} \kappa^2 & \kappa \\
	\kappa & \dfrac{\rho+2}{\rho+1}\\
	\end{bmatrix}
	}_{\mymat{\tilde{Q}}}
\end{equation}
and we have the simplified expressions:
\begin{equation}
\left\lbrace
\begin{array}{rcl}
W_1 & = & 1 + \dfrac{\left(u_7\LiAllOut\right)}{1+\rho} <\vec{L}'|\vec{\epsilon}>\\
\\
W_3 & = & \left(u_7\LiAllOut\right) <\vec{\tilde{L}}|\vec{\epsilon}> + \left(u_7\LiAllOut\right)^2 <\vec{\epsilon}|\mymat{\tilde{Q}}|\vec{\epsilon}> + \left(u_7\LiAllOut\right)^3 \tilde{Z}_3\\
\\
\tilde{Z}_3 & = & \dfrac{1}{1+\rho} \left[ \epsilon_7^3 + (1+\rho) \left[ \epsilon_7^2\left(\kappa\epsilon_6\right) +  \epsilon_7\left(\kappa\epsilon_6\right)^2 \right]+ \rho \left(\kappa \epsilon_6\right)^3\right] \\
\\
\mathsf{N}_7 & = & i_7 u_7  \left[ \left(u_7\LiAllOut\right) <\vec{L}'|\vec{\epsilon}> + \dfrac{1}{1+\rho} \left(u_7\LiAllOut\right)^2 <\vec{L}'|\vec{\epsilon}>^2 \right]\\
\\
\mathsf{N}_6 & = & \kappa \mathsf{N}_7\\
\end{array}
\right.
\end{equation}

\end{document}


