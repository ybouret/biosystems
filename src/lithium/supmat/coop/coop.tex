\documentclass[aps,onecolumn,11pt]{revtex4}
\usepackage{graphicx}
\usepackage{amssymb,amsfonts,amsmath,amsthm}
\usepackage{chemarr}
\usepackage{bm}
\usepackage{pslatex}
\usepackage{xfrac}
\usepackage[dvipsnames]{xcolor}
\usepackage{bookman}
\usepackage{dsfont}
%\usepackage{mathptmx}
%usepackage{hyperref}
%\usepackage{rotating}
\usepackage{fancybox}

\newcommand{\mychem}[1]{\mathtt{#1}}
\newcommand{\myconc}[1]{\big[#1\big]}

\newcommand{\Faraday}{\mathrm{F}}
\newcommand{\spx}{\mychem{X}}
\newcommand{\spLi}[1]{{\!~^{#1}\mychem{Li}}}
\newcommand{\Li}[1]{\myconc{\spLi{#1}}}
\newcommand{\spproton}{\mychem{H}}
\newcommand{\proton}{\myconc{\spproton}}

\newcommand{\myleak}[1]{\left.{#1}\right\vert_{\mathrm{leak}}}
\newcommand{\myout}[1]{{#1}_{\tiny\mathrm{out}}}
\newcommand{\LiOut}[1]{\myout{\Li{#1}}}
\newcommand{\spLiOut}[1]{\myout{\spLi{#1}}}

\newcommand{\myrotate}[2]{\rotatebox[origin=c]{#1}{#2}}

\newcommand{\mytrn}[1]{{#1}^{\!\mathsf{T}}}
\newcommand{\mymat}[1]{{\bm{#1}}}
\newcommand{\mydet}[1]{{\left|{#1}\right|}}

\newcommand{\LiAll}{\Lambda}
\newcommand{\LiAllOut}{\myout{\LiAll}}


\begin{document}
\title{Cooperativity}
\maketitle
%\tableofcontents


\section{Passive Behaviour}
We have two species $\spLi{6}$ and $\spLi{7}$ that may leak through the membrane:
\begin{equation}
	\partial_t \myleak{\Li{u}} = k_u \left( e^{\dfrac{-\Faraday V_m }{RT}} \LiOut{u} - \Li{u}\right)
\end{equation}

\section{One Layer}

\subsection{Scheme}
\begin{equation}
\boxed{
\begin{array}{ccccc}
 & & E_{06}  &  & \\
 &  \myrotate{45}{$\xrightleftharpoons[d_{06}]{+\spLiOut{6},\;a_{06}}$} &   & \myrotate{-45}{$\xrightarrow[-\spLi{6}]{+ \spproton, \; k^p_6}$} &  \\
E_{00}  &  & \xleftarrow{\text{ recycling } k_h } &   & E_{0H} \\
  & \myrotate{-45}{$\xrightleftharpoons[d_{07}]{+\spLiOut{7},\;a_{02}}$} &   & \myrotate{+45}{$\xrightarrow[-\spLi{7}]{+ \spproton, \; k^p_7}$} & \\
 & & E_{07} & & \\
 \end{array}
 }
\end{equation}

\subsection{Equations}

\begin{equation}
\left\lbrace
\begin{array}{rcl}
\partial_t E_{00} & = & D_{06}-A_{06} + D_{07}-A_{07} + v_h\\
\partial_t E_{06} & = & -D_{06}+A_{06} -v^p_6 \\
\partial_t E_{07} & = & -D_{07}+A_{07} -v^p_7\\
\partial_t E_{0H} & = & v^p_6 + v^p_7 - v_h\\
\mathsf{E}       & = & E_{00}+E_{01}+E_{02} + E_{0H} = E_{0H} + {\displaystyle \sum_{x\leq y\leq 1} E_{xy}}\\
\end{array}
\right.
\end{equation}
with
\begin{equation}
\left\lbrace
\begin{array}{rcl}
A_{06} &= &a_{06} E_{00} \LiOut{6}\\
A_{07} &= &a_{07} E_{00} \LiOut{7}\\
D_{06} &= &d_{06} E_{06}\\
D_{07} &= &d_{07} E_{07}\\
v^p_6  &=& k^p_6 E_{06} \proton\\
v^p_7  &=& k^p_7 E_{07} \proton\\
v_h    &=& k_h   E_{0H}
\end{array}
\right.
\end{equation}

\subsection{Steady-State}
\subsubsection{Raw Equations}
We assume that the internal components are at a steady-state:
\begin{equation}
\left\lbrace
\begin{array}{rcll}
	E_{06}^\star & = & \dfrac{a_{06}}{d_{06}+k^p_6 \proton } E_{00} \LiOut{6} & = J_6 E_{00} \LiOut{6}\\
	\\
	E_{07}^\star & = & \dfrac{a_{07}}{d_{07}+k^p_7 \proton } E_{00} \LiOut{7} & = J_7 E_{00} \LiOut{7}\\
\end{array}
\right.
\end{equation}

\centerline{\shadowbox{\it We expect a simplification if $d_{01},d_{02} \gg k^p_1 \proton, k^p_2 \proton$.}}

Finally:
\begin{equation}
\mathsf{E} = E_{00}\left(1+J_6\LiOut{6}+J_7\LiOut{7}\right) + E_{0H} \Leftrightarrow E_{00} = \dfrac{\mathsf{E}-E_{0H}}{1+J_6\LiOut{6}+J_7\LiOut{7}}
\end{equation}
and:
\begin{equation}
\left\lbrace
\begin{array}{rcl}
	E_{06}^\star & = & \dfrac{J_6\LiOut{6}}{1+J_6\LiOut{6}+J_7\LiOut{7}} \left(\mathsf{E}-E_{0H}\right)\\
	\\
	E_{07}^\star & = & \dfrac{J_7\LiOut{7}}{1+J_6\LiOut{6}+J_7\LiOut{7}} \left(\mathsf{E}-E_{0H}\right)\\
\end{array}
\right.
\end{equation}
so that we have the three equations:
\begin{equation}
\left\lbrace
\begin{array}{rcl}
	\partial_t \Li{u}  & = & v^p_u +\partial_t \myleak{\Li{u}}  = \dfrac{k^p_u J_u \LiOut{u}}{1+\sum_v J_v \LiOut{v}} \left(\mathsf{E}-E_{0H}\right) \proton + k_u \left( e^{\frac{-\Faraday V_m }{RT}} \LiOut{u} - \Li{u}\right) \\
	\\
	\partial_t E_{0H} & = & -k_h E_{0H} + \dfrac{\sum_v k^p_u J_u \LiOut{u}}{1+\sum_v J_v \LiOut{v}} \left(\mathsf{E}-E_{0H}\right) \proton \\
\end{array}
\right.
\end{equation}
\subsubsection{Normalisation}
\begin{equation}
\left\lbrace
\begin{array}{rcl}
	\Li{u} & = & \epsilon_u \LiAllOut\\
	\\
	\beta_u & = & \dfrac{\Li{u}}{\LiOut{u}}\\
	\\
	\alpha  & = & \dfrac{E_{0H}}{\mathsf{E}}\\
	\\
	J_0 & = & \epsilon_6 J_6  + \epsilon_7 J_7 \\
\end{array}
\right.
\end{equation}

Leading to the three equations:
\begin{equation}
\left\lbrace
\begin{array}{rcl}
	\partial_t \beta_u & = & \mathsf{E} \left[\dfrac{k^p_u J_u}{1+J_0 \LiAllOut}\right] \left(1-\alpha\right) \proton
	 + k_u \left( e^{\dfrac{-z_u F V_m }{RT}}- \beta_u\right)\\
	\\
	\partial_t \alpha  & = &  -k_h\alpha + \LiAllOut \left[\dfrac{ \sum_v k^p_v \epsilon_v J_v}{1+J_0\LiAllOut}\right] \left(1-\alpha\right) \proton\\
\end{array}
\right.
\end{equation}

In particular, with
\begin{equation}
	\Theta = e^{\dfrac{-\Faraday V_m }{RT}}
\end{equation}

\begin{equation}
	\left.\dfrac{\beta_u}{\beta_v}\right\vert_{t\to0} \simeq \left.\dfrac{\partial_t \beta_u}{\partial_t\beta_v}\right\vert_{t\to0}
	= \dfrac{\mathsf{E} \left[\dfrac{k^p_u J_u}{1+J_0 \LiAllOut}\right] \proton_0 + k_u  \Theta_0
	}
	{
	\mathsf{E} \left[\dfrac{k^p_v J_v}{1+J_0 \LiAllOut}\right] \proton_0 + k_v  \Theta_0
	}
\end{equation}


\section{Two Layers}
\subsection{Scheme}

\begin{equation}
\boxed{
\begin{array}{ccccccc}
 & &        &                                                  & E_{66} & & \\
 & &        & \myrotate{45}{$\xrightleftharpoons[d_{66}]{+\spLiOut{6},\;a_{66}}$} & &  \myrotate{-45}{$\xrightarrow[-\spLi{6}]{+ \spproton, \; k^p_{66}}$}& \\
 & & E_{06} &  & \xleftarrow{\text{ recycling } k_h } & & E_{6H}\\
 &  \myrotate{45}{$\xrightleftharpoons[d_{06}]{+\spLiOut{6},\;a_{06}}$} &   & \myrotate{-45}{$\xrightleftharpoons[d_{67}]{+\spLiOut{7},\;a_{67}}$} & & \myrotate{45}{$\xrightarrow[-\spLi{7}]{+ \spproton, \; k^p_{67}}$}&\\
E_{00} & &  & & E_{67}(=E_{76}) & & \\ 
  & \myrotate{-45}{$\xrightleftharpoons[d_{07}]{+\spLiOut{7},\;a_{07}}$} &  & \myrotate{45}{$\xrightleftharpoons[d_{76}]{+\spLiOut{6},\;a_{76}}$} & & \myrotate{-45}{$\xrightarrow[-\spLi{6}]{+ \spproton, \; k^p_{76}}$} & \\
  & & E_{07} &   & \xleftarrow{\text{ recycling } k_h } & & E_{7H}\\
  & &  & \myrotate{-45}{$\xrightleftharpoons[d_{77}]{+\spLiOut{7},\;a_{77}}$} & & \myrotate{45}{$\xrightarrow[-\spLi{7}]{+ \spproton, \; k^p_{77}}$} &\\
  & &  &  & E_{77} & &\\

 \end{array}
 }
\end{equation}

\subsection{Equations}
\begin{equation}
\left\lbrace
\begin{array}{rcl}
\partial_t E_{00} & = & D_{06}-A_{06} + D_{07}-A_{07}\\
\partial_t E_{06} & = & -D_{06}+A_{06} - A_{66} + D_{66} - A_{67} + D_{67} + v_{6H}\\
\partial_t E_{07} & = & -D_{07}+A_{07} - A_{77} + D_{77} - A_{76} + D_{76} + v_{7H}\\
\partial_t E_{66} & = & A_{66}-D_{66} -v^p_{66}\\
\partial_t E_{67} & = & A_{67}-D_{67} + A_{76}-D_{76} - (v^p_{67}+v^p_{76})\\
\partial_t E_{77} & = & A_{77}-D_{77} - v^p_{77}\\
\partial_t E_{6H} & = & v^p_{66}+v^p_{67} - v_{6H}\\
\partial_t E_{7H} & = & v^p_{77}+v^p_{76} - v_{7H}\\
\mathsf{E}      & = & {\displaystyle \sum_{x\leq y} E_{xy}}+E_{6H}+E_{7H}\\
\end{array}
\right.
\end{equation}
with
\begin{equation}
\left\lbrace
\begin{array}{rcl}
A_{06}   & = & a_{06} E_{00} \LiOut{6}\\
A_{07}   & = & a_{07} E_{00} \LiOut{7}\\
D_{06}   & = & d_{06} E_{06}\\
D_{07}   & = & d_{07} E_{07}\\
A_{66}   & = & a_{66} E_{06} \LiOut{6} \\
D_{66}   & = & d_{66} E_{66}\\
A_{77}   & = & a_{77} E_{07} \LiOut{7} \\
D_{77}   & = & d_{77} E_{77}\\
A_{67}   & = & a_{67} E_{06} \LiOut{7}\\
D_{67}   & = & d_{67} E_{67}\\
A_{76}   & = & a_{76} E_{07} \LiOut{6}\\
D_{76}   & = & d_{76} E_{67}\\
v_{6H}   & = & k_h E_{6H}\\
v_{7H}   & = & k_h E_{7H}\\
v^p_{66} & = & k^p_{66} E_{66} \proton \\
v^p_{77} & = & k^p_{77} E_{77} \proton \\
v^p_{67} & = & k^p_{67} E_{67} \proton \\
v^p_{76} & = & k^p_{76} E_{67} \proton \\
\end{array}
\right.
\end{equation}

And
\begin{equation}
	\partial_t \LiOut{6} = v^p_{66}+v^p_{76} + \partial_t \myleak{\LiOut{6}},\;\;
	\partial_t \LiOut{7} = v^p_{77}+v^p_{67} + \partial_t \myconc{\spx_2}\vert_{leak}
\end{equation}

\subsection{Steady State}


\subsubsection{Second layer as a function of the first layer}

We define:
\begin{itemize}
\item the first layer variables:
\begin{equation}
	\vec{E}_1 = \begin{bmatrix}
	E_{06}\\
	E_{07}\\
	\end{bmatrix}
\end{equation}
\item and the second layer variables:
\begin{equation}
	\vec{E}_2 = \begin{bmatrix}
	E_{66}\\
	E_{67}\\
	E_{77}\\
	\end{bmatrix}
\end{equation}
\item the outer concentrations:
\begin{equation}
	\vec{C} = 
	\begin{bmatrix}
	\LiOut{6}\\
	\LiOut{7}\\
	\end{bmatrix}
	=
	\LiAllOut
	\begin{bmatrix}
	\epsilon_6\\
	\epsilon_7\\
	\end{bmatrix}
\end{equation}
\end{itemize}

Using the three equations describing the second layer and assuming that the internal pre-equilibrium is fast, we express:

\begin{equation}
\boxed{
E_{xy} = <{\vec{C}} \vert \mymat{F_{xy}} \vert \vec{E}_1 >
}
\end{equation}
with
\begin{equation}
\left\lbrace
\begin{array}{rcl}
\mymat{F}_{11} & = & 
\begin{bmatrix}
	f_{66} & 0 \\
	0 & 0\\
\end{bmatrix}, \; f_{66} = \dfrac{a_{66}}{d_{66}}\\
\\
\mymat{F}_{77} & = & 
\begin{bmatrix}
	0 & 0 \\
	0 & f_{77}\\
\end{bmatrix}, \; f_{77} = \dfrac{a_{77}}{d_{77}}\\
\\
\mymat{F}_{67} & = & 
\begin{bmatrix}
	0 & f_{76}\\
	f_{67} & 0\\
\end{bmatrix}, \; f_{67} = \dfrac{a_{67}}{d_{76}+d_{67}},\; f_{76} = \dfrac{a_{76}}{d_{76}+d_{67}}\\
\end{array}
\right.
\end{equation}
and the second layer mass is
\begin{equation}
E_{66} + E_{67} + E_{77} = <{\vec{C}} \vert \mymat{F} \vert \vec{E}_1 >, \;\;
 \mymat{F} 
 = \begin{bmatrix}
	f_{66} & f_{76}\\
	f_{67} & f_{77}\\
\end{bmatrix}
\end{equation}
and


\begin{equation}
\left\lbrace
\begin{array}{rcll}
	\partial_t \Li{6} & = & <\vec{C}|\mymat{G}_6|\vec{E}_1> \proton+ k_6\left(\Theta \LiOut{6} - \Li{6}\right), &
	\mymat{G}_6 = 
	\begin{bmatrix}
	f_{66}k^p_{66} & f_{76} k^p_{67} \\
	f_{67}k^p_{67} & 0 \\
	\end{bmatrix}
	\\
	\\
	\partial_t \Li{7} & = & <\vec{C}|\mymat{G}_7|\vec{E}_1> \proton + k_1\left(\Theta \LiOut{6} - \Li{6}\right), &
	\mymat{G}_2 = 
	\begin{bmatrix}
	0              & f_{76} k^p_{76}\\
	f_{67}k^p_{76} & f_{77}k^p_{77}\\
	\end{bmatrix}
	\\
\end{array}
\right.
\end{equation}


\subsubsection{First layer as a function of the external components}
We define the following constants:
\begin{equation}
\left\lbrace
\begin{array}{rcl}
K_6 & = & d_{67} f_{76} \\
K_7 & = &  d_{76} f_{67}\\
\end{array}
\right.
\end{equation}
so that the first layer equation becomes:
	
\begin{equation}
\begin{bmatrix}
d_{06} + C_7 K_7 & -C_6K_6\\
-C_7K_7 & d_{07} + C_6 K_6 \\
\end{bmatrix}
\vec{E}_1 = k_h \vec{E}_H 
+ E_{00}
\underbrace{
\begin{bmatrix}
a_{06} & 0 \\
0 & a_{07} \\
\end{bmatrix}
}_{\mymat{A}_0}
\vec{C}
\end{equation}

\end{document}

%%%

\mathfrak{D} & = & d_{01} d_{02}+ C_2 K_2 d_{02}+C_1K_1 d_{01}\\
\\
\mymat{M}_1  & = &  \begin{bmatrix}
   d_{02} + K_1 C_1 & K_1 C_1 \\
   K_2C_2 & d_{01} + K_2 C_2 \\
 \end{bmatrix} =  \begin{bmatrix}
   d_{02}  &  0 \\
   0  & d_{01}   \\
   \end{bmatrix} + |\vec{C}><\vec{K}|
\\
\\
\vec{E}_H & = & \begin{bmatrix}
 E_{1H}\\
 E_{2H}\\
 \end{bmatrix}\\
 \\
 \mymat{A}_0 & = & \begin{bmatrix}
 a_{01} & 0 \\
 0 & a_{02}\\
 \end{bmatrix}\\
\end{array}
\right.
\end{equation}
to write:
\begin{equation}
\boxed{
	\vec{E}_1 =  \dfrac{1}{ \mathfrak{D} }
	\mymat{M}_1	
 \left(
k_h 
 \vec{E}_H
 +E_{00} 
 \mymat{A}_0
 \vec{C}
\right)
}
\end{equation}
and we find that
\begin{equation}
\left.\vec{E}_1\right\vert_{t\to0} =  \dfrac{E_{000}}{ \mathfrak{D} }
	\mymat{M}_1	
 \mymat{A}_0
 \vec{C}
\end{equation}
And we can express
\begin{equation}
	\partial_t \myconc{X_u}_{t\to0}  =   	<\vec{C}|\mymat{G}_u|\vec{E}_{1,t\to0}> + k_u\Theta C_u 
\end{equation}

\subsubsection{Coupling by mass conservation}
We use the vectors
\begin{equation}
	\vec{P}_1 = 
	\begin{bmatrix}
	1\\
	1\\
	\end{bmatrix},
	\;\;
\vec{Q}_1  =  \vec{P}_1 + \mytrn{\mymat{F}}\vec{C} \\
\end{equation}

\begin{equation}
\left\lbrace
\begin{array}{rcl}
\mathfrak{E} & = & E_{00} + \underbrace{E_{01} + E_{02}}_{<\vec{P}_1|\vec{E}_1>} + \underbrace{E_{11} + E_{12} + E_{22}}_{<\vec{C}|\mymat{F}|\vec{E}_1>} + E_{1H} + E_{2H} \\
\\
\mathfrak{E} & = & E_{00} + <\vec{Q}_1|\vec{E}_1> +  <\vec{P}_1|\vec{E}_H>\\
\\
\mathfrak{D}\mathfrak{E} & = & 
\mathfrak{D} E_{00} + 
<\vec{Q}_1|\mymat{M}_1|\left(
	k_h  \vec{E}_H +E_{00} \mymat{A}_0 \vec{C} \right)> +  \mathfrak{D} <\vec{P}_1|\vec{E}_H>\\
	\\
 & = & E_{00} \left( \mathfrak{D} + <\vec{Q}_1|\mymat{M}_1|\mymat{A}_0\vec{C}>\right) + 
\mathfrak{D} <\vec{P}_1|\vec{E}_H> + k_h <\vec{Q}_1|\mymat{M}_1|\vec{E}_H>\\
\\
& = & E_{00} \left( \mathfrak{D} + <\mytrn{\mymat{M}}_1\vec{Q}_1|\mymat{A}_0\vec{C}>\right) + < \mathfrak{D} \vec{P}_1 + k_h\mytrn{\mymat{M}}_1 \vec{Q_1} | \vec{E}_H>\\
\end{array}
\right.
\end{equation}
and
\begin{equation}
	E_{00} = 
	\dfrac
	{\mathfrak{D}\mathfrak{E}-< \mathfrak{D} \vec{P}_1 + k_h\mytrn{\mymat{M}}_1 \vec{Q_1} | \vec{E}_H>}
	{\left( \mathfrak{D} + <\mytrn{\mymat{M}}_1\vec{Q}_1|\mymat{A}_0\vec{C}>\right)}
\end{equation}
and
\begin{equation}
	E_{000} = 
	\dfrac
	{\mathfrak{D}\mathfrak{E}}
	{\left( \mathfrak{D} + <\mytrn{\mymat{M}}_1\vec{Q}_1|\mymat{A}_0\vec{C}>\right)}
\end{equation}
leading to
\begin{equation}
	\partial_t \myconc{X_u}_{t\to0}  =   \dfrac{\mathfrak{E}}{\left( \mathfrak{D} + <\mytrn{\mymat{M}}_1\vec{Q}_1|\mymat{A}_0\vec{C}>\right)}	
	<\vec{C}|\mymat{G}_u|\mymat{M}_1\mymat{A}_0\vec{C}> + k_u\Theta C_u 
\end{equation}
so that
\begin{equation}
	\partial_t {\beta_{u,t\to0}} = \dfrac{B_u C_0}{1+L_u C_0 + W_u C_0^2} + k_u \Theta
\end{equation}

\end{document}
