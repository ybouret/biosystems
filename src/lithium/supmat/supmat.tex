\documentclass[aps,onecolumn,10pt]{revtex4}
\usepackage{graphicx}
\usepackage{amssymb,amsfonts,amsmath,amsthm}
\usepackage{chemarr}
\usepackage{bm}
\usepackage{pslatex}
\usepackage{xfrac}
\usepackage[dvipsnames]{xcolor}
%\usepackage{bookman}
\usepackage{dsfont}
\usepackage{mathptmx}
\usepackage{hyperref}
\usepackage{rotating}

%%%%%%%%%%%%%%%%%%%%%%%%%%%%%%%%%%%%%%%%%%%%%%%%%%%%%%%%%%%%%%%%%%%%%%%%%%%%%%
%%
%%
%% Style
%%
%%
%%%%%%%%%%%%%%%%%%%%%%%%%%%%%%%%%%%%%%%%%%%%%%%%%%%%%%%%%%%%%%%%%%%%%%%%%%%%%%
\newcommand{\mychem}[1]{\mathtt{#1}}
\newcommand{\myconc}[1]{\left\lbrack{#1}\right\rbrack}

\newcommand{\spLi}[1]{{~^{\mychem{#1}}\mychem{Li}}}
\newcommand{\Li}[1]{\myconc{\spLi{#1}}}

\newcommand{\spEout}{\mychem{E}}
\newcommand{\Eout}{\myconc{\spEout}}

%\newcommand{\spLiEin}[1]{\left\lbrace\spLi{#1}\spEout\right\rbrace_{\mathrm{in}}}
%\newcommand{\LiEin}[1]{\myconc{\spLiEin{#1}}}

\newcommand{\spLiE}[1]{\left\lbrace\spLi{#1}\spEout\right\rbrace}
\newcommand{\LiE}[1]{\myconc{\spLiE{#1}}}


%\newcommand{\spLiEout}[1]{\left\lbrace\spLi{#1}\spEout\right\rbrace_{\mathrm{out}}}
%\newcommand{\LiEout}[1]{\myconc{\spLiEout{#1}}}

\newcommand{\spLiIn}[1]{{\spLi{#1}}_{\mathrm{in}}}
\newcommand{\LiIn}[1]{\myconc{\spLiIn{#1}}}

\newcommand{\spLiOut}[1]{{\spLi{#1}}_{\mathrm{out}}}
\newcommand{\LiOut}[1]{\myconc{\spLiOut{#1}}}

\newcommand{\spEHin}{\mychem{EH}}
\newcommand{\EHin}{\myconc{\spEHin}}
\newcommand{\spproton}{\mychem{H}}
\newcommand{\proton}{\myconc{\spproton}}

\newcommand{\mytrn}[1]{{#1}^{\!\mathsf{T}}}
\newcommand{\mymat}[1]{{\bm{#1}}}
\newcommand{\mydet}[1]{{\left|{#1}\right|}}

\newcommand{\ratioLi}{ {\left(\dfrac{\Li{7}}{\Li{6}}\right)} }
\newcommand{\deltaLi}{ {\delta\!\!\!\spLi{7}} }
\newcommand{\deltaLiOut}{{\deltaLi}_{\mathrm{out}}}

\newcommand{\LiAll}{\Lambda}
\newcommand{\LiAllOut}{{\LiAll}_{\mathrm{out}}}

\newcommand{\NHE}[1]{\mychem{NHE}{\!-\!#1}}

%%%%%%%%%%%%%%%%%%%%%%%%%%%%%%%%%%%%%%%%%%%%%%%%%%%%%%%%%%%%%%%%%%%%%%%%%%%%%%
%%
%%
%% Document
%%
%%
%%%%%%%%%%%%%%%%%%%%%%%%%%%%%%%%%%%%%%%%%%%%%%%%%%%%%%%%%%%%%%%%%%%%%%%%%%%%%%


\begin{document}
\title{Supplementary Material...}
\maketitle

\section{Setting up the Model}

\subsection{Description And Notations}
Let us describe how a cell may intake a lithium isotope from the outer medium, which is formally described by the transformation
of $\spLiOut{x}$ into $\spLiIn{x}$ for $x=6,7$. We distinguish two main paths.

\begin{itemize}
\item The Enzymatic path, using $\NHE{1}$ as the enzyme $\spEout$ works as follows:
	\begin{itemize}
	\item
	\item
	\end{itemize}

\item The Passive path (or leak) consists in the use by the lithium species of some ion channels, following the electro-osmotic gradient as described by the Goldman-Hodkins-Katz equations. 

\end{itemize}

\begin{equation}
\left\lbrace
\begin{array}{cccl}
\spLiOut{x} + \spEout & \xrightleftharpoons[~~k^d_x~~]{~~k^a_x~~} & \spLiE{x} & (\text{pre-equilibrium})\\
\\
\spLiE{x} + \spproton & \xrightleftharpoons[~~k^q_x~~]{~~k^p_x~~} & \spEHin + \spLiIn{x} & (\text{Lithium intake and reverse path})\\
\\
\spEHin               & \xrightarrow{~~k_h~~} & \spEout + \spproton_{\mathrm{out}} & (\text{Enzyme recycling})\\
\\
\spLiOut{x}           & \xrightleftharpoons[~~k_x~~]{~~k_x e^{-\frac{FV_m}{RT}}~~} & \spLiIn{x} & (\text{Passive Leak})\\
\end{array}
\right.
\end{equation}

\subsection{Kinetic Scheme}



\end{document}
