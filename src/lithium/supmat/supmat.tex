\documentclass[aps,onecolumn,10pt]{revtex4}
\usepackage{graphicx}
\usepackage{amssymb,amsfonts,amsmath,amsthm}
\usepackage{chemarr}
\usepackage{bm}
\usepackage{pslatex}
\usepackage{xfrac}
\usepackage[dvipsnames]{xcolor}
%\usepackage{bookman}
\usepackage{dsfont}
\usepackage{mathptmx}
\usepackage{hyperref}
\usepackage{rotating}

%%%%%%%%%%%%%%%%%%%%%%%%%%%%%%%%%%%%%%%%%%%%%%%%%%%%%%%%%%%%%%%%%%%%%%%%%%%%%%
%%
%%
%% Style
%%
%%
%%%%%%%%%%%%%%%%%%%%%%%%%%%%%%%%%%%%%%%%%%%%%%%%%%%%%%%%%%%%%%%%%%%%%%%%%%%%%%
\newcommand{\mychem}[1]{\mathtt{#1}}
\newcommand{\myconc}[1]{\left\lbrack{#1}\right\rbrack}

\newcommand{\spLi}[1]{{~^{\mychem{#1}}\mychem{Li}}}
\newcommand{\Li}[1]{\myconc{\spLi{#1}}}

\newcommand{\spEout}{\mychem{E}}
\newcommand{\Eout}{\myconc{\spEout}}

%\newcommand{\spLiEin}[1]{\left\lbrace\spLi{#1}\spEout\right\rbrace_{\mathrm{in}}}
%\newcommand{\LiEin}[1]{\myconc{\spLiEin{#1}}}

\newcommand{\spLiE}[1]{\left\lbrace\spLi{#1}\spEout\right\rbrace}
\newcommand{\LiE}[1]{\myconc{\spLiE{#1}}}


%\newcommand{\spLiEout}[1]{\left\lbrace\spLi{#1}\spEout\right\rbrace_{\mathrm{out}}}
%\newcommand{\LiEout}[1]{\myconc{\spLiEout{#1}}}

\newcommand{\spLiIn}[1]{{\spLi{#1}}_{\mathrm{in}}}
\newcommand{\LiIn}[1]{\myconc{\spLiIn{#1}}}

\newcommand{\spLiOut}[1]{{\spLi{#1}}_{\mathrm{out}}}
\newcommand{\LiOut}[1]{\myconc{\spLiOut{#1}}}

\newcommand{\spEHin}{\mychem{EH}}
\newcommand{\EHin}{\myconc{\spEHin}}
\newcommand{\spproton}{\mychem{H}}
\newcommand{\proton}{\myconc{\spproton}}

\newcommand{\mytrn}[1]{{#1}^{\!\mathsf{T}}}
\newcommand{\mymat}[1]{{\bm{#1}}}
\newcommand{\mydet}[1]{{\left|{#1}\right|}}

\newcommand{\ratioLi}{ {\left(\dfrac{\Li{7}}{\Li{6}}\right)} }
\newcommand{\deltaLi}{ {\delta\!\!\!\spLi{7}} }
\newcommand{\deltaLiOut}{{\deltaLi}_{\mathrm{out}}}

\newcommand{\LiAll}{\Lambda}
\newcommand{\LiAllOut}{{\LiAll}_{\mathrm{out}}}

\newcommand{\NHE}[1]{\mychem{NHE}{\!-\!#1}}
\newcommand{\todo}[1]{\framebox{\textbf{\color{WildStrawberry}{#1}}}}

%%%%%%%%%%%%%%%%%%%%%%%%%%%%%%%%%%%%%%%%%%%%%%%%%%%%%%%%%%%%%%%%%%%%%%%%%%%%%%
%%
%%
%% Document
%%
%%
%%%%%%%%%%%%%%%%%%%%%%%%%%%%%%%%%%%%%%%%%%%%%%%%%%%%%%%%%%%%%%%%%%%%%%%%%%%%%%


\begin{document}
\title{Supplementary Material...}
\maketitle

\section{Setting up the Model}

\subsection{Description And Notations}
Let us describe how a cell may intake a lithium isotope from the outer medium, which is formally described by the transformation
of $\spLiOut{x}$ into $\spLiIn{x}$ for $x=6,7$. We distinguish two main paths.

\begin{itemize}
\item We assume that the Enzymatic Path, using $\NHE{1}$ as the enzyme $\spEout$ works as follows. A more detailed description is found in \todo{ref}.
	\begin{itemize}
	\item The outer lithium is associated with a membrane enzyme:
	\begin{equation}
		\label{eq:pre}
		\spLiOut{x} + \spEout  \xrightleftharpoons[~~k^d_x~~]{~~k^a_x~~} \spLiE{x},
	\end{equation}
	where $k^a_x$ and $k^d_x$ respectively are the association and the dissociation rate constants.
	
	\item Then the associated complex exchange an inner proton with a lithium ion:
	\begin{equation}
		\label{eq:xch}
		\spLiE{x} + \spproton   \xrightleftharpoons[~~k^q_x~~]{~~k^p_x~~}   \spEHin + \spLi{x}, 
	\end{equation}
	where $k^p_x$ and $k^q_x$ respectively are the forward and reverse protonation rate constants.
	
	\item The attached inner proton is then transported out of the cell to recycle the enzyme:
	\begin{equation}
			\spEHin   \xrightarrow{~~k_h~~}   \spEout + \spproton_{\mathrm{out}} 
	\end{equation}	
	where $k_h$ is the apparent recycling constant rate, which may vary with $\proton$.
	\end{itemize}

\item The Passive Path (or Leak) consists in the use by the lithium species of some ion channels, following the electro-osmotic gradient as described by the Goldman-Hodkins-Katz \todo{ref} equations and leading to:
	\begin{equation}
	\label{eq:ghk}
		\spLiOut{x} \xrightleftharpoons[~~k_x~~]{~~k_x e^{-\frac{FV_m}{RT}}~~}   \spLi{x} 
	\end{equation}
where $V_m$ is the membrane potential, $F$ is the Faraday's constant, $R$ is the perfect gas constant, and $T$ is the absolute temperature.

\end{itemize}

\subsection{Full Kinetic Scheme}
\subsubsection{Lithium dependent species}
We define the 11 following rates:
\begin{equation}
	\label{eq:rates}
\left\lbrace
\begin{array}{rcl}
	v^a_x & = & k^a_x \Eout \LiOut{x} \\
	v^d_x & = & k^d_x \LiE{x} \\
	v^p_x & = & k^p_x \LiE{x} \proton\\
	v^q_x & = & k^q_x \Li{x} \EHin\\
	v_h   & = & k_x \EHin\\
	v^l_x & = & k_x\left(\Theta \LiOut{x} - \Li{x}\right),\;\;\Theta = e^{-\frac{FV_m}{RT}}\\
\end{array}
\right.
\end{equation}
We deduce the full constrained kinetic scheme:
\begin{equation}
	\label{eq:full}
\left\lbrace
\begin{array}{rcl}
\partial_t \EHin & = & \\
\end{array}
\right.
\end{equation}

\subsubsection{Proton Kinetics}
...

\subsubsection{Potential Kinetics}
...

\subsection{Semi-Stationary Kinetic Scheme}


\end{document}
